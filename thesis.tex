%%%%%%%%%%%%%%%%%%%%%%%%%%%%%%%%%%%%%%%%%
%This documentclass loads the packages  %
%            setspace                   %
% and                                   %
%           fancyhdr.                   %
%You may have to                        %
%get these if your TeX distribution     %
%doesn't have them.                     %
%%%%%%%%%%%%%%%%%%%%%%%%%%%%%%%%%%%%%%%%%
\documentclass{ttuthes2007}
%%%%%%%%%%%%%%%%%%%%%%%%%%%%%%%%%%%%%%%%%%%%%%
%Include any other add-on  packages you need:%
%%%%%%%%%%%%%%%%%%%%%%%%%%%%%%%%%%%%%%%%%%%%%%
%\usepackage[utf8]{inputenc}
\usepackage[T1]{fontenc}
\usepackage{lmodern}
%\usepackage[labelfont=bf,labelsep=period]{caption}
\usepackage{multirow}
\usepackage{textcomp}
%\usepackage[colorlinks=true,citecolor=black,linkcolor=black]{hyperref}
\usepackage{afterpage}
\usepackage{pdflscape}
\usepackage{hhline}
\usepackage{enumitem}
\usepackage{amsmath,graphicx,bm,mathtools}
\usepackage{acronym,setspace,natbib,float,blindtext,hyperref,caption,siunitx,tikz} 

\newcommand{\colvec}[2][.8]{%
  \scalebox{#1}{%
    \renewcommand{\arraystretch}{.8}%
    $\begin{Bmatrix}#2\end{Bmatrix}$%
  }
}

\newcommand\tikzmark[2]{%                                                       
\tikz[remember picture,baseline] \node[above, outer sep=0pt, inner sep=0pt]     
(#1){\phantom{#2}};%                                                            
}                                                                               
\newcommand\link[2]{%                                                           
\begin{tikzpicture}[remember picture, overlay, >=stealth, shift={(0,0)}]        
  \draw[->] (#1) to (#2);                                                       
\end{tikzpicture}%                                                              
}        


%%%%%%%%%%%%%%%%%%%%%%%%%%%%%%%%%%
%EDIT  (Running head--  REQUIRED)%
%%%%%%%%%%%%%%%%%%%%%%%%%%%%%%%%%%
\rhead{\small Texas Tech University, \textit{Binod Rajbhandari}, December 2020}	%update your name and graduation date-year here

%%%%%%%%%%%%%%%%%%%%%%%%%%%%%%%%%%%%%%%%%%%%%%%
%Uncomment if the grad school doesn't like the%
%line under the  running head:                %
%%%%%%%%%%%%%%%%%%%%%%%%%%%%%%%%%%%%%%%%%%%%%%%
\renewcommand{\headrulewidth}{0pt}


%%%%%%%%%%%%%%%%%%%%%%%%%%%%%%%%%%%%%%%%%%%%%%%%%%
%Spacing -- Do you want double or one-and-a-half?%
%%%%%%%%%%%%%%%%%%%%%%%%%%%%%%%%%%%%%%%%%%%%%%%%%%
%\doublespacing
\onehalfspacing
%%%%%%%%%%%%%%%%%%%%%%%%%%%%%%%%%%%%%%%%%%%%%%%%%%%%%%%%%%%%%
%Leave the one you want uncommented.                        %
%In places where single-line-spacing is appropriate         %
%e.g, extended quotations, you can enclose the material     %
%in a singlespacing environment (with \begin{singlespacing} %
% ...  \end{singlespacing}                                  %
%%%%%%%%%%%%%%%%%%%%%%%%%%%%%%%%%%%%%%%%%%%%%%%%%%%%%%%%%%%%%


%%%%%%%%%%%%%%%%%%%%%%%%%%%%%%%%%%%%%%%%%%%%%%%%%%
%Other preamble stuff, e.g., theorem environments%
%or newcommands go here:                         %
% e.g.                                           %
%%%%%%%%%%%%%%%%%%%%%%%%%%%%%%%%%%%%%%%%%%%%%%%%%%
% \newtheorem{theorem}{Theorem}
% \newtheorem{proposition}[theorem]{proposition}
% \newtheorem{question}{Question}
% \newtheorem{conjecture}{Conjecture}




\begin{document}
\setlength{\parindent}{10ex}
%%%%%%%%%%%%%%%%%%%%%%%%%%%%%%%%%%%%%%%%%%%%%%%%%%%%%%%%
%TITLE PAGE -- Edit the spacing commands after each \\ %
% if necessary                                         %
%%%%%%%%%%%%%%%%%%%%%%%%%%%%%%%%%%%%%%%%%%%%%%%%%%%%%%%%
\begin{titlepage}
\vbox to  \textheight{
\begin{singlespacing}
\begin{center}
First Search for $r$-modes gravitational waves from Crab Pulsar.\\[15pt]  %Edit
by\\[15pt]
Binod Rajbhandari,\\[15pt]   %Edit to put in your name and whatever degrees you already have
A Dissertation\\[15pt]   % or Thesis
In\\[15pt]
Physics\\[15pt]
Submitted to the Graduate Faculty\\
of Texas Tech University in\\
Partial Fulfillment of\\
the Requirements for\\
the Degree of\\[15pt]
Doctor of Philosophy\\[30pt]  %Edit (or Master of YYY)
Approved\\[15pt]
Dr. Benjamin Owen\\
Chair of Committee\\[15pt] %Edit
Dr. Joseph D. Romano\\[15pt] %Edit
Dr. Alessandra Corsi\\[15pt] %Edit (add/remove names if you have more or fewer committee members)
Dr. David Ian Jones\\[15pt] %Edit
Dr. Mark Sheridan\\ %Edit (look up the info on the Graduate School's website)
Dean of the Graduate School\\[30pt]
December 2020      %Edit
\end{center}
\end{singlespacing}
\vfill}
\end{titlepage}
%%%%%%%%%%%%%%%%%%%
%End of title page%
%%%%%%%%%%%%%%%%%%%

%%%%%%%%%%%%%%%%%%%%%%%%%%%%%%%%%%%%%%%%%%%%%%%%%%%%%%%
%Copyright page -- delete or comment out if not needed%
%usage: \copyrightpage{year of appearance}{Name}      %
%%%%%%%%%%%%%%%%%%%%%%%%%%%%%%%%%%%%%%%%%%%%%%%%%%%%%%%
\copyrightpage{2020}{Binod Rajbhandari} %Name should be same as on title page
%%%%%%%%%%%%%%%%%%%%%%%%
%\end of copyright page%
%%%%%%%%%%%%%%%%%%%%%%%%

%%%%%%%%%%%%%%%%%%%%%%%
%Start of frontmatter %
%You need this:       %
%%%%%%%%%%%%%%%%%%%%%%%
\frontmatter


%%%%%%%%%%%%%%%%%%%%%%%%%%%%%
%Acknowledgements           %
%Comment out or delete      %
%if not  wanted             %
%%%%%%%%%%%%%%%%%%%%%%%%%%%%%
\chapter{\textbf{Acknowledgements}}
%I am thankful to my research advisor Professor Benjamin Owen for introducing me
%to the ground breaking research on Gravitational waves. Although I was stranger
%to a topics on gravitational waves, your constant support, guidance and
%encourangement made it possible to to understand the dark side of the universe. 
%I am also grateful for showing me the Data Analysis which added a different
%knwoledge in my academic career.
%I owe a lot to a former Postdoc Ra Inta and Santiago Caride for guiding me
%during the beginning phase, when I was completely struggling to
%understand the research. 


%%%%%%%%%%%%%%%%%%%%%%%%%
%End of acknowledgements%
%%%%%%%%%%%%%%%%%%%%%%%%%
\newpage
\chapter{Statements}
Chapter 4 and 6 consists of a work that has been or soon to be published in the 
Physical Review Letter D. The idea on chapter 4 came during the beginning work
of the thesis. The paper is written by Professor Benjamin Owen, I plotted the
figure and verified the calculations and formulas. The data analysis on Chapter
6 is performed by myself and I also wrote most part of the paper. Professor Owen
edited the paper into the final updated form. 
When I was a member of LIGO Scientific Collaboration, I made a significant
contribution in one of the paper by locating a glitch in a Crab pulsar during
the second observational run of Advanced LIGO. I found the glitch while reading
the glitch catalogue  of Jodrell Bank Observatory. The member of the group had
to rerun the search. 

%%%%%%%%%%%%%%%%%%%
%Table of Contents%
%%%%%%%%%%%%%%%%%%%
\tableofcontents	%Leave this here (table will auto-update as you write new chapters/subsections/appendix)

%%%%%%%%%%%%%%%%%%%%%%%%%%%%%%%%%%%%%%%%%%%%%%%%%%
%Abstract -- Delete or comment out if not wanted:%
%%%%%%%%%%%%%%%%%%%%%%%%%%%%%%%%%%%%%%%%%%%%%%%%%%
\chapter{\textbf{Abstract}}
Neutron stars are the most dense form of matter, the density comparable to the
density of atomic nucleus.  If the neutron star is rotating, then, just like
Rossby wave on Earth atmosphere, motion of this fluid will be susceptible to the
Coriolis force. These non-radial oscillations, known as $r$-modes, can be very
long lived---perhaps thousands of years. Because the fluid is so dense, these
r-modes may be viable sources of continuous gravitational waves.  R-modes are
the current quadrupole which oscillates at four thirds the spin frequency. The
modes are damped by viscosity but can be unstable to gravitational radiation via
the Chandrasekhar-Friedman-Schutz instability. When relativistic corrections are
taken into consideration the mode frequency can be 1.39 to 1.57 times the spin
frequency of the star and the frequency derivative can be roughly estimated in
terms of the star's measured spin-down parameter. We show for the first time how
to construct searches over appropriate ranges of frequencies and spin-down
parameters to target $r$-modes from known pulsars. 

We present the first searches for gravitational waves from
$r$-modes of the Crab pulsar, coherently and separately integrating data from
three stretches of the first two observing runs of Advanced LIGO using the
$\mathcal{F}$-statistic. The band of frequencies was 41.1--45.6\,Hz.  The
frequency and derivatives searched were based on radio measurements of the
pulsar's spin-down parameters  as described in \citet{PhysRevD.100.064013}.  We
did not find any evidence of gravitational waves. The best upper limits on the
gravitational wave amplitude  were $1.61\times10^{-25}$(O1), $1.45\times10^{-25}
$(first half of O2) and $1.20\times10^{-25} $(second half of O2). These are the
first upper  limits on gravitational waves from $r$-modes of a known pulsar to
beat the spin-down limit, and they do so by more than an order of magnitude
in amplitude or two orders of magnitude in luminosity.  
%%%%%%%%%%%%%%%%%
%End of abstract%
%%%%%%%%%%%%%%%%%


%%%%%%%%%%%%%%%%%%%%%%%%%%%%%%%%%%%%%
%List of tables and list of figures %
%Delete or comment out if not needed%
%%%%%%%%%%%%%%%%%%%%%%%%%%%%%%%%%%%%%
\listoftables	%Leave this here. List will auto-update when you build a new table

\listoffigures	%Leave this here. List will auto-update when you insert new figures
%%%%%%%%%%%%%%%%%%%%%%%%%%%%%%%%%%%%
%End of lists of tables and figures%
%%%%%%%%%%%%%%%%%%%%%%%%%%%%%%%%%%%%
\newpage
\textbf{Acronyms}
\begin{acronym}[OCSVM]
\acro{NS}{Neutron Star}                                                      
\acro{EM}{electromagnetic wave}                                              
\acro{GW}{gravitational wave}                                                
\acro{LIGO}{Laser Interferometer Gravitational wave Observatory}             
\acro{O1}{first observing run}                                               
\acro{O2}{second observing run}                                              
\acro{SFT}{short fourier transform}                                          
\acro{SNR}{signal-to-noise-ratio}  
\acro{PSD}{power spectral density}
\acro{CW}{continuous wave}
\end{acronym}



%%%%%%%%%%%%%%%%%%%%%%%%
%MAIN PART OF  DOCUMENT%
%%%%%%%%%%%%%%%%%%%%%%%%

\mainmatter

%%%%%%%%%%%%%%%%%%%%%%%%%%%%%%%%%%%%%%%%%%%%%%%%%%%%%%%%%%%%%%%%%%%%%%%%%%%%%%%%%%%%%%%%%
%%%%%%%%%%%%%%%%%%%%%%%%%%%%%%%%%%%%%%%%%%%%%%%%%%%%%%%%%%%%%%%%%%%%%%%%%%%%%%%%%%%%%%%%%
%%% 							Chapter 1											  %%%
%%%%%%%%%%%%%%%%%%%%%%%%%%%%%%%%%%%%%%%%%%%%%%%%%%%%%%%%%%%%%%%%%%%%%%%%%%%%%%%%%%%%%%%%%
%%%%%%%%%%%%%%%%%%%%%%%%%%%%%%%%%%%%%%%%%%%%%%%%%%%%%%%%%%%%%%%%%%%%%%%%%%%%%%%%%%%%%%%%%

\chapter{\textbf{Introduction}}
    On September 14 2015, two detector of Laser Interferometer Gravitational
Observatory detected a chirp signal with 10 ms apart ~\cite{Abbott_2016}. The
signal frequency went from 35 Hz to 250 Hz with a peak amplitude of $10^{-21}$.
The signal matches the waveform of two binary black hole of mass 36 $M_\odot$
and 29 $M_\odot$ spiraling each other at 410 $Mpc$(1.4 billion light years) to
form a remnant of 62 $M_\odot$ and the remaining 3 $M_\odot$ is released as
gravitational waves.  The false alarm rate of this event is 1 per 203,000 years
which is of significance level 5.1$\sigma$ which makes it of astrophysical
origin.  After,some complicated data analysis to invalid any terrestrial origin
of signal, the event was announced in 11 February 2016 to the world. This
discovery led to a 2017 Nobel prize in Physics award to Kip Thorne, Rai Weiss and
Barry Barish.
\begin{figure}[bht!]                                                              
        \includegraphics[width=\textwidth]{figure/BBH.png}                          
	\caption{The first detection of \ac{GW} from a binary black hole
merger by two \ac{LIGO} detector~\cite{Abbott_2016}.}
\end{figure}      

During the second observation run of Advanced LIGO and Virgo(August 17, 2017),
the network of three detector observed a gravitational wave from two binary
neutron star merger~\cite{Abbott_2017}. The signal was first detected in Virgo
detector and in Livingston after 22 ms and later in Hanford in another 3 ms
limiting the sky localization to 28 square degrees. The signal frequency went
form 30 Hz to 2048 Hz within a minute. The total mass of combined neutron star
is $2.74_\odot$ where $0.025 M_\odot$ were radiated as \ac{GW} energy at a
luminosity distance of 40 Mpc(130 million light years). After 1.7 s a associated
burst of gamma rays was detected by Fermi telescope from a same  location. Other
spectrum of \ac{EM} waves were discovered ranging from few hours for optical
light to a few weeks for radio waves.  This brings the first cosmic event that
were seen in both gravitational waves and electromagnetic spectrum starting the
multi-messenger astronomy. 
\begin{figure}[bhtp!] 
        \includegraphics[width=\textwidth]{figure/GW170817.png}
	\caption{The first detection of binary neutron star merger (GW170817) by
the \ac{LIGO} detector~\cite{Abbott_2017}. The figure shows a spectrogram of the \ac{GW} event
GW170817. The signal direction was on the blind side of the VIRGO detector, this
was important to predict the sky location of GW170817.  }
        \label{GW170817}                                                             
\end{figure}      

The third observation of LIGO made 56 \ac{GW} detection of which $GW190412$ was
a significant due to its asymmetric mass. The $30M_\odot$ merged with
$8M_\odot$ companion at 2 billion light-years distance that made a final remnant 
 of  $37M_\odot$ and $1M_\odot$ was lost in \ac{GW}
energy.~\cite{collaboration2020gw190412} The asymmetric
mass black hole can radiate \acp{GW} at higher multipole other than quadrupole.
The \ac{GW} frequencies are related to the orbital frequency by the relation, 
$f_{lm}=mf_{orb}$. For this event, the interferometer detected both
$(l,m)=(2,\pm 2)$ with $f_{GW}=2f_{orb}$ and higher harmonics $(l,m)=(3,\pm 3)$ with
$f_{GW}=3f_{orb}$, so the \ac{GW} frequencies were in ratio of 3:2 as in musical
fifth.

The detection of \ac{GW} opened a new way to understand the universe. This not
only proved the Einstein Theory of General Relativity but this is the first
detection of binary blackhole merger with a  mass of blackhole never been
observed before. The neutron star collision was seen not only in \ac{GW} but also
in all \ac{EM} spectrum. This event showed the heavier elements like uranium and
gold are formed through a r-process nucleosynthesis where a nuclei gets absorbed
by heavier elements nucleus such as iron and the nucleus decays into proton and
electron increasing the proton number of nucleus and forming new elements. 

Recently, during the third observation run of Advanced \ac{LIGO}, the
three network of interferometer detected a two black hole of mass $85M_\odot$ and $65M_\odot$
merged into a black hole of mass $142M_\odot$~\cite{Abbott_2020}. This is the first detection of
Intermediate mass black hole. This 142 solar mass black hole fills up the gap
between the stellar mass black hole (less than $100M_\odot$) and super massive
black hole ($10^6-10^9 M_\odot$). Since, this is the heaviest black hole
detected by \ac{LIGO}, the frequency of the signal is less than the other
binary black hole merger. The signal lasted for 100 milliseconds with peak
frequency of 60 Hz.

%%%%%%%%%%%%%%%%%%%%%%%%%%%%%%%%%%%%%%%%%%%%%%%%%%%%%%%%%%%%%%%%%%%%%%%%%%%%%%%%%%%%%%%%%
%%%%%%%%%%%%%%%%%%%%%%%%%%%%%%%%%%%%%%%%%%%%%%%%%%%%%%%%%%%%%%%%%%%%%%%%%%%%%%%%%%%%%%%%%
%%% 							Chapter 2											  %%%
%%%%%%%%%%%%%%%%%%%%%%%%%%%%%%%%%%%%%%%%%%%%%%%%%%%%%%%%%%%%%%%%%%%%%%%%%%%%%%%%%%%%%%%%%
%%%%%%%%%%%%%%%%%%%%%%%%%%%%%%%%%%%%%%%%%%%%%%%%%%%%%%%%%%%%%%%%%%%%%%%%%%%%%%%%%%%%%%%%%

\chapter{Gravitational Waves}
Compact objects such as black hole warps the space-time, when these bulk objects
moves at relativistic speed, the changing warpage in space-time propagates at
speed of light known as gravitational waves~\cite{thorne1995gravitational}. The
detection of gravitational wave has created the new era in astronomy. This
brought the same excitement in astronomy community as the discovery of pulsars,
observation of jets from super-massive black hole from the radio telescopes.
\acp{GW} can travel a long distances without getting absorbed or scatter whereas
\acp{EM} get absorbs or scatter easily. So, the universe that can be observed
through \acp{GW} are opaque to \ac{EM}. 

\section{Theory of Gravitational Waves}
The Principle of Equivalence describes an equation of motion of freely
falling particle as:
\begin{equation}
\frac{d^2x_\lambda}{d\tau^2}+\Gamma^\lambda_{\mu\nu}\frac{dx_\mu}{d\tau}\frac{dx_\nu}{d\tau}=0
\end{equation}
where $\mu$ and $\nu$ are the space-time coordinates, the indices goes from 0-3.
$\tau$ is the proper time and $\lambda$ is a parametrization.\\
and $\Gamma^\lambda_{\mu\nu}$ is the affine connection:
\begin{equation}
\Gamma^\mu_{\mu\nu} =\frac{\partial x^\lambda}{\partial
\xi^\alpha}\frac{\partial^2\xi^\alpha}{\partial x_\mu \partial x_\nu}
\end{equation}
\begin{equation}
d\tau^2=dt^2 - dx^2 = -g_{\mu\nu}dx_\mu dx_\nu
\end{equation}
where $g_{\mu\nu}$ is a metric tensor that describes space-time curvature.
\subsection{Newtonian Limit}
	For a slowly moving particle 
\begin{equation}
\frac{dx^i}{d\tau}<< \frac{dt}{d\tau}
\end{equation}
\begin{equation}
\frac{d^2x_\mu}{d\tau^2}+\Gamma^\lambda_{00}\left(\frac{dt}{d\tau}\right)^2=0
\end{equation}
For a stationary field, $\Gamma^\lambda_{00}$ can be written as:
\begin{equation}
\Gamma^\lambda_{00}= -1/2 g^{\mu \nu} \frac{\partial g_{00}}{\partial x^\nu}
\end{equation}	
For a weak gravitational field, we can write the metric in Minkwoski plus small
perturbation:
\begin{equation}
g_{\mu \nu}= \eta_{\mu \nu} + h_{\mu \nu}
\end{equation}
\begin{equation}
\Gamma^\lambda_{00}= -1/2 g^{\mu \nu} \frac{\partial h_{00}}{\partial x^\nu}
\end{equation}	
\begin{equation}
\frac{d^2x_\mu}{d\tau^2}=1/2 g^{\mu \nu} \frac{\partial h_{00}}{\partial x^\nu} \left(\frac{dt}{d\tau}\right)^2
\end{equation}
For $\partial_0h_{00}=0$ and $\mu=0$ component becomes:
\begin{equation}
\frac{d^2t}{d\tau^2}=0
\end{equation}
This means $dt/d\tau$ is constant. And
\begin{equation*}
\eta_{\mu\nu}=
 \begin{pmatrix}
    -1 & 0 & 0 & 0 \\
    0 & 1 & 0 & 0 \\
    0 & 0 & 1 & 0 \\
    0 & 0 & 0 & 1 
 \end{pmatrix}
\end{equation*}

\begin{equation}                                                                
\frac{d^2x^i}{d\tau ^2}=1/2 \left(\frac{dt}{d\tau}\right)^2 \partial _i h_{00}
\end{equation} 

Dividing both sides by $\left(\frac{dt}{d\tau}\right)^2$, we get,
\begin{equation} \label{GR}                                                               
\frac{d^2x^i}{dt^2}=1/2 \partial _i h_{00}
\end{equation} 

On Newtonian case,
\begin{equation} \label{Newton}
\frac{d^2x}{dt^2}=-\nabla \Phi
\end{equation}

Comparing \ref{GR} and \ref{Newton},we get,
\begin{equation} \label{eq:14}
h_{00}=2\Phi
\end{equation}
And, comparing with the metric $g_{\mu \nu}= \eta_{\mu \nu} + h_{\mu \nu}$
\begin{equation}\label{eq:15}
g_{00}=-(1+2\Phi)
\end{equation}
In Newtonian theory, gravitational potential can be written in Poisson
equation:
\begin{equation} \label{eq:16}
\nabla ^ 2\Phi = 4\pi G\rho
\end{equation}
where, $\Phi=-\frac{GM}{R}$ for a point mass like distribution and $\rho$ is a
mass density. In Newtonian mechanics mass is the only source of gravitational
field but in general relativity we need a quantity known as stress energy
density($T_{\mu\nu}$). And, the gravitational potential can be written as a metric
tensor ($h_{\mu\nu}$).
In weak field limit, the rest energy can be written as:
\begin{equation}\label{eq:17}
T_{00}=\rho
\end{equation}
From \ref{eq:14}, \ref{eq:16} and \ref{eq:17}
\begin{equation} \label{eq:18}
\nabla ^2h_{00}=-8\pi GT_{00}
\end{equation}
	
	For a vacuum field equation the Ricci tensor $R_{\mu \nu} = 0$ since
there is no gravitational field. But the equation with matter will be:
\begin{equation} \label{eq:19}
R_{\mu \nu} = 8 \pi GT_{\mu \nu}
\end{equation}
In a local inertial frame, the stress-energy tensor is conserved in curved
space-time.
\begin{equation} \label{eq:20}
\nabla _\mu T_{\mu \nu} =0
\end{equation}
from \ref{eq:19} and \ref{eq:20},
\begin{equation}
\nabla _\mu R_{\mu \nu} =0
\end{equation}
light. In GR, Einstein field equation is written as:
\begin{equation}
R_{\mu\nu} -\frac{1}{2}g_{\mu\nu}=-\kappa T_{\mu\nu}
\end{equation}	
where $R_{\mu\nu}$ is a Ricci tensor,$g_{\mu\nu}$ is a metric tensor, R is a scalar
curvature and $T_{\mu\nu}$ is a stress energy tensor. The LHS describes the
curvature of space-time and RHS defines the energy density.\\
Compact object such as black hole and neutron star warps the
space-time, and when such masses accelerates it changes the stress-energy on the
space-time. Such a disturbance propagation known as GW travels at a speed of

\subsection{Linearized Theory of Gravity}
	In the weak field limit, i.e when space-time curvature is small,
Einstein field equation can be written as:
\begin{equation} \label{eq:23}
g_{\mu \nu}= \eta_{\mu \nu} + h_{\mu \nu}
\end{equation}
When $h_{\mu \nu}$ is very small we can expand $h_{\mu \nu}$ to first order
approximation. 
	In vacuum $T_{\mu \nu} =0$ so, the Einstein filed equation becomes:
\begin{equation}\label{eq:21}
R_{\mu \nu} =0
\end{equation}
where,
\begin{equation} \label{eq:22}
R_{\mu \nu} = \frac{\partial \Gamma ^\gamma _{\mu \nu}}{\partial x^\gamma} -
\frac{\partial \Gamma ^\gamma _{\mu \gamma}}{\partial x^\nu}+ \Gamma ^\gamma
_{\mu \nu}\Gamma ^\delta _{\gamma \delta}-\Gamma ^\gamma _{\mu \delta}\Gamma
^\delta_{\nu \gamma}
\end{equation}
The last two terms in \ref{eq:22} are quadratic in $h_{\mu \nu}$ so are negligible
in linear approximation.The perturbation in Ricci tensor is given by:
\begin{equation} \label{eq:23}
\delta R_{\mu \nu} = \frac{\partial \delta \Gamma ^\gamma _{\mu \nu}}{\partial 
x^\gamma} - \frac{\partial \delta \Gamma ^\gamma _{\mu \gamma}}{\partial x^\nu}
\end{equation}
The first order perturbation in Christoffel symbol:
\begin{equation} \label{eq:24}
\delta \Gamma ^\gamma _{\mu \nu}= 1/2 \eta^{\gamma \delta}\left(\frac{\partial
h_{\delta \mu }}{\partial x^\nu} + \frac{\partial h_{\delta \nu }}{\partial
x^\mu} -\frac{\partial  h_{\mu \nu }}{\partial x^\delta}\right)
\end{equation}
\begin{equation} \label{eq:25}
\delta R_{\mu \nu} = \frac{1}{2}\left[ -\left(-\frac{\partial^2}{\partial
t^2}+\nabla ^2\right)\tilde h_{\mu\nu} + \delta_\mu V_\nu + \delta_\nu V_\mu
\right]
\end{equation}
\begin{equation} \label{eq:26}
V_\mu = \delta_\gamma h^\gamma_\mu - \frac{1}{2}\delta_\mu h^\gamma_\gamma
\end{equation}
	If we choose $V_\mu=0$ similar to Lorentz condition in electromagnetism,
 the linearized equation becomes:
\begin{equation} \label{eq:27}
\left(-\frac{\partial^2}{\partial t^2}+\nabla ^2\right)\tilde h_{\mu\nu} =0 
\end{equation}

%We can define a transverse reverse of $h_{\mu \nu} 
%\begin{equation}
%\tilde h_{\mu \nu} \approx h_{\mu \nu} -1/2 \eta_{\mu \nu}h
%\end{equation}
Now, the gauge condition becomes:
\begin{equation}\label{eq:28}
\partial _\mu \tilde h^\mu _\lambda = 0
\end{equation}
The Einstein field equation in vacuum becomes:
\begin{equation}\label{eq:29}
\left(-\frac{\partial^2}{\partial t^2}+\nabla ^2\right)\tilde h_{\mu\nu} =0
\end{equation}


\subsection{Gravitational wave radiation}
The eqn \ref{eq:27} is the plane wave solution given by:
\begin{equation} \label{eq:30}
h^{\mu \nu} = A^{\mu \nu}e^{ik_\sigma x^\sigma}
\end{equation}
$A^{\mu \nu}$ is a polarization tensor where it has an information of
polarization and amplitudes of the gravitational waves.  Apart from gauge
transformation we cam use the coordinate transformation to simplify the $A_{\mu
\nu}$
\begin{equation}
x^{'\mu}= x^\mu +x_i ^\mu(x)
\end{equation}
\begin{equation}
\left(-\frac{\partial^2}{\partial t^2}+\nabla ^2\right)\xi_\mu(x) =0
\end{equation}
Since $A_{\mu \nu}$ also satisfies the wave equation, the transformation will
eliminate any of the four components of $A_{\mu \nu}$.
\begin{equation}
\begin{aligned}
h_{ti}=0, \\
h^\nu_\nu =0,
\end{aligned}
\end{equation}
or $A_{ti}=0$ and $A_\nu ^\nu=0$
Using the gauge condition $V_\mu =0$
\begin{equation}
\begin{aligned}
V_t=\frac{\partial h^t _t}{\partial t}=0,\\
V_i=\frac{\partial h^j _i}{\partial x^j}=0
\end{aligned}
\end{equation}
From the above equation,
\begin{equation}                                                                
\begin{aligned} 
A_{tt}=0, \\
k_j A_{ij}=0,
\end{aligned}                                                                   
\end{equation} 
This means that the gravitational waves are transverse and the wave vector k
determines the direction of wave propagation. The time component vanishes and
since the wave are transverse $A_{zi}=0$ assuming $\vec{k}=(0,0,\omega)$. Now,
we are left with $2 \times 2$ symmetric matrix in xy-plane with trace equals to
zero. With coordinate transformation and gauge conditions, the linearized
Einstein equation can be written in terms of two dimensionless amplitudes
$h_+$ and $h_\times$ and two polarization vectors $\hat e_+$ and $\hat e_\times$.

\begin{equation*}                                                               
h_{\mu\nu}=                                                                  
 \begin{pmatrix}                                                                
    0 & 0 & 0 & 0 \\                                                           
    0 & h_+ & h_\times & 0 \\                                                            
    0 & h_\times & -h_+ & 0 \\                                                            
    0 & 0 & 0 & 1                                                               
 \end{pmatrix}                                                                  
\end{equation*} 
The metric$h_\mu\nu$ can be written as sum of two polarization components.   
\begin{equation}
h_{\mu\nu}=h_+\hat e_+ +h_\times \hat e_\times
\end{equation}
The polarization tensor are defined as:
\begin{equation}
\begin{aligned}
\hat e_+ = \hat e_x \otimes \hat e_x - \hat e_y \otimes \hat e_y \\
\hat e_\times = \hat e_x \otimes \hat e_y + \hat e_y \otimes \hat e_x
\end{aligned}
\end{equation}
Or we can write the metric in matrix form:
\begin{equation*}                                                               
h_{\mu\nu}=     
 \begin{pmatrix}                                                                
    0 & 0 & 0 & 0 \\                                                            
    0 & 1 & 0 & 0 \\                                                            
    0 & 0 & -1 & 0 \\                                                            
    0 & 0 & 0 & 0                                                               
 \end{pmatrix}
h_+  
+
 \begin{pmatrix}                                                                
    0 & 0 & 0 & 0 \\                                                            
    0 & 0 & 1 & 0 \\                                                            
    0 & 1 & 0 & 0 \\                                                            
    0 & 0 & 0 & 0                                                               
 \end{pmatrix}
h_\times                                                                  
\end{equation*} 

\begin{figure}[h!]                                                              
	\includegraphics[width=\textwidth]{figure/polarization.png}
	\caption{The figure shows the effect of passing gravitational wave
through a ring of particle. The top figure shows the plus polarization and the
bottom shows cross polarization. Figure from ~\citet{Schutz}}                                                     
        \label{fig:polarization}
\end{figure}

	The effect of passing gravitational wave is to compress and stretch
alternately in transverse direction. Let us assume a ring of particle in a xy
plane and a gravitational wave travelling on a z-direction. This will cause the
particle to contract in x-axis at the same time expand on y-axis. These
contraction and expansion will oscillate on passing of gravitational waves. 

\subsection{Gravitational Wave Luminosity}
	\ac{GW} must carry energy which was famously described by Feynman
sticky bead argument. \ac{GW} energy cannot be localized for a certain region,
however we can say a certain amount of energy is contained in a macroscopic
region. The stress energy of \acp{GW} is given by:
\begin{equation}
T_{\mu\nu}^{GW} = \frac{1}{32\pi}<h_{jk,\mu}^{TT}h_{jk,\nu}^{TT}>
\end{equation}
where <> means averaging over several wavelength and TT means Transverse
Traceless.\\
For a wave propagating in z-direction, it has three non-zero components.
\begin{equation}
T_{00}^{GW}=\frac{1}{16\pi}<\dot{h}_+^2+\dot{h}_\times ^2>
\end{equation}
If we suppose \acp{GW} has a frequency f and the amplitude $h_+ = h+\times = h$,
then, $\dot{h_+} = \dot{h_\times} = \frac{1}{2}(2\pi fh)^2$
S, the energy density of the \acp{GW} is:
\begin{equation}
T_{00}^{GW} = \frac{\pi}{4}\frac{c^3}{G}f^2 h^2=0.3\left(\frac{f}{1kHz}\right )^2
\left(\frac{h}{10^{-21}}\right )^2\frac{W}{m^2}
\end{equation}
\\
The spin down limit for Crab amplitude strain is $h\approx3*10^{-24}$ with
r-mode frequency around 41Hz.So, the total energy flux radiated by Crab pulsar
on earth due its loss of rotational energy will be
$4.5*10^{-9}\frac{W}{m^2}$. This is small compared to the radiation flux emitted
by the sun($1400W/m^2$) but the \ac{GW} from the binary black merger will be
m nearby cluster will be several magnitude greater than the solar flux.

\subsection{Generation of Gravitational Waves}

	The \ac{GW} is quadrupole in nature as the mass(monopole) and
momentum(dipole) are conserved. On Electromagnetism we can have a dipole
radiation as there can be positive and negative charges.
The Einstein quadruple formula for \acp{GW} can be written as:
\begin{equation}
h_{\mu\nu}= \frac{2}{r}\frac{G}{c^4}I_{\mu\nu}^{TT}
\end{equation}
where $I_{\mu\nu}$ is a quadrupole moment.


Using the Einstein quadruple formula the \ac{GW} luminosity can also be written
as:
\begin{equation}
\frac{dE}{dt} = -\frac{G}{c^5}<\dddot{I}_{\mu\nu}\dddot{I}_{\mu\nu}>
\end{equation}
The power radiated by \ac{GW} is similar to the power radiated by the electric
dipole, where the expression is written as
$\frac{dE}{dt}=\frac{\mu_o}{6\pi}<\ddot{p}^2>$, where p is an electric
dipole.Like, we discussed earlier, in \ac{GW} the first non vanishing term is
quadrupole whereas in \acl{EM} it is dipole.

\section{Indirect evidence of Gravitational Waves}
	
Gravitational radiation will remove the energy from a binary system or a
rotating neutron star. Hulse and Taylor cite found a binary system with a
neutron star and a pulsar named PSR B1913+16 orbital period were decaying which
can be explained by \acl{GW} emission. Both companion has a mass of $1.4 M_o$
with a orbital period of 8 hrs, the rate of orbital decay was around $10 \mu s$
as predicted by General relativity within 0.2\%. The disparity between predicted
and observed was due to the poor measurement of distance and proper motion of
pulsar as well as the radius of Sun's galactic orbit cite. This discovery led to
a 1193 Nobel prize in Physics for Russell Hulse and Joseph Taylor.
\begin{figure}[bht!]
	\includegraphics[width=\textwidth]{figure/pulsar.png}
	\caption{The orbital decay of binary pulsars $ PSR B1913+16$ due to the
loss of \ac{GW} energy. The image is from \cite{Weisberg_2010}}
\end{figure}  

\section{Gravitational Wave Detector}
The challenge of detecting \ac{GW} is the amplitude of \ac{GW} are really small
 ($\approx10_{-18}m$), so the noises in detector can easily mask the signal.
Some of the \ac{GW} detectors includes resonant bar, interferometers, pulsar
timing arrays. 

\subsection{Resonant bars}

The resonant bar works on a simple concept when the frequency of
\ac{GW} matches the bar, the detector will resonates and ring like a bell. This
was the first \ac{GW} detector built by Joseph Weber which consists of an
aluminum bar of 2 meters in length and a pizzo-electric sensors. But the
noises($10^{-16}$) surpasses the \ac{GW} signals($10^{-21}$) by many order of
magnitude and the GW frequency should match the resonant bar frequency for
resonances. He claimed the \ac{GW} detection coming from center of galaxy which
lead to construction of similar detector by other physicist. But none of the
other detector saw any signals and the claim might have been a glitch in the
instrument.
\subsection{Interferometer}
The \ac{GW} interferometry works on the simple interferometer principle
developed by Michelson and Morley which consists of a laser, beam splitter,
mirror and a detector. The laser emits a monochromatic wave towards a beam
splitter which splits the beam into two perpendicular direction. The beam gets
reflected by the mirrors, passes through the beam splitter and meets at the
photo-detector. The two waves superimpose at the detector to give constructive or
destructive interference depending on the phase difference. When \acp{GW} passes
the interferometer the relative length between two arms oscillates changing the
intensity of light at photo-detector. Actually, the light reflects several times
between mirrors before they combine, this makes the effective length of the
interferometer more than its physical arm length.

There are two working \ac{LIGO},one at Livingston, Louisiana and Hanford, Washington 
separated by 3000km apart. Each arm of the interferometer are 4km in length. The
VIRGO interferometer in Italy and KAGRA(3km arms length) in Japan are also
operational. It takes at least three detector to known the sky location of 
\ac{GW} sources, its intrinsic amplitude, polarization angle and the
distance.~\cite{Schutz_2011}. But for the long living \ac{CW} signals the Doppler effect
due to Earth's orbital and sidereal motion can help to locate the source. The
network of detector will also build up the signal to noise ratio(SNR) as the
network SNR is the sum of individual detector SNR~\cite{Schutz_2011}.
\begin{equation}
\rho_N^2=\sum_{k=1}^N \rho_k^2 ,
\end{equation}

\subsubsection{Principle of Interferometer}
This section is based on the Sathyaprakash paper ~\cite{Sathyaprakash_2009}. The
\ac{GW} interferometer is based on the principle the passing of \ac{GW} will
change the length of the arm of the interferometer. If there is a \ac{GW}
passing then the arrival time of emitted laser will be different than when no
\ac{GW} passes. Lets assume a photon is emitted at time $t_{start}$ and travels
in x-direction and returns back time $t_{return}$.
\begin{equation}
\frac{dt_{return}}{dt}=1+\frac{1}{2}{(1-cos\theta)h_+(t+2L)-(1+cos\theta)h_+(t)
+ 2cos\theta h_+[t+L(1-cos\theta)]}
\end{equation}
For a small L approximation compare to the wavelength of \ac{GW}.
\begin{equation}
\frac{dt_{return}}{dt} =1+sin^2\theta L \dot{h}_+ (t)
\end{equation}
Or for x-arm, it can be written as:
\begin{equation}
\left(\frac{dt_{return}}{dt})\right)_{x-arm}=1+L\hat{e}_x.\dot{h}.\hat{e}_x
\end{equation}
Similarly for y-arm:
\begin{equation}                                                                
\left(\frac{dt_{return}}{dt}\right)_{y-arm}=1+L\hat{e}_y.\dot{h}.\hat{e}_y  
\end{equation}  
The difference between the arrival time of photon from two arms is:
\begin{equation}
\left(\frac{d\delta t_{return}}{dt}\right)=L(\hat{e}_x\otimes\hat{e}_x-(\hat{e}_y\otimes\hat{e}_y)\dot{h}
\end{equation}
Or the path difference between two arms as measured by the central observer's:
\begin{equation}
\delta t_{return}(t)=d:h
\end{equation}
where $d=L(\hat{e}_x\otimes\hat{e}_x-(\hat{e}_y\otimes\hat{e}_y)$ and 
$d:h=d_{lm}h_{lm}$
\\
This can be written in terms of change in length of two arms:
\begin{equation}
\delta L_{return}(t)=\frac{1}{2}d:h 
\end{equation}
The $\frac{1}{2}$ is because the light travels back and forth for the
interferometer arm length change.
\subsubsection{Interferometer antennae pattern}

	The \ac{GW} astronomy are done with combinations of detectors around the
globe. The \ac{GW} polarization might not be same in all detector frame, so it
is more convenient to choose a polarization tensor in some sky plane with respect
to \ac{GW} sources. Let $\hat\epsilon_+$ and $\hat\epsilon_\times$ be two polarization
tensor in sky frame as shown in figure \ref{fig:polarization}. Let $\psi$ be the rotating angle from
detector frame to the source frame also known as polarization angle.
\begin{equation}
\begin{aligned}
\epsilon_+ = e_+cos2\psi + e_\times sin2\psi, \\
\epsilon_\times = -e_+sin2\psi + e_\times cos2\psi.
\end{aligned}
\end{equation}
\begin{figure}[h!]
	\includegraphics[width=\textwidth]{figure/antennae.jpg}
	\caption{The left figure shows the basis vector in the sky plane with
respect to the detector frame. The right figure shows the effect of rotation of
the basis vector of sky plane by angle $\psi$. Image from
~\cite{Sathyaprakash_2009}}
\end{figure}  

Let $F_+$ and $F_\times$ be the antennae pattern functions on the sky frame
defined as:
\begin{equation}
F_+=d:e_+,  F_\times=d:e_\times
\end{equation}
The maximum value of $F_+$ and $F_\times$ is 1. We can define $\theta$ \&
$\phi$ be a spherical coordinates in a detector reference frame. We can write
the antennae pattern in detector frame by setting $\psi=0$ and using the
rotation matrix.
\begin{equation*}                                                               
R_\beta^\alpha=                                                                  
 \begin{pmatrix}                                                                
    \cos\phi & \sin\phi & 0 \\                                                            
    -\cos\theta \sin\phi & \cos\theta \cos\phi & \sin\theta \\                                                            
    \sin\theta \sin\phi & -\sin\theta \cos\phi & \cos\theta                                                              
 \end{pmatrix}                                                                  
\end{equation*}   

For a plus polarization $t_{\alpha\beta}= R^{-1}e_+R$

\begin{equation*}                                                               
\begin{split}
t_{\alpha\beta} & =                                                                  
 \begin{pmatrix}                                                                
    \cos\phi & \sin\phi & 0 \\                                                            
    -\cos\theta \sin\phi & \cos\theta \cos\phi & \sin\theta \\                                                            
    \sin\theta \sin\phi & -\sin\theta \cos\phi & \cos\theta                                                              
 \end{pmatrix}^T                                                                  
\begin{pmatrix}                                                                
    1 & 0 & 0 \\                                                            
    0 & -1 & 0 \\                                                            
    0 & 0 & 0                                                               
 \end{pmatrix}       
 \begin{pmatrix}                                                                
    \cos\phi & \sin\phi & 0 \\                                                     
    -\cos\theta \sin\phi & \cos\theta \cos\phi & \sin\theta \\                        
    \sin\theta \sin\phi & -\sin\theta \cos\phi & \cos\theta                           
 \end{pmatrix}\\  
            &=
     \begin{pmatrix}                                                                 
    \cos^2\phi-\cos^2\theta\sin^2\phi & (1+\cos^2\theta)\sin\phi\cos\phi &
\sin\theta\cos\theta\sin\phi \\                                                                
    (1+\cos^2\theta)\sin\phi\cos\phi & \sin^2\phi-\cos^2\theta\cos^2\phi &
-\sin\theta\cos\theta\cos\phi \\                                                               
    \sin\theta\cos\theta\sin\phi & -\sin\theta\cos\theta\cos\phi & -\sin^2\theta                                                                   
 \end{pmatrix}     
\end{split}     
\end{equation*}   
So, the antennae pattern for the plus polarization is $F_+=1/2t_{\alpha\beta}S_+$
\begin{equation}
F_+ = (1+\cos^2\theta)\cos2\phi
\end{equation}
Similarly for cross polarization:
\begin{equation}
F_\times=\cos\theta\sin2\phi
\end{equation}
In the source frame where $\psi\neq0$, the antennae pattern is given by:
\begin{align}
F_+=\frac{1}{2}(1+\cos^2\theta)\cos2\phi\cos2\psi-\cos\theta\sin2\phi\sin2\psi\\
F_\times=\frac{1}{2}(1+\cos^2\theta)\cos2\phi\sin2\psi-\cos\theta\sin2\phi\cos2\psi
\end{align}

\subsubsection{Noises in Interferometer}

The amplitude strain of the binary black-hole merger few billion light years are
of $h \approx10^{-21}$. When \ac{GW} passes by, the change in arm length of
interferometer is thousand times smaller than the diameter of proton. This small
change in length due to \acp{GW} can be easily drowned by different sources of
noises. The LIGO arms consist of a vacuum tubes which is one trillionth of
atmospheric pressure. This is important as even presence of few molecules will
scatter the laser beam and the moving air hitting the mirror will change the
length travelled by the laser beam mimicking as \ac{GW} signal.
 At lower frequencies(below 10 Hz), the seismic noise including earth
quake and other ground vibrations can move the mirror disturbing the \ac{GW}
signal. LIGO use both active and passive damping to isolate signal from noise.
Active damping includes some sensor that tracks ground vibration and moves the
test mass in out of phase to cancel the noise. The passive damping suspends the
40kg mirror by a four pendulum to isolate mirror from noise. Other sources of
noises are thermal noise(internal vibration and heating of mirror due to laser)
and shot noise(uncertainty in no. of quantized photons). 
Other sources of noise are narrow spectral lines that might imitate as \acp{CW}.
This will be discuss later in section{}, the lines are produced from power
lines, blinking LED's, hardware injections, resonance modes of mirror
suspensions. Some of the instrumental lines are shown in \ref{fig:CWnoise}
\begin{figure}[bht!]
	\includegraphics[width=\textwidth]{figure/CWnoise.png}
	\caption{The noise curve for the Livingston, Hanford and Virgo
interferometer during the \ac{O2} run. The spectral lines are known sources such as
60 Hz power line, blinking LED, calibration lines, violin modes. Figure from
~\cite{Abbott_2019} }
	\label{fig:CWnoise}
\end{figure}

\section{Sources of Periodic gravitational waves}
The continuous gravitational waves amplitudes are weaker than the violent black
hole merger. But the periodic waves lives for a longer time, so a longer
observational time is possible. The sources of periodic waves includes the
rotating neutron stars and a neutron star or a white dwarf in a binaries. A
rotating neutron star will emit a gravitational waves due to a deviation from
axisymmetry(mass quadrupole) or from a fluid oscillation inside neutron
star (current quadrupole). 

\subsection{Neutron Stars}

Neutron stars are one of the densest object in the universe, comparable of being
a giant nucleus. The density at the inner core will be around $10^15 g/cm^3$
which is 3 times the nuclear density to being few $g/cm^3$ at the crust. They
are formed in a Type II supernovae explosion of massive
star$(>8M_\odot)$.~\cite{Lattimer_2004}. The explosion happens when the core
mass exceeds the Chandrasekhar limit$(1.44M_\odot)$ where the electron
degeneracy pressure cannot withhold the gravitational pressure. The neutron star
masses range from $(1-2)M_\odot$ and radius of $(10-14)km$ which depends on
different equation of state. 
	
The neutron star structure can be divided into outer,inner crust and outer, inner
core.(this paragraph will follow cite). The atmosphere of neutron star varies
from few cm in hot \ac{NS}$(T\sim3\times10^6K)$ to few millimeters in cold one
$(T\sim3\times10^5K)$. The radiation from atmosphere gives a valuable
information of \ac{NS} surface temperature, chemical composition, magnetic
field, mass and radius. The outer crust goes deep into a density of
$\rho=10^{11g/cm^3}$ and contains non degenerate electron gas. The inner crust
can be up to a kilometer in depth and density might reach saturation nuclear density$(\rho_o=2.8\times10^{14}g/cm^3)$ and consists of
electrons, free neutrons and neutron-rich atomic nuclei. The outer core might be
several kilometer deep and density can reach up to $5\times10^{14}g/cm^3$ and
primary rich in neutrons.The inner core which density can reach $10\rho_o$ and
on this extreme matter density, it is hard to predict the composition. 

\subsection{Pulsars}
Pulsar are the rotating neutron star that produce a narrow \ac{EM} pulse that
acts like a lighthouse beam. When the narrow beam sweep to the line of sight of
the earth, we see the pulsation.  It was discovered by Jocelyn Bell in 1967, she
was seeing a continuous pulse every time the radio antennae was pointed to the
line of sight of pulsar. The pulsar are one of the most precise clock in the
universe. The spin period increases slowly due to the loss of rotational energy
in powering the nebulae, radiating gravitational waves. 

The pulsar magnetic moment are not aligned with the rotational axis, so the
magnetic moment changes when the pulsar rotates. This time varying magnetic
moment radiates energy known as magnetic dipole radiation. The power radiated by
pulsar in magnetic dipole radiation is:
\begin{equation}
P= \frac{B^2R^6\Omega^4\sin{\alpha}^2}{6c^3}
\end{equation}
where, B is a magnetic field, R is the radius, $\Omega$ is a rotational
frequency, $\alpha$ is angle between pulsar rotational axis and magnetic axis,
and c is the speed of light.  

The energy loss of pulsar can be estimated by the slowing down of pulsar
frequency. The energy of pulsar is written as :$E=\frac{1}{2} I \nu^2$. The
typical value of moment of inertia(I) of neutron star is $10^{38}kgm^2$. The
derivative of energy gives the rotational energy loss.
\begin{equation}
\frac{dE}{dt}= I\nu\dot{\nu}
\end{equation}

For a Crab pulsar of $\nu=29.65Hz$ and $\dot{\nu}= -3.68\times10^{-10}Hz/s$, the
rotational energy loss is $4.33*10^{31}W$. Most of the Crab rotational energy
are lost in powering Crab nebulae in synchrotron and inverse Compton radiation.
Only $1\%$ are lost  in the observed pulsation due to electromagnetic radiation.
The characteristic age of pulsar is given by:
\begin{equation}
t= \frac{1}{n-1}\frac{\nu}{\dot{\nu}}
\end{equation} 
The term n is known as braking index given by, $n=\nu\ddot{\nu}/\dot{\nu}^2$. 
The braking index value is n=1 for pulsar wind, n=3 for magnetic dipole
radiation, n=5 for mass quadrupole \ac{GW} radiation and n=7 for current
quadrupole radiation.

\section{Gravitational Wave from Neutron Star}
An individual spinning neutron star can emit a quasi-monochromatic \acp{GW} by
various mechanism. The signals strength are weak but long lived. The frequency
of continuous \acp{GW} decreases slowly due the loss of rotational energy. The
detection on continuous \ac{GW} will give a valuable information on neutron star
equation of state.
\subsection{Mountains on neutron stars}
A non-axisymmetric rotating star can produce a periodic \ac{GW} due to a time
varying mass quadrupole. Let us consider a asymmetric neutron star rotating
around z-axis with a moment of inertia $I_{xx}\neq I_{yy}=I_{zz}$ and
ellipticity,$\epsilon=(I_{xx}-I_{yy})/I_{zz}$

\begin{figure}[h!]                                                            
        \includegraphics[width=\textwidth]{figure/CW.png}                 
        \caption{continuous gravitational wave emission due to mountain in
neutron star. Image:Ra Inta}
        \label{fig:CW}                                                 
\end{figure}     

The principle of moment of inertia needs to be transform from rotating frame to 
the inertial frame as \acp{GW} lives in an inertial frame. We will use a rotation
matrix R for transformation.
$I_{inertial}= R^T I R$
where, 
\begin{equation}
R=
\begin{pmatrix}
\cos\theta & \sin\theta & 0 \\
-\sin\theta & \cos\theta & 0 \\
0 & 0 & 1
\end{pmatrix}
\end{equation}

\begin{equation}
I_{inertial}=
\begin{pmatrix}
(I_{xx}-I_{yy})(\cos2\theta -1) & \frac{1}{2}\sin2\theta(I_{xx}-I_{yy}) & 0 \\
\frac{1}{2}\sin2\theta(I_{xx}-I_{yy}) & (I_{xx}-I_{yy})(1-\cos2\theta) & 0  \\
0 & 0 & I_{zz}
\end{pmatrix}
\end{equation}

The double derivative of moment of inertia is given by:
\begin{equation} \label{2ndMI}
I_{\mu\nu}=-16\pi ^2f^2(I_{xx}-I_{yy})
\begin{pmatrix}
0 & 0 & 0 & 0 \\
0 & cos(4\pi\omega t) & 2sin(4\pi\omega t) & 0 \\
0 & 2sin(4\pi\omega t) & -cos(4\pi\omega t) & 0 \\
0 & 0 & 0 & 0 \\
\end{pmatrix}
\end{equation}

The gravitational wave luminosity of rotating is:
\begin{equation}\label{GWLuminosity}
L= \frac{G}{c^5}<\dddot{I}_{\mu\nu}\dddot{I}^{\mu\nu}>
\end{equation}

Combining \ref{2ndMI} \& \ref{GWLuminosity}, we get,
\begin{equation}\label{Lgw}
L_{grav}= \frac{32}{5}\omega ^6 I_{zz}^2\epsilon ^2
\end{equation}
where, $\epsilon =\frac{I_{xx}-I_{yy}}{I_{zz}}$ is defined as ellipticity. When
ellipticity is zero, there will not be any gravitational wave emission as there
won't be any non-axisymmetric components.

The gravitational wave frequency from non-axisymmetric neutron star will be
twice the spin frequency as the mass deformation will repeat after half
revolution.

The amplitude strain of \ac{GW} is given by:
\begin{equation}
h_{\mu\nu}=\frac{2G}{c^4d}\ddot{I}_{\mu\nu}
\end{equation}


\begin{equation} \label{2ndMI}
h_{\mu\nu}=-\frac{32\pi ^2\omega^2 \epsilon I_{zz}}{d}
\begin{pmatrix}
0 & 0 & 0 & 0 \\
0 & \cos(4\pi \omega t)(1+\cos ^2i) & 2\sin(4\pi \omega t)\cos i & 0 \\
0 & 2\sin(4\pi\omega t)\cos i & -\cos(4\pi\omega t)(1+\cos ^2i) & 0 \\
0 & 0 & 0 & 0 
\end{pmatrix}
\end{equation}
where i is the inclination angle which is the angle between the spin axis and
the inertial observer. When $i=90^\circ$,the amplitude strain will be:
\begin{equation}
h_0=\frac{4\pi ^2Gf^2 \epsilon I_{zz}}{c^4d}
\end{equation}
\subsection{$r$-modes}
The other kind of \ac{GW} emission is the non-radial oscillation of fluid
inside neutron star known as $r$-modes. These modes are unstable to
gravitational radiation so probably lives for a long time in a rotating neutron
stars. These osillating modes carry away the rotating energy from neutron star
slowing down its spin frequency. R-modes amplitude grow up to a large value just
after the formation of core collapse neutron star. This might explain the
slowing down of spin frequency of neutron star that are born in a Keplerian
limit.The detection of $r$-modes will help to
understand the interior and composition of neutron star. It may even give the
unknown spin frequency of neutron stars where the narrow beam of light is not
pointing towards earth. The theory of $r$-modes will be discuss in more detain in next chapter.
\begin{figure}
\centering
	\includegraphics[width=0.4\textwidth]{figure/333.png}
	\caption{$r$-modes pattern of a oscillating neutron star. Image:Chad
Hanna}
\end{figure}
The \ac{GW} luminosity due to $r$-mode emission can be written in geometrized
unit(c=G=1) as~\cite{Owen:2010ng}:
\begin{equation}\label{eq:modeenergy}
\frac{dE}{dt}=\frac{4}{25}M R^3 \alpha \omega^8\tilde{J}
\end{equation}
where, M,R,$\omega$, are the mass, radius, angular velocity of the neutron
star.$\alpha$ is the amplitude of $r$-modes. $\tilde{J}=1.635\times10^{-2}$ and
defined as~\cite{Owen:1998xg}:
\begin{equation}
\tilde{J}=\frac{1}{MR^3}\int_0^R\rho r^6dr.
\end{equation}
The rotational kinetic of neutron star can be written as:
$E=1/2I_{zz}\omega^2$. So, taking the derivative of energy will give:
\begin{equation}\label{eq:rotenergy}
\frac{dE}{dt}= I_{zz}\omega\dot{\omega}
\end{equation}
Combining Eqn ~\ref{eq:modeenergy} and \ref{eq:rotenergy}.
\begin{equation}\label{eq:braking}
\dot{\omega}=\frac{4}{25 I_{zz}}M R^3 \alpha\tilde{J} \omega^7 
\end{equation}
Let us define $4/(25 I_{zz})M R^3 \alpha\tilde{J}=K$, so\\
$\dot{\omega}=K \omega^7$
Taking derivative of \ref{eq:braking}.
\begin{equation}
\ddot{\omega}= K\omega^6\dot{\omega}
\end{equation}
Subsituting $K=\dot{\omega}/\omega^7$
\begin{equation}\label{eq:modebraking}
\frac{\omega\ddot{\omega}}{\dot{\omega}^2}=7
\end{equation}
The equation ~\ref{eq:modebraking} is defined as braking indices(n), and the
braking indices of pulsar emitting $r$-modes is 7. Braking torque generally
defines the spin down torque due to emission of energy. The braking indices for
pulsar wind is 1, 3 for magnetic dipole radiation and 5 for gravitational waves 
emission due to mountain on neutron stars. From eqn ~\ref{Lgw} we can easily
calculate the braking indices for \ac{GW} due to mountains on neutron star.
\subsubsection{$r$-mode amplitude}
The energy of the $r$-mode is proportional to the dimensionless amplitude
parameter $\alpha$. The amplitude parameter is roughly equal to the ratio of
velocity perturbation($\delta v$) to the rotational velocity at equator. The
intrinsic strain amplitude due to $r$-mode \ac{GW} can be written
as~\cite{Owen:2010ng}:
\begin{equation*}                                                               
h_0 =\sqrt{\frac{8\pi}{5}}\frac{G}{c^5}r^{-1}\omega^3 \alpha M R^3 \tilde{J}      
\end{equation*}                                                                 
\text{Inverting}                                                                
\begin{equation*}                                                               
\alpha = \sqrt{\frac{5}{8\pi}}\frac{c^5}{G} r \frac{1}{\omega^3} h_0 \frac{1}{M   
R^3 \tilde{J}}                                                                  
\end{equation*}                                                                 
\begin{equation*}                                                               
\alpha=\sqrt{\frac{5}{8\pi}}\frac{c^5}{G}\left(\frac{1}{M                       
R^3 \tilde{J}}\right)\left(\frac{1kpc\times 10^{-24}}{(2\pi \times
100)^3}\right)
\left(\frac{h_o}{10^{-24}}\right)\left(\frac{r}{1kpc}\right)\left(\frac{100Hz}{f}\right)^3
\end{equation*}                                                                 
Here we converted $\omega=2\pi f$, and r is the distance of neutron star from
the gravitational wave detector. \text{$M=1.4M_\odot$, $R=11.7$km \& $\tilde{J}=0.0164$}                         
\begin{equation} 
\alpha =0.028\left(\frac{h_o}{10^{-24}}\right)\left(\frac{r}{1kpc}\right)\left(\frac{100Hz}{f}\right)^3
\end{equation}           
%%%%%%%%%%%%%%%%%%%%%%%%%%%%%%%%%%%%%%%%%%%%%%%%%%%%%%%%%%%%%%%%%%%%%%%%%%%%%%%%%%%%%%%%%
%%%%%%%%%%%%%%%%%%%%%%%%%%%%%%%%%%%%%%%%%%%%%%%%%%%%%%%%%%%%%%%%%%%%%%%%%%%%%%%%%%%%%%%%%
%%% 							Chapter 2											  %%%
%%%%%%%%%%%%%%%%%%%%%%%%%%%%%%%%%%%%%%%%%%%%%%%%%%%%%%%%%%%%%%%%%%%%%%%%%%%%%%%%%%%%%%%%%
%%%%%%%%%%%%%%%%%%%%%%%%%%%%%%%%%%%%%%%%%%%%%%%%%%%%%%%%%%%%%%%%%%%%%%%%%%%%%%%%%%%%%%%%%


\chapter{\textbf{R-modes}}

\section{Introduction}
The other continuous gravitational wave emission is the oscillation of fluid
inside the neutron star which is the time varying current quadrupole.  R-modes
have been a subject of interest for many physicist as the oscillation of fluid
inside the neutron stars  will help in understanding the interior of neutron
star and its equation of state. R-modes are the current quadrupole  given by
velocity perturbation in azimuthal direction
where the restoring force is the Coriolis force. 
The velocity perturbation is given by~\cite{Owen:1998xg}:
\begin{equation}
\delta{\upsilon_j}=\alpha\omega R(r/R)^ l Y_j^{B,l,l}e^{i \omega t}
\end{equation}
where $\vec{Y}_{lm}^B$ is a magnetic type vector spherical harmonic given by:
\begin{equation}
\vec{Y}_{lm}^B=[l(l+1)]^{-1/2} r \vec{\nabla} \times (r \vec{\nabla}Y_{lm})
\end{equation}
It is similar to long
wavelength Rossby waves in Earth's atmosphere or ocean which are responsible for
heat circulation. R-modes are the monochromatic wave but the frequency decreases
slowly due to spinning down of neutron star rotational velocity.  The $l=m=2$.
modes couples with gravitational radiation, so these are the dominant modes.
\begin{figure}[h!]
  \centering
  \begin{minipage}[b]{0.5\textwidth}
    \includegraphics[width=\textwidth]{figure/rmodes.png}
  \end{minipage}
  \hfill
  \begin{minipage}[b]{0.5\textwidth}
    \includegraphics[width=\textwidth]{figure/rmodes1.png}
  \end{minipage}
\caption{Top figure: Top and side view of $l = m = 2$ $r$-mode oscillation.
Bottom figure: The four pattern flows backward with angular velocity $-1/3\Omega$. The
left path shows the motion of individual fluid element that depends on the mode
amplitude. Figure from ~\cite{lindblom2001neutron}.}
\end{figure}
 Viscous mechanism dampens the mode but a mechanism known as
Chandrasekhar-Friedman-Schutz(CFS) instability
~\cite{PhysRevLett.24.611}\cite{1978ApJ...222..281F} drives the R-modes due to
gravitational waves emission. CFS instability can be understood as, on a non
rotating star, gravitational waves releases positive angular momentum from
forward moving mode and negative angular momentum from backward moving mode
damping both modes. For a rotating star, backward moving modes are dragged
forward as viewed from an inertial frame emitting positive angular momentum. But
on rotating frame, modes has negative angular momentum, so the gravitational
waves increases the amplitude of modes instead of damping it. This discrepancy
of modes oscillation between rotating and an inertial frame is known as CFS
instability. This instability causes the amplitude of mode to
grow.~\cite{Owen_2000} 

\section{Viscous damping}
The $r$-modes amplitude grow exponentially and spin down the pulsar abruptly
unless some non-linear mechanism stops the growth~\cite{}. But milliseconds
pulsar exists, that shows $r$-modes saturates at some amplitudes. So, there
needs to be a damping mechanism that saturates the mode amplitude.  The fluid
inside neutron stars are subject to different viscous mechanism damping the
$r$-modes. The viscosity mostly depends on the temperature of the stars.

Shear viscosity has a shorter timescale at lower temperature that can be
calculated by neutron-neutron scattering cross sections~\cite{Owen_2000}. The dissipation
of mode energy is determined by momentum transport and neutrons, protons and
electrons energy present in the neutron star~\cite{1987ApJ...314..234C}. When
the temperature falls below superfluid temperature, the neutron and proton
transforms into superfluid states and doesn't contribute dissipation due to
momentum transport~\cite{1987ApJ...314..234C}. So, only electron-electron
scattering will able to transport momentum during superfluid state. The
dissipation time scale due to shear viscosity from neutron-neutron and
electron-electron scattering are~\cite{ANDERSSON_2001}:
\begin{align*}
t_{sv(nn)}&=\left(\frac{M}{1.4M_\odot}\right)^{-5/4}\left(\frac{R}{10km}\right)^23/4\left(\frac{T}{10^9K}\right)^2\\
t_{sv(ee)}&=\left(\frac{M}{1.4M_\odot}\right)^{-1}\left(\frac{R}{10km}\right)^5\left(\frac{T}{10^9K}\right)^2
\end{align*}
For $n=1$ polytrope star, $t_{sv(nn)}=6.7\times 10^7 s$ and
$t_{sv(ee)}=2.2\times 10^7 s$ . So, neutron star are more viscous in superfluid
state than normal state, which is different from generally known superfluidity
of $He_4$.~\cite{1987ApJ...314..234C}

Bulk viscosity arises from the compression and rarefaction of the fluid that
disturbs the beta equilibrium ($p+e^- \leftrightarrow n +
\nu_e$)~\cite{Owen_2000}. The neutrino carries away the energy from the stars.
Bulk viscosity dominates at higher temperature ($T>10^9 K$) but for $T > 10^{12}
K$, the stars becomes opaque to neutrino. This is because at higher temperature,
the neutrino mean free path length will be less  than the neutron star
radius~\cite{page2013stellar}. So, $r$-modes amplitude might grow
for a new born supernovae and release the rotational energy through
gravitational radiation. At lower temperature there won't be enough thermally
excited nucleons to undergo the Urca process.

The bulk viscosity timescale for n = 1 polytrope is~\cite{ANDERSSON_2001}:
\begin{equation}
t_{bv}=2.4 \times 10^{10}
\left(\frac{M}{1.4M_\odot}\right)^{-1}\left(\frac{R}{10km}\right)^5\left(\frac{T}{10^9K}\right)^{-6}\left(\frac{\nu}{1000 Hz}\right)^2
\end{equation}
So, we can see from above equation, the shear viscosity dissipation decreases
with temperature while the bulk viscosity increases with temperature.

The gravitational radiation makes the mode unstable in rotating stars while
viscosity damps the mode. The $r$-mode growth time scale needs to be fast enough not to be damped by the
viscosity. The $l=m=2$ modes has the shortest timescale and higher multipole has
significantly weaker instabilities~\cite{ANDERSSON_2001}. The timescale of
\ac{GW} radiation of $r$-modes is~\cite{ANDERSSON_2001}:
\begin{equation}
t_{gw}=-47                                                       
\left(\frac{M}{1.4M_\odot}\right)^{-1}\left(\frac{R}{10km}\right)^{-4} \left(\frac{T}{10^9K}\right)^{-6}\left(\frac{\nu}{1000
Hz}\right)^6
\end{equation}
where the negative sign means the mode is unstable.

\section{$r$-mode instability window}
The dissipative timescales can be written as~\cite{Owen_2000}:
\begin{equation}
\frac{1}{t}=-\frac{1}{2E}\frac{dE}{dt}=\frac{1}{t_{gw}}+\sum_v \frac{1}{t_v}
\end{equation}
The mode is stable when $t>0$ and unstable when $t<0$. 
The instability of mode is defined by a critical frequency such that:
$\frac{1}{t_{gw}}+\sum_v \frac{1}{t_v}= 0$
The gravitational radiation timescale depends on the spin frequency and viscous
timescale depends on temperature. 

The width of instability window increases with spin
frequency of neutron star\ref{fig:instability}. On new born supernovae remnants
$r$-modes amplitude grows up exponentially until it reaches a saturation point or
the waves breaks into daughter modes when non-linear hydrodynamic processes
comes in effect~\cite{Owen_1998}. Neutron star are born in a Keplerian limit,
and loose most of the rotational energy due to neutrino cooling.  R-modes might explain the spinning down of old
neutron star into a present low frequency. The evolution of neutron star also
depends on the saturation amplitude and spin frequency the r-modes enters
the instability window~\cite{Alford_2014}. The neutron star with
saturation amplitude($\alpha$)=1 can spin down in 12.3 years with final
temperature of $12.6 \times 10^8 K$ whereas with $\alpha=10^{-4}$ can spin-down a
pulsar in $6.11 \times 10^7$ years with temperature of $2 \times 10^8
K$~\cite{Alford_2014}. Some author claims
$r$-modes saturation amplitude is well below unity and amplitude greater than
unity decays into other fluid modes~\cite{Gressman_2002}. The $r$-mode can split
into daughter mode which amplitude grows exponentially if the parent mode
exceeds certain threshold, where the daughter mode takes energy from parent mode
\cite{Arras_2003}.
\begin{figure}[bht!]                                                            
        \includegraphics[width=\textwidth]{figure/rmodesamplitude.png}                         
	\caption{Left figure: The spin and temperature  evolution of $1.4 M_\odot$ neutron star
for different saturation amplitude and initial spin frequency entering the
instability window. The dotted curve shows the instability window. Right figure: The spin evolution with respect to time. Figure from ~\cite{Alford_2014}}
        \label{fig:instability}                                                     
\end{figure}  






\section{R-modes frequency}
The neutron stars interior are composed of neutron rich ocean. In a rotating
star, the liquid surface are subject to a Coriolis force similar to the Rossby
waves. These non-radial oscillation known as $r$-modes are unstable to the
gravitational radiation. They might live for a long time even in the presence of
viscosity.
The harmonic time dependent displacement vector of mode propagating in azimuthal
direction can be written as~\cite{ANDERSSON_2001}:
\begin{equation}\label{modesvector}
\vec{\xi}= \vec{\xi} e^{i(m\varphi+\omega_r t)}
\end{equation}
where m is the magnetic moment, $\omega_r$ is the oscillation of modes in
rotating frame.
The inertial frame are related to rotating frame by the equation:
\begin{equation}\label{inertialconvt}
\left(\frac{d}{dt}\right)_i=\left(\frac{\partial}{\partial t}\right)_r+\vec{u}.\nabla=\partial_t+ \Omega
\partial_\varphi
\end{equation}
Plugging \ref{modesvector} in \ref{inertialconvt}, we get,
\begin{equation}
\omega_i=\omega_r - m\Omega
\end{equation}

The frequency of modes in rotating frame is:
\begin{equation}\label{rotatingframe}
\omega_r = \frac{2m\Omega}{l(l+1)}
\end{equation}

The dominant $r$-mode is with spherical harmonic (l=m=2). The other mode has
weaker instability, the growth time of instability increases by a magnitude
with increase of l~\cite{ANDERSSON_2001}.
So, $\omega=2/3\Omega$, and plugging this into \ref{inertialconvt},
$\omega_i = -4/3\Omega$
The negative sign is the discrepancy of mode as seen in the rotating and
inertial frame and this discrepancy causes the modes amplitude to grow instead
of damping from gravitational radiation. 

\subsection{Factors affecting $r$-mode frequency}
The $r$-mode frequency will deviate from the Newtonian case when relativistic
error are considered. 


\subsubsection{General Relativity}
The $r$-mode frequency are dependent on a compactness of star which is the ratio
of mass over radius($M/R$).The compactness can be parametrized as:
\begin{equation}\label{compactness}
\frac{M}{R}=\left(\frac{M}{1.4M_\odot}\right)\left(\frac{10km}{R}\right)\left(\frac{1.4M_\odot}{10^4m}\right)
\end{equation} 
One solar mass in geometrized units(c=G=1) is equal to 1474. So, plugging the
numbers in \ref{compactness} gives:
\begin{equation}
\frac{M}{R}=0.21\left(\frac{M}{1.4M_\odot}\right)\left(\frac{10km}{R}\right)
\end{equation} 
The maximum compactness is given by~\cite{LATTIMER_2007}:
\begin{equation}
R \geq 2.83GM/c^2
\end{equation}
In geometrized units, the above equation will be $M/R \leq 0.35$
\citet{LATTIMER_2007} gives the probable minimum mass of neutron
star($1_M\odot$) will have radius of 14.5 km, making $M/R=0.103$.
\citet{Idrisy_2015} gives a conservative limit of compactness, $0.11 \leq M/R
\leq 0.31$. They have used the lowest mass as $1.02M_\odot$ and maximum mass of
neutron star as $2.76M_\odot$. The frequency of $r$-modes in relation to
compactness($M/R$) and spin frequency($\nu$) is~\cite{Idrisy_2015}: 
\begin{equation}
f=(-1.379 + 0.079(M/R) - 2.25(M/R)^2)*\nu
\end{equation}  
Plugging in the range of compactness(0.11-0.31), the $r$-mode frequency,
$f=(1.39-1.57)\nu$
The effect of general relativity will increase the $r$-mode frequency in an
inertial frame by few percentage.

\subsubsection{Rapid rotation}
The second order correction is important for a rapidly rotating neutron stars.
On, Newtonian star, the second order formula for $r$-mode frequency is given by 
\cite{Lindblom_1999}.
\begin{equation}
f/\nu=\kappa_0 - 2 + \kappa_2 \Omega^2/\pi G \rho_0
\end{equation}
where $\kappa_0=\sigma_I/\Omega$, $\kappa_2$ is a dimensionless parameter,
$\rho$ is a average mass density.
As $\kappa_0=2/3$, the above equation is:
\begin{equation}
f/\nu = -4/3 + \kappa_2 \Omega^2/\pi G \rho_0
\end{equation}

\begin{equation}
\frac{\Omega^2}{\pi G \rho_0}=0.145 \left(\frac{\nu}{716 Hz}\right)^2
\left(\frac{R}{10km}\right)^3 \left(\frac{1.4M_\odot}{M}\right)
\end{equation}

For the Crab with spin frequency $29.6 Hz$, taking mass and radius as
$1.4M_\odot$ and 12.53 km, and $\kappa_2 = 0.29$~\cite{Lindblom_1999} , the
rotational correction  will change $r$-mode frequency by 0.01\%. The correction
will be around 8\% for a fastest rotating neutron star. So, the rapidly rotation
correction will be less than the general relativity correction but it can be
significant for a pulsar rotating near mass shedding limit.

\subsubsection{Crust core coupling}
The crust are formed when neutron star cools off to temperature below
$10^{10}K$.  The friction between neutron fluids and the solid crust can dampen
the $r$-mode.  If the $r$-amplitude is high enough it can heat the crust-core
interface~\cite{Lindblom_2000}. The thermal heating can also lose in thermal
conduction and neutrino emission on core-crust boundary. The elastic storing
force on crust is less than the Coriolis force, so for a fast rotating neutron
star the crust will oscillate like a liquid with low shear
modulus~\cite{Levin_2001}. When the spin frequency increases, the $r$-mode
frequency will rise to meet the elastic crust modes causing avoided
crossing~\cite{Levin_2001}. The departure from the $r$-mode frequency in rotating
frame ($\omega_r=2/3\Omega$) is significant for spin frequency interval
$0.05\leq\Omega/\Omega_k\leq0.1$~\cite{Idrisy_2015}. The avoided crossing is
similar to the mode splitting in coupled pendulum by varying the length of one
pendulum. 

\subsection{$r$-mode frequency parameter after relativistic correction}
On \cite{Yoshida:2004gk} , the $r$-mode frequency is written in terms of correction due to the
compactness and rapid rotation. And, $r$-mode frequency is a decreasing function
of $T/|W|$, where $T/|W|$ is the ratio of rotational energy to the gravitational
binding energy. 
\begin{equation}\label{eq:T/W}
\frac{f}{\nu}=(1.41 - 1.50) - (1.23-1.95)T/|W|
\end{equation}	
The relativistic correction (1.41 - 1.50) are more precisely calculated to (1.39
- 1.57)~\cite{Idrisy_2015}. For, a mass shedding of 0.1,the equation ~\ref{eq:T/W} can be written as :
\begin{equation}
\frac{f}{\nu}=(1.39 - 1.57) - (0-0.195)\frac{\nu}{\nu_k^2}
\end{equation}	
where $\nu_k$ is the Keplerian velocity. The lower limit of the second order
correction of $r$-mode frequency is not known, so we choose it to zero. The
correct parameter of $r$-mode frequency will be discuss in details in the next
chapter. 
%%%%%%%%%%%%%%%%%%%%%%%%%%%%%%%%%%%%%%%%%%%%%%%%%%%%%%%%%%%%%%%%%%%%%%%%%%%%%%%%%%%%%%%%%
%%%%%%%%%%%%%%%%%%%%%%%%%%%%%%%%%%%%%%%%%%%%%%%%%%%%%%%%%%%%%%%%%%%%%%%%%%%%%%%%%%%%%%%%%
%%% 							END OF SECOND CHAPTER								  %%%
%%%%%%%%%%%%%%%%%%%%%%%%%%%%%%%%%%%%%%%%%%%%%%%%%%%%%%%%%%%%%%%%%%%%%%%%%%%%%%%%%%%%%%%%%
%%%%%%%%%%%%%%%%%%%%%%%%%%%%%%%%%%%%%%%%%%%%%%%%%%%%%%%%%%%%%%%%%%%%%%%%%%%%%%%%%%%%%%%%%

\chapter{How to search for gravitational waves from $r$-modes of known pulsars}
\section{Introduction}

In terms of strain, the most sensitive searches for \acp{GW} are those for
continuous waves, signals emitted by spinning neutron stars which need not be
in binaries.
The most sensitive searches for continuous \acp{GW} are those for known
pulsars, for which a timing solution derived from \ac{EM} observations
allows coherent integration of years of data~\cite[and references
therein]{Riles:2017evm}.
Most known pulsar searches (most recently~\cite{Authors:2019ztc}) have
targeted \ac{GW} frequencies precisely double (or occasionally equal to) the
observed spin frequency of each pulsar, based on electromagnetic pulse timing
and assuming an emission model of a ``mountain''---a mass quadrupole rotating
with the star.
Some known pulsar searches (most recently~\cite{Abbott:2019bed}) have instead
targeted narrow frequency bands of a fraction of a~Hz, losing some sensitivity
and costing more than fully targeted search, but allowing for some
uncertainties in the \ac{EM} timing parameters and in the physics such as the
possibility of free precession.

Neutron stars might also emit \acp{GW} via $r$-modes, rotation-dominated
quasi-normal modes driven unstable by gravitational radiation with frequencies
roughly 4/3 the spin frequency of the star~\cite[and references
therein]{Paschalidis:2016vmz}.
There are many uncertainties in the damping mechanisms that compete with the
instability and in the amplitudes attainable by $r$-modes due to nonlinear
hydrodynamics and other saturation mechanisms.
Within those uncertainties, $r$-modes might be oscillating at relatively low
amplitudes in fast spinning young pulsars up to several thousand years after
their birth, in rapidly accreting neutron stars, and in millisecond pulsars
perhaps a long time after accretion stops~\cite[and references
therein]{Glampedakis:2017nqy}.

So far \ac{GW} searches have only set upper limits on $r$-modes in broad band
(hundreds to thousands of~Hz) searches for non-pulsing neutron stars (most
recently~\cite{Abbott:2018qee}), but searches could be done for $r$-modes from
known pulsars too.
The main issue is determining the frequency band to search.
The uncertainty in $r$-mode frequency for a known pulsar is typically a
few~Hz~\cite{Idrisy:2014qca}, broader than previous narrow band pulsar
searches but not as broad as previous $r$-mode searches.
For long integrations, which are the most sensitive searches, some thought
needs to be given to spin-down parameters as well.

The main caveat is that $r$-modes might not truly be unstable (damping might
beat driving) in all or most neutron stars once all damping mechanisms are
taken into account.
Another caveat is that $r$-modes might be unstable but saturate at amplitudes
too small to be detectable with present and near-future detectors.
Predictions of saturation amplitudes~\cite{Arras:2002dw} are indeed too small
to detect for most pulsars and present detectors~\cite{Owen:2010ng}.
But calculations of saturation amplitude might be wrong and \ac{GW} detectors
are improving.
Also, it takes some time to develop and refine a \ac{GW} search, so it is
worthwhile to start now.

In this article we describe how to perform searches for \acp{GW} from
$r$-modes of known pulsars using minimal adaptations of existing code.
In particular the directed search pipeline used most recently in
Ref.~\cite{Abbott:2018qee}, based on the implementation of the
$\mathcal{F}$-statistic in LALSuite~\cite{LALSuite}, is easily adaptable for
this purpose.
We present a method for choosing a search parameter space and estimate
computational costs and sensitivities using that method.
(The parameter space was partially estimated once before~\cite{Ian}.)
We show that interesting searches of data from LIGO's \ac{O1} and \ac{O2} are
feasible already, and that future searches of more sensitive data sets will be
even more interesting.
Along the way we highlight the main issues affecting realistic observations,
which leads naturally to a list of suggestions for future work both by
theorists and by data analysts.

\section{Assumptions}

\subsection{Physics}

We assume that the \ac{GW} frequency evolution $f(t)$ in the reference frame
of the solar system barycenter is
\begin{equation}
\label{ft}
f(t) = f\left( t_0 \right) + \dot f\left( t_0 \right) \left( t-t_0 \right) +
\frac{1}{2} \ddot f\left( t_0 \right) \left( t-t_0 \right)^2,
\end{equation}
where $t_0$ is some reference time (often the beginning of the observation),
dots indicate time derivatives, and we shall use a simple $f$ to indicate
$f(t_0)$ from now on.
Physically, Eq.~(\ref{ft}) assumes that the signal frequency does not change
too fast (such as from glitches or higher derivatives) or too erratically
(such as from timing noise).
Timing noise is unlikely to be an issue for the integration times (a year or
less) considered here~\cite{Ashton:2014qya}.
Glitches can be avoided by checking \ac{EM} observations.
As we shall see below, higher derivatives are not a problem, and for some
searches even the second derivative is not needed.

Throughout we consider the lowest order (current quadrupole) $r$-mode, since
it is the fastest driven by \ac{GW} emission and the least damped by most
forms of viscosity.

We shall make frequent use of the ``spin-down limits''~\cite{Owen:2010ng} on
intrinsic \ac{GW} strain~\cite{Jaranowski:1998qm}
\begin{equation}
h_0^\mathrm{sd} \simeq 1.6\times10^{-24} \left( \frac{\mbox{1 kpc}}{r} \right)
\left( \frac{\left|\dot f\right|} {10^{-10}\mbox{ Hz/s}} \frac{\mbox{100 Hz}}
{f} \right)^{1/2}
\label{h0sd}
\end{equation}
(where $r$ is the distance to the pulsar) and on the $r$-mode amplitude
parameter~\cite{LMoO:1998prl}
\begin{equation}
\alpha_\mathrm{sd} \simeq 0.033 \left( \frac{\mbox{100 Hz}} {f} \right)^{7/2}
\left( \frac{|\dot{f}|} {10^{-10}\mbox{ Hz\,s}^{-1}} \right)^{1/2}.
\end{equation}
These correspond to the assumption that all of the observed $\dot f$ is due to
\ac{GW} emission via $r$-modes.
That is clearly an unrealistic assumption, especially when considering the
observed values of second derivatives too; but these limits serve as useful
milestones for search sensitivity.
These numerical forms of the limits assume certain neutron star structure
parameters and also assume that the ratio of \ac{GW} frequency to spin
frequency is 4/3, so they are uncertain by a factor of two or
so~\cite{Owen:2010ng}.
While they should not be taken too literally, these spin-down limits give a
rough idea of which \ac{GW} searches are most interesting.

There are many debates on the growth and damping timescales of the $r$-modes,
and on the saturation amplitude (which determines the long term strength of
\ac{GW} emission).
We largely bypass them, though we note that $\alpha$ is predicted to saturate
at order $10^{-4}$ or lower~\cite{Arras:2002dw}.
Theoretical uncertainties in these quantities are great and might best be
resolved by observations which do not rely too much on theoretical guidance to
pick targets.
Attempts at relatively model independent predictions or connections to
electromagnetic observations sometimes favor or disfavor one pulsar or
another.
Alford and Schwenzer~\cite{Alford:2012yn} argue that $r$-modes in young
neutron stars spinning down shut off above 60~Hz under a wide variety of
conditions, making PSR~J0537\textminus6910 the best candidate.
They also propose that the braking index (see below) could be as low as four
in some $r$-mode dominated pulsars rather than seven as implied for
constant~$\alpha$ evolution~\cite{Owen:1998xg}.
Certainly J0537\textminus6910 has the best (lowest)
$\alpha_\mathrm{sd}$ for all young pulsars~\cite{Owen:2010ng} (order
$10^{-1}$), and there are hints that the braking index between its frequent
glitches might be seven~\cite{Andersson:2017fow}.
There are also arguments that millisecond pulsars will emit \acp{GW} from
$r$-modes for a long time, but only at small amplitudes even if they have high
spin-down limits~\cite{Bondarescu:2013xwa, Alford:2014pxa}.
And temperature observations of some pulsars might indicate small $r$-mode
amplitudes due to constraints on viscous heating~\cite{Schwenzer:2016tkf}.
But as observers we note that the universe holds many surprises, and we
consider searches for any pulsar with an attainable spin-down limit.

An $r$-mode emits \acp{GW} at the mode's oscillation frequency $f$ in an
inertial frame of reference.
This frequency is a function of star's spin frequency $\nu$ of the
form~\cite[e.g.]{Yoshida:2004gk}
\begin{equation}
\label{fnu}
f/\nu = A - B \left( \nu / \nu_K \right)^2 + O(\nu)^4,
\end{equation}
where we have chosen the signs so that $A>0$ and $B>0.$
(The sign of the second term is generic to retrograde modes, such as all
$r$-modes, and is due to the effects of gravitational redshift and dragging of
inertial frames on the Coriolis force~\cite{Paschalidis:2016vmz}.)
Both parameters depend on the (usually unknown) mass $M$ and radius $R$ of the
neutron star and the still uncertain equation of state, and both are
dimensionless numbers of order unity.
The rapid rotation calculation of $r$-mode frequencies in
Ref.~\cite{Yoshida:2004gk} indicates that the $O(\nu)^4$ remainder in
Eq.~(\ref{fnu}) is negligible except (for some stars) when the star is
spinning almost at its Kepler frequency $\nu_K,$ the frequency at which
centrifugal force tears it apart.
Most pulsars, including those we find most interesting for this analysis, spin
at much less than the 716~Hz of the fastest observed
frequency~\cite{Manchester:2004bp}, so we will neglect the $O(\nu)^4$
remainder in Eq.~(\ref{fnu}).
We also assume that any backbending (nonmonotonic $f$ vs.\ $\nu$) due to a
possible phase transition~\cite{Glendenning:1997fy} occurs at much higher
frequencies and does not appreciably affect Eq.~(\ref{fnu}).

In Eq.~(\ref{fnu}) we neglect many physical effects which should have only a
small effect on the ranges of $A$ and $B.$
We assume that any differential rotation has a negligible effect on the mode
frequency.
In practice this seems likely to be true, even though the $r$-mode tends to
generate a small amount of differential rotation analogous to Stokes
drift~\cite{Friedman:2017wfi}.
We assume that the magnetic field's direct effect (through restoring force) on
the $r$-mode frequency is small.
Several studies (most recently~\cite{Jasiulek:2016epr}) indicate that this is
true for the relatively low magnetic fields of pulsars spinning in the LIGO
band.
While superfluidity generally has only a tiny effect on the mode
frequencies~\cite{Lindblom:1999wi}, it does split each normal fluid $r$-mode
into two where the neutron and proton fluids are co-moving or
counter-moving~\cite{Andersson:2001bz}; and the latter mode has an additional
restoring force due to entrainment of the two fluids.
These modes in general have slightly different frequencies, but the
counter-moving modes tend to be much more damped by mutual friction.
Thus we can assume that the \ac{GW} emission is almost all through co-moving
modes, which have frequencies almost identical to normal fluid modes.

More serious is the issue of avoided crossings between $r$-modes and other
modes.
For example, Levin and Ushomirsky~\cite{Levin:2000vq} pointed out that, as a
star spins down, the $r$-mode and a torsional mode of the solid crust
(vibrating at of order 100~Hz in a nonrotating star) will swap identities.
More sophisticated calculations~\cite[e.g.]{Glampedakis:2006ap} support this
idea.
A similar issue arises with the coupling between the $r$-modes and the buoyant
force responsible for $g$-modes~\cite{Kantor:2017xuo}.
In either case, in the vicinity of the other mode's frequency the avoided
crossing introduces large errors into the simple $r$-mode frequency dependence
posited in Eq.~(\ref{fnu}).
We shall neglect this, and assume that the pulsars we consider have $\nu$
safely away from the major avoided crossings.
In a search for many pulsars, most of them are likely to satisfy this
assumption; but it is a potentially serious issue worthy of more research.

The other potentially major effect is the coherence time of the $r$-modes.
If mode-mode coupling calculations such as Ref.~\cite{Brink:2004kt} are
correct, a saturated $r$-mode exists in rough equilibrium with two ``daughter
modes'' but may occasionally undergo abrupt phase shifts~\cite{Ira}.
A similar situation could hold if one attempts to take advantage of the
relatively broad band of frequencies to search for $r$-modes from a pulsar
which does not have \ac{EM} timing contemporary with a \ac{GW} data run---the
pulsar could have glitched during the \ac{GW} run, introducing a phase error
at a random time.
Such phase errors would reduce the sensitivity of coherent data analysis
methods described below~\cite{Ashton:2017wui}.

\subsection{Data analysis}

We assume a search method based on coherent integration using a minimal
adaptation of the code used in several broad band directed searches, most
recently Ref.~\cite{Abbott:2018qee}.
This code implements the multi-interferometer
$\mathcal{F}$-statistic~\cite{Jaranowski:1998qm, Cutler:2005hc}, which
combines matched filters in such a way as to account for the daily changes of
the interferometers' beam patterns as the Earth rotates.
This particular code implements the ``demodulated'' $\mathcal{F}$-statistic
first used in Ref.~\cite{Abbott:2003yq}.
A ``resampled'' implementation of the
$\mathcal{F}$-statistic~\cite{Patel:2009qe} could speed up computations
considerably, although it would present some difficulties in dividing up the
parameter space.
The $\mathcal{F}$-statistic also quickly maximizes signal-to-noise ratio over
the (typically unknown) angles describing the orientation of the neutron
star's spin axis.
In Gaussian noise, $2\mathcal{F}$ is drawn from a $\chi^2$ distribution with
four degrees of freedom.
In the presence of a signal, the $\chi^2$ is noncentral and the power
signal-to-noise ratio (if large) is approximately $\mathcal{F}/2.$

For some pulsars a wind nebula indicates the orientation of the star's spin
axis, and in that case a similar statistic called the
$\mathcal{G}$-statistic~\cite{Jaranowski:2010rn} can achieve slightly better
sensitivity.
Since the $\mathcal{G}$-statistic is not included in the implementation of the
$\mathcal{F}$-statistic in LALSuite~\cite{LALSuite} used in
Ref.~\cite{Abbott:2018qee}, we do not consider it further here; but it is a
natural avenue of future improvement.

We assume that the \ac{GW} search can use a single sky position.
Pulsar positions are typically known to sub-arcsecond
precision~\cite{Manchester:2004bp} and the sky resolution of a directed
continuous \ac{GW} search is two orders of magnitude less precise at the
frequencies of most pulsars~\cite[e.g.]{Abbott:2018qee}.

We do not assume we have a coherent \ac{EM} pulsar timing solution throughout
the \ac{GW} observation.
Narrow band searches for pulsars such as~\cite{Abbott:2019bed} are already
broad enough to extrapolate \ac{EM} timing from old observations, and since
our search bands turn out to be broader they are even more robust.
The main worry is whether there is a glitch during the \ac{GW} observation, in
which case it will effectively cut the integration time and reduce the
signal-to-noise ratio~\cite{Ashton:2017wui}.

We do assume that, in cases where the pulsar is a component of a binary, the
binary's orbital parameters are known well enough to avoid requiring a search
over them.
This is true for most binary pulsars except those in low mass x-ray binaries.
Since those are accreting and therefore the spin can undergo random walks, we
neglect them for our present purposes.

\section{Parameter space}

Under the assumptions stated above, the parameter space of a search consists
of ranges of $f,$ $\dot f,$ and possibly $\ddot f.$
With some uncertainties, these can be calculated as functions of the
\ac{EM}-observed spin $\nu$ and spin-down parameters $\dot\nu$ and $\ddot\nu.$

\subsection{General expressions}

Assuming that $A$ and $B$ do not change appreciably with time, the time
derivatives of Eq.~(\ref{fnu}) yield
\begin{eqnarray}
\label{fdot}
\dot f / \dot \nu &=& A - 3B \left( \nu/\nu_K \right)^2,
\\
\label{fddot}
\ddot f / \ddot \nu &=& A - \left( 3 + 6/n \right) B \left( \nu/\nu_K
\right)^2,
\end{eqnarray}
where $n = \nu \ddot \nu / \dot \nu^2$ is the braking index of the pulsar.
Thus an observation of $(\nu, \dot\nu, \ddot\nu)$ and calculations of $\nu_K$
and the ranges of $(A,B)$ determine the ranges of $(f, \dot f, \ddot f)$ to be
searched---in principle.
In practice $\ddot\nu$ tends to have significant uncertainties, affecting the
choice of $\ddot f$ as discussed below.

To get the frequency range of a search we insert the ranges of $A$ and $B$
into Eq.~(\ref{fnu}) and---for now---assume that $\nu_K$ is known to obtain
\begin{eqnarray}
(f/\nu)_{\min} &=& A_{\min} - B_{\max} \left( \nu/\nu_K \right)^2,
\\
(f/\nu)_{\max} &=& A_{\max} - B_{\min} \left( \nu/\nu_K \right)^2.
\end{eqnarray}

To determine the range of $\dot{f}$ for a given $f,$ note that
Eqs.~(\ref{fnu}) and~(\ref{fdot}) can be combined to write
\begin{equation}
\dot f / \dot \nu = f / \nu - 2B \left( \nu/\nu_K \right)^2.
\end{equation}
Then we simply have the range
\begin{eqnarray}
\left( \dot{f} / \dot\nu \right)_{\min} &=& f / \nu - 2B_{\max} \left(
\nu/\nu_K \right)^2,
\\
\left( \dot{f} / \dot\nu \right)_{\max} &=& f / \nu - 2B_{\min} \left(
\nu/\nu_K \right)^2.
\end{eqnarray}

Note that Eqs.~(\ref{fnu}) and~(\ref{fdot}) combined determine $A$ and
$B/\nu_K^2$ as
\begin{eqnarray}
A &=& \left( 3f / \nu - \dot{f} / \dot{\nu} \right) /2,
\\
B/\nu_K^2 &=& \left( f / \nu - \dot{f} / \dot{\nu} \right) / \left( 2\nu^2
\right).
\end{eqnarray}
These relations can be used for parameter estimation from a \ac{GW} detection:
Once $f$ and $\dot f$ are known, we find $A$ and $B/\nu_K^2,$ which in turn
can yield information on $M$ and $R$ and the equation of
state~\cite{Yoshida:2004gk, Idrisy:2014qca}.
The equations for $A$ and $B$ also can be used to write
\begin{equation}
\ddot{f} / \ddot{\nu} = \dot{f} / \dot{\nu} - (3/n) \left( f / \nu - \dot{f} /
\dot{\nu} \right),
\end{equation}
which in principle uniquely determines $\ddot f$ in terms of $f,$ $\dot f,$
and \ac{EM}-observed quantities.

In practice, $\ddot\nu$ (or equivalently $n$) measurements are available only
for a few pulsars; and even then they may have large errors when measured over
short baselines.
For example, the monthly fits to $\ddot\nu$ for the Crab pulsar provided by
Jodrell Bank can vary by a factor of a few from the long term average and can
even change sign~\cite{Crab2015}.
It is not clear how much of this timing noise is due to magnetospheric effects
and how much is due to a genuine fluctuating torque on the star.
Since the $r$-mode frequency is determined mainly by the Coriolis force, we
are interested in the latter but not the former.
However at the moment we wish to be cautious in our choice of parameter space.
As a practical matter, current codes including that used in
Ref.~\cite{Abbott:2018qee} cut the parameter space into computing batch jobs
in a way such that the range of $\ddot f$ can depend on $f$ but not on $\dot
f.$
Plugging in the full range of $\dot f$ we can get
\begin{eqnarray}
\left( \ddot f / \ddot\nu \right)_{\min} &=& f/\nu - 2(1+3/n) B_{\max} \left(
\nu/\nu_K \right)^2,
\\
\left( \ddot f / \ddot\nu \right)_{\max} &=& f/\nu - 2(1+3/n) B_{\min} \left(
\nu/\nu_K \right)^2
\end{eqnarray}
as functions of $f.$
To err on the safe side by covering more parameter space, we can take the
minimum $\ddot f$ as zero.
Since our ``safe side'' $B_{\min}$ vanishes (see below), the maximum $\ddot f$
can be taken to be simply $\ddot\nu f/\nu,$ using the highest $\ddot\nu$
observed during the \ac{GW} integration.

At the moment these overly broad parameter ranges are not a concern, because
\ac{O1} searches are computationally cheap and even \ac{O2} searches are not
extravagant (see below).
These parameter ranges could be refined later for longer searches, when
computational cost is more of an issue, for example by calculating the range
of $B$ and exploring its consequences.

\subsection{Numerical ranges of parameters}

The range of $A$ is fairly well known.
The most recent calculation~\cite{Idrisy:2014qca} used the general
relativistic slow rotation approximation~\cite{Lockitch:2000aa,
Lockitch:2002sy} to compute $A$ for a variety of neutron-star equations of
state, obtaining 1.39 $\le A \le$ 1.57 depending almost purely on $M/R.$
Since that calculation was published, the big new constraint on the neutron
star equation of state is the lack of a large tidal effect in the binary
neutron-star merger GW170817~\cite{TheLIGOScientific:2017qsa}.
This disfavors large radii and low $A.$
But for the rest of this paper, to be conservative (cover a wide range of
parameters), we shall use the $A_{\min}$ and $A_{\max}$ quoted above.

The range of $B$ is less well known than the range of $A.$
The best general relativistic calculation~\cite{Yoshida:2004gk} drops the slow
rotation approximation, but adds the Cowling approximation (neglecting the
metric perturbation) and gives numbers only for two equations of state and two
$M/R$ values.
And the equations of state are polytropes rather than realistic equations of
state with the adiabatic index varying depending on the density.
The errors due to the Cowling approximation can be estimated as a few percent,
which is not of too much concern here; but the uncertainty from the stellar
models is more serious.

We estimate the range of $B$ from Ref.~\cite{Yoshida:2004gk} as follows:
Their Eq.~(15) gives $B\left( \nu/\nu_K \right)^2$ as 1.23--1.95 times the
ratio of kinetic to potential energy for the four stellar models considered.
For our purposes the most interesting model is their model $c,$ a polytrope of
adiabatic index~2 and $M/R=0.1$ which yields the number 1.95 (and for which
the slow rotation approximation is very accurate all the way to the Kepler
frequency).
In Fig.~2 of Ref.~\cite{Yoshida:2004gk} the sequence for model $c$ terminates
at a kinetic-to-potential energy ratio of 0.1, and this termination point
corresponds to $\nu=\nu_K,$ the ``Kepler frequency'' or maximum spin frequency
of the star.

Hence we can write $B \simeq 0.195$ for this stellar model, which should set a
safe upper limit on $B_{\max}$ for the following reasons:
The results of Ref.~\cite{Yoshida:2004gk} show that $B$ increases for smaller
$M/R$ and for lower adiabatic index (higher polytropic index).
The value $M/R=0.1$ for their model $c$ is smaller than post-GW170817 bounds,
indicating we are safe there.
An adiabatic index of~2 also errs on the safe side, since piecewise polytropic
fits to realistic equations of state~\cite{Read:2008iy} yield higher indices.
The bound on $B_{\min}$ is less clear without detailed calculations, but
$B_{\min}=0$ is well beyond the range quoted and should be safe.

Last we consider $\nu_K.$
The fastest observed spin frequency for a neutron star is about
716~Hz~\cite[and references therein]{Paschalidis:2016vmz}.
The Kepler frequency is expected to scale roughly as its Newtonian dependence
$M^{1/2} R^{-3/2},$ even in general relativity~\cite[and references
therein]{Paschalidis:2016vmz}.
Neutron star radii are roughly constant for a given equation of state, while
reliable mass measurements range over almost a factor of
two~\cite[and references therein]{Ozel:2016oaf}.
To err on the safe side (high $B_{\max}$), we assume that the 716~Hz pulsar is
on the high end of the mass range and our pulsar is on the low end so that it
has a lower Kepler frequency.
Then we can safely take $\nu_K = \mbox{716 Hz}/\sqrt{2} \simeq 506$~Hz, erring
on the safe side by assuming a possible factor of two difference in mass.

To summarize, we recommend for the moment a broad parameter space with ranges
\begin{eqnarray}
\label{range0}
f_{\min} = \nu \left( A_{\min} - B_{\max} \frac{\nu^2} {\nu_K^2} \right),
&\quad&
f_{\max} = \nu\, A_{\max},
\\
\label{range1}
\dot f_{\min} = -\dot\nu \left( \frac{f}{\nu} - 2B_{\max} \frac{\nu^2}
{\nu_K^2} \right),
&\quad&
\dot f_{\max} = -\dot\nu \frac{f}{\nu},
\\
\label{range2}
\ddot f_{\min} = 0,
&\quad&
\ddot f_{\max} = \ddot\nu \frac{f}{\nu},
\end{eqnarray}
where $\ddot\nu$ is the maximum value consistent with \ac{EM} observations,
and the other parameters are $A_{\min} = 1.39,$ $A_{\max} = 1.57,$ $B_{\max} =
0.195,$ and $\nu_K = 506$~Hz.

\section{Computational cost}

We rely heavily on the search parameter space metric~\cite{Wette:2008hg}
\begin{equation}
g_{ij} = \pi^2 \left(
\begin{array}{ccc}
T^2/3 & T^3/6 & T^4/20 \\
T^3/6 & 4T^4/45 & T^5/36 \\
T^4/20 & T^5/36 & T^6/112
\end{array}
\right)
\end{equation}
where the indices are labeled in the order $(f, \dot f, \ddot f)$ and also can
be labeled 0, 1, 2.
Here $T$ is the time from beginning to end of the integration.
This metric controls the density and placement of templates (parameter values
of matched filters) by relating coordinate distances (parameter differences)
to loss of signal-to-noise ratio~\cite{Owen_1996}.
The mismatch $g_{ij} \Delta \lambda^i \Delta \lambda^j$ between two signals
with parameters displaced by $\Delta \lambda^i$ is the fractional loss in
optimal power signal-to-noise ratio due to filtering one with the parameters
of the other.
In continuous \ac{GW} searches template banks are often constructed so that
the worst case mismatch between any signal and the nearest template is 0.2.
We calculate metric components neglecting the amplitude modulation of the
$\mathcal{F}$-statistic and including only phase terms, an approximation which
works well for integrations of many days~\cite{Prix:2006wm}.

To determine which pulsars require $\ddot f,$ we compute $g_{22} \ddot
f_{\max}^2$ to determine the maximum mismatch due to neglecting $\ddot f.$
(Note that this does not allow for the possible mitigating effect of varying
$f$ and $\dot f$ somewhat, and thus it is a conservative estimate.)
If the mismatch is comparable to or greater than 0.2, $\ddot f$ is needed.
First we evaluate this criterion using $\ddot\nu$ values taken from the ATNF
catalogue~\cite{Manchester:2004bp}.
In many cases these values are unknown or are known to be contaminated by
timing noise (such as when they are negative).
However the values of $\nu$ and $\dot\nu$ are typically well measured.
Hence we also check the need for $\ddot f$ using $\ddot\nu = 7 \dot\nu^2 /
\nu$ (implied by a braking index of 7), and if either this mismatch or the one
using the observed $\ddot\nu$ satisfies the criterion we consider a search
over $\ddot f$ to be necessary.

We consider three values of $T:$
First $1.12\times10^7$~s and $2.32\times10^7$~s, the lengths of \ac{O1} and
\ac{O2} respectively, then one year or $3.15\times10^7$~s which might be
characteristic of future LIGO runs~\cite{Aasi:2013wya}.
We only consider pulsars with $f_{\max}$ greater than 10~Hz since, even at
design sensitivity, LIGO noise increases rapidly below that frequency and the
$r$-mode amplitudes required to emit at the spin-down limit become enormous.
We find that for \ac{O1} the Crab, Vela, and several others need $\ddot f;$
while for \ac{O2} and a one-year integration most pulsars with measured
$\ddot\nu$ and several without it need $\ddot f.$

We test the need of the third frequency derivative using $g_{33} = \pi^2 T^8 /
2025$~\cite{Wette:2008hg}, the third derivative of $\nu$ from the ATNF
catalogue~\cite{Manchester:2004bp} when it is given, and the $n=7$ value of
$91 \dot\nu^3 / \nu^2$ for the third derivative when it is not given.
(The numerical factor 91 is $n(2n-1)$ in general, and can be obtained by
differentiating the definition of the braking index.)
For \ac{O1} no pulsar needs a third derivative.
For \ac{O2} no pulsar with a spin-down limit above the noise (see below) needs
a third derivative.
For a one year integration the Crab (alone of the pulsars detectable at the
spin-down limit) is on the edge of needing a third derivative.
Since this is a conservative estimate and the need can be mitigated by
slightly shortening $T$ and our focus here is on how to adapt current data
analysis codes, which do not include the third derivative, we do not address
third derivatives further here.

Under the assumption that two frequency derivatives are needed and three are
not, the proper volume of the parameter space $\sqrt{g} \int df\, d\dot f\,
d\ddot f$ integrated over the ranges in Eq.~(\ref{range0})--(\ref{range2}) is
approximately
\begin{equation}
\label{propvol}
\sqrt{g} \nu \left| \dot\nu \right| \ddot\nu B_{\max} \left( \nu/\nu_K
\right)^2 \left[ A_{\max}^2 - A_{\min}^2 \right].
\end{equation}
(Here we have dropped the $B$ terms in $f_{\min}$ and $f_{\max},$ since they
are small corrections, and $g$ indicates the determinant of the metric.)
To estimate the number of templates, we divide by the proper volume per
template~\cite{Owen_1996},
\begin{equation}
V = \left( 2 \sqrt{\mu/3} \right)^3 \simeq 0.138
\end{equation}
for a three-dimensional template bank mismatch $\mu$ of 0.2.
In cases where $\ddot f$ is not needed, these expressions need to be modified,
but those cases are so computationally cheap that they are not an issue.
In practice the number of templates is modified from these estimates by the
vagaries of the actual template placement code, typically dominated by the
problem of covering the edges of long narrow stretches of parameter space.
Our tests with LALSuite~\cite{LALSuite} show that a real search might use
three times as many templates as these ideal numbers.
Since this factor can vary for each search, we do not include it further or
attempt to estimate the numbers too precisely.

To get values for the number of templates, we use $\sqrt{g} \simeq T^6 \times
8.41\times10^{-3}$ where $T$ is measured in seconds.
We find that for \ac{O1} using ATNF values of $\ddot\nu,$ the Crab requires
$6\times10^9$ templates and the others generally require one or more orders of
magnitude fewer than the Crab.
Taking catalogue numbers at face value, the exception is J0537\textminus6910
which requires $1.5\times10^{10}$ templates.
Using the maximum $\ddot f$ derived from a braking index of 7, the Crab
triples to about $2\times10^{10}$ templates.
Using ATNF spin-downs, for \ac{O2} the Crab and PSR~J0537\textminus6910
require of order $5\times10^{11}$ and $1\times10^{12}$ templates, and for a
one year integration they require $3\times10^{12}$ and $7\times10^{12}.$
The latter number is the same as the first directed search for an isolated
neutron star~\cite{Abadie:2010hv}, whose cost was modest by the standards of
continuous \ac{GW} data analysis.

Again, these numbers are likely to be larger in reality due to the template
placement algorithms in LALSuite~\cite{LALSuite}.
And, although we make rough blanket statements here, for a real search each
pulsar needs some investigation into timing noise and glitches.
For example, while PSR~J0537\textminus6910 might be a very interesting pulsar
to search, it is known to glitch frequently and because it is visible only in
x-rays it is important to maintain satellite timing~\cite{Andersson:2017fow}.
Even with timing, if this pulsar glitches in the middle of a \ac{GW} observing
run, the run will need to be divided into segments.
For another example, since in \ac{O2} the noise performance of the Hanford
interferometer was usually significantly worse than the Livingston
interferometer at the frequencies of most pulsars with high spin-down limits,
some pulsar searches might use only data from Livingston with little loss in
sensitivity.

To convert template numbers to computational cost, we run a piece of the
latest directed search code used in~\cite{Abbott:2018qee} to estimate the
computational cost per \ac{SFT} of 30 minutes of data per template.
On the LIGO-Caltech cluster Broadwell and Skylake benchmarking nodes the cost
is generally somewhat less than 50~ns per \ac{SFT} per template, depending on
network activity and disk throughput.
For \ac{O1} the number of \acp{SFT} was about $6\times10^3.$
For \ac{O2} the number is about double, but we still use $7\times10^3$
assuming a search which does not integrate data from the Hanford
interferometer because its noise is significantly worse than that at
Livingston for the low frequencies considered here.
For a one year search of future data we assume two interferometers at a duty
cycle of 70\% each, comparable to the most stable past operation of the
interferometers, resulting in $2.5\times10^4$ \acp{SFT}.

Under these assumptions the cost of a Crab search is of order 500 core-hours,
$5\times10^4,$ or $1\times10^6$ for \ac{O1}, \ac{O2}, or one year
respectively.
This indicates that searching all pulsars for \ac{O1} and \ac{O2} is not a
computational problem, even though the number of templates is likely to be
larger in reality.
The one-year figure for the Crab is comparable to the total power used in a
bundle of recent directed searches~\cite{Abbott:2018qee}.
It is not outrageously expensive, but indicates that soon it will be desirable
to reduce the costs through a combination of theory and data analysis
innovations.

The density of templates per unit frequency is useful in estimating
sensitivity (below) and in load balancing the code.
Derived similarly to the proper volume~(\ref{propvol}) but omitting the $\int
df$ and dividing by the volume per template $V,$ this density is approximately
\begin{equation}
2\sqrt{g} B_{\max} \left( \nu/\nu_K \right)^2 \left| \dot\nu \right| \ddot\nu
f/\nu / V.
\end{equation}
For \ac{O1} at the maximum end of the frequency ranges, this takes the values of
$1.1\times10^9$ and $1.4\times10^9$\,Hz$^{-1}$ for the Crab and J0537
respectively.
For \ac{O2} the corresponding numbers are about $9\times10^{10}$ and
$1.1\times10^{11},$ and for a one year integration they are about
$5\times10^{11}$ and $7\times10^{11}.$

\section{Sensitivity}

\begin{figure*}
\includegraphics[angle=270,width=\linewidth]{figure/rmeth1}
\caption{
\label{fig1}
Spin-down limits for interesting pulsars (horizontal lines) and sensitivity
estimates (other curves), both in terms of intrinsic strain vs.\ \ac{GW}
frequency.
Spin-down limits are taken from Eq.~(\ref{h0sd}) in the text.
Sensitivity estimates are taken from Eq.~(\ref{h0Td}) and the paragraph
containing it.
}
\end{figure*}

\begin{figure*}
\includegraphics[angle=270,width=\linewidth]{figure/rmeth2}
\caption{
\label{fig2}
Same as the previous figure, for higher frequencies.
There are fewer pulsars here, but the spin-down limits on $r$-mode amplitude
are generally closer to predictions of saturation amplitude.
}
\end{figure*}

We express the sensitivity of each search in terms of upper limits on $h_0$
that can be placed in the absence of a detection.
This is slightly pessimistic---the upper limits are conservative by design and
it is plausible that a somewhat fainter signal could be detected---but it
facilitates comparison with published upper limits from previous searches for
continuous \acp{GW}.
The precise definition of $h_0^\mathrm{UL}$ we use is the same as for instance
in Ref.~\cite{Abbott:2018qee}.
It is a 95\% confidence limit on a population of injected signals with fixed
$h_0$ but varying frequency (within a small band), frequency derivatives, and
angles of inclination and polarization.

The sensitivity of a search of data from a single detector with stationary
noise can be expressed as~\cite{Wette:2011eu}
\begin{equation}
\label{h0Td}
h_0^\mathrm{UL} = \frac{5}{2} \hat\rho \sqrt{ \frac{S_h} {T_d} },
\end{equation}
where $S_h$ is the strain noise \ac{PSD} and $T_d$ is the amount of data.
(In general $T_d$ is less than the integration span $T$ times the number of
interferometers due to maintenance, earthquakes, and so on.)
For multiple detectors or non-stationary noise the \ac{PSD} in
Eq.~(\ref{h0Td}) is replaced by a weighted sum~\cite{Jaranowski:1998qm,
Cutler:2005hc}.
For observations of many days at most sky locations, the sum is very close to
the harmonic mean of noise \acp{PSD}, so we will use the harmonic mean when we
give numbers later.
The statistical factor $\hat\rho$ is iteratively estimated to sufficient
precision using the method of Wette~\cite{Wette:2011eu}, using the template
densities above and assuming that the upper limits are placed on 0.1\,Hz
frequency bands.
(This upper limit band might be chosen differently for different searches, but
its effect on sensitivity is negligible.)
The factor $5\hat\rho/2$ ranges about 33--38 for the searches considered here,
comparable to the factor for directed searches~\cite{Abbott:2018qee} and about
triple the factor for exact timing searches of known
pulsars~\cite{Authors:2019ztc}.
As an alternative, one can write this in terms of sensitivity depth, or the
square root of the \ac{PSD} divided by $h_0$ as in
Ref.~\cite{Dreissigacker:2018afk}.
For the searches considered here, the sensitivity depth is on the order of
110--170\,Hz$^{-1/2}.$
This is comparable to values achieved~\cite{Dreissigacker:2018afk} with narrow
band pulsar searches such as Ref.~\cite{Abbott:2019bed}, or somewhat better
due to longer integration times.
We consider noise \acp{PSD} for \ac{O1}~\cite{H1O1, L1O1}, \ac{O2}~\cite{H1O2,
L1O2}, Advanced LIGO design~\cite{Design}, and the recently funded A+
design~\cite{A+}.
Hence we show four sensitivities:
\ac{O1}, \ac{O2}, and one year integrations at Advanced LIGO and A+ design.

In Figs.~\ref{fig1} and~\ref{fig2} we plot our sensitivity measure vs.\
frequency for the four cases mentioned above, superposed on a set of spin-down
limits for known pulsars from the ATNF catalogue~\cite{Manchester:2004bp}.
Most of the pulsars plotted have already been searched for \acp{GW} at
$f=2\nu$ (and some also at $f=\nu$) in previous LIGO and Virgo papers based on
exact pulsar timing solutions~\cite{Authors:2019ztc}.
Most of the pulsars whose spin-down limits are accessible with existing data
are young and energetic (and sometimes glitchy) like the Crab, and most are
shown in Fig.~\ref{fig1}.
As with known-timing searches, the Crab is the first spin-down limit to become
accessible (already in \ac{O1}), and several more including Vela soon follow.
For later noise curves some middle-aged pulsars (in Fig.~\ref{fig1}) and some
recycled millisecond (in Fig.~\ref{fig2}) pulsars become accessible.
Most notable in Fig.~\ref{fig2} are J0537\textminus6910 and
J0437\textminus4715.
Due to its frequency and proximity to Earth, the latter has
$\alpha_\mathrm{sd}$ of order $10^{-5}$---much lower, and hence more feasible,
than the other accessible pulsars, although $h_0^\mathrm{sd}$ indicates this
pulsar will require at least A+ to detect.

Not all of these pulsars are timed concurrently with LIGO-Virgo observing
runs.
Since the searches proposed here cover broad frequency bands, the uncertainty
in frequency and spin-down parameters is not an issue---unlike the $\nu$ and
$2\nu$ searches.
Our proposed searches do suffer in sensitivity if a pulsar glitches during the
integration, though; and they cannot account for the fluctuating torques
likely in accreting systems.
The glitch issue means that frequent x-ray timing of J0537\textminus6910 will
be important for future \ac{GW} observing runs~\cite{Andersson:2017fow}.

\section{Discussion}

We have shown that searches for continuous \acp{GW} from $r$-modes of known
pulsars can beat the spin-down limits on some pulsars in existing data for
reasonable computational costs.
Although the $r$-mode amplitudes required for detection in such data are
higher than predicted by theory, this work serves as a starting point for
future improvements.
Spin-down limits for many more pulsars will be attainable in the next few
years.

Part of our goal is to point out what theory work could be most important to
help observations.
It is crucial to get the range of mode frequencies and spin-down parameters
right, and helpful to narrow the range down and reduce costs.
Relating frequencies and spin-down parameters more precisely to neutron star
properties will also help measure the latter once a signal is detected.

The most important feature of the search is the $r$-mode frequency range.
Avoided crossings such as that with $t$-modes in the crust could widen the
parameter ranges of some pulsars well beyond what we consider here.
Some pulsars could be undetectable without addressing the avoided crossings
problem.
Updated ranges of the $A$ and $B$ parameters of Eq.~(\ref{fnu}) would also
help in terms of narrowing the parameter space and hence reducing the
computational costs, which will grow to be substantial in coming years.

On the computational side, a pipeline that divides parameter space in a way
suitable for use of the resampled $\mathcal{F}$-statistic~\cite{Patel:2009qe}
would allow for significantly reduced computational cost for long
integrations.

More estimates of saturation amplitudes would be helpful.
This is a very difficult problem, and essentially has been addressed only by
one approach~\cite{Arras:2002dw}.

It would also help to be sure of the coherence time.
If saturated $r$-modes in equilibrium with other modes occasionally experience
phase jumps, this would render long coherent integrations of \ac{GW} data
problematic.

The coherence time issue leads into future work for data analysis:
Accretion and glitches can also introduce issues which encourage development
of alternatives to the straightforward coherent integrations considered here.
Adaptations of semi-coherent techniques developed for other
searches~\cite{Sun:2017zge, Suvorova:2017dpm, Dergachev:2011pd,
Ashton:2018qth} could be fruitful for $r$-modes from known pulsars too.
For those pulsars with a known inclination angle, upgrading from the
$\mathcal{F}$-statistic to the $\mathcal{G}$-statistic will also improve
sensitivity.

\section*{acknowledgments}

We are grateful to the continuous waves search group of the LIGO Scientific
Collaboration, particularly Ian Jones and Karl Wette, for helpful discussions.
This work was supported by NSF grant PHY-1607673.
This paper has been assigned document number LIGO-P1900173.













%%%%%%%%%%%%%%%%%%%%%%%%%%%%%%%%%%%%%%%%%%%%%%%%%%%%%%%%%%%%%%%%%%%%%%%%%%%%%%%%%%%%%%%%%
%%%%%%%%%%%%%%%%%%%%%%%%%%%%%%%%%%%%%%%%%%%%%%%%%%%%%%%%%%%%%%%%%%%%%%%%%%%%%%%%%%%%%%%%%
%%% 							END OF Fourth CHAPTER								  %%%
%%%%%%%%%%%%%%%%%%%%%%%%%%%%%%%%%%%%%%%%%%%%%%%%%%%%%%%%%%%%%%%%%%%%%%%%%%%%%%%%%%%%%%%%%
%%%%%%%%%%%%%%%%%%%%%%%%%%%%%%%%%%%%%%%%%%%%%%%%%%%%%%%%%%%%%%%%%%%%%%%%%%%%%%%%%%%%%%%%%

\chapter{\textbf{Data Analysis}}
\section{Introduction}
Rapidly rotating neutron stars emits continuous gravitational waves due to the
axis asymmetry or oscillation of the fluids. The pulsar are slowly spinning down
due to loss in rotational energy, so the gravitational waves might not
necessarily be monochromatic for a long duration. And, the \acp{GW} are also
Doppler shifted by the daily rotation and orbital motion of the earth. Since,
the interferometer antennae pattern response to the direction of the sources,
the signals are also amplitude modulated due to rotation of earth. The
significant part of the continuous gravitational wave data analysis is to match
the quasi-monochromatic waves to the theoretical waveform. The observational
time needs to be long enough  to extract the \ac{GW} signal buried on
the noisy data. 

The continuous \acp{GW} search can be divided into different categories,
according to the information of the sources. The pulsar like Crab are routinely
monitored in radio observation, so we know the spin frequency and sky
coordinates of Crab pulsar. The neutron star radiates a \ac{GW} twice the spin
frequency due to the mass quadrupole. So, the search with known waveform of the
\ac{GW} is known as Targeted search. They are computationally cheaper and more
sensitive.The $r$-modes frequency deviates from the Newtonian case when
relativistic correction are considered. 
Other type of search is Directed search, the sky coordinates are known but not
the spin frequency. The supernovae remnant like Cassiopeia A are not seen
emitting the pulses, it's probably because the narrow beam of light doesn't
sweep towards the earth. So, a large frequency band is required to search
\ac{GW} from Cassiopeia A which increases the computational cost. The type of
search for unknown neutron star are known as All sky search. There are around
$10^8$ neutron star population in Milky way but only 2000 are observed. Some of
the neutron stars that are not observed in \ac{EM} radiation can be detected in
\ac{GW}.  One of the all sky
search was done for the whole sky from 20 to 1922 Hz~\cite{Abbott_2019a}. Due to
the computational limitation, the all-sky search have short obeservational time
so they are less sensitive than other type of continuous \ac{GW} searches.

\begin{table}[ht]                                                               
\centering                                                                                                                                      
\begin{tabular}{lcccc}                                                          
\hline                                                                          
Type of Search&Sky Location&frequency&Computational Cost&Sensitivity\\          
\hline                                                                          
Targeted&Known&Known&\tikzmark{a}{}&\tikzmark{c}{}\\                            
Narrowband&Known&Known&&\\                                                      
Directed&Known&Unknown&&\\                                                      
All Sky&Unknown&Unknown&\tikzmark{b}{}&\tikzmark{d}{}\\                         
\hline                                                                          
\end{tabular}                                                                   
\link{a}{b}\link{d}{c}                                                          
\caption{Different types of continuous gravitational wave search}
\label{tab}                                                                    
\end{table}         

 On this chapter we will discuss the construction of a signal in
\acp{CW}, the computational cost of the search by looking at the algorithm to
generate template bank. We will also talk about how we claim a detection of real
signals and sensitivity of our search.

\section{Signal Model} 
We here try to derive the model of the waveform considering both Doppler and
amplitude modulation of the signal. This will be a summary of the data analysis
paper by \citet{Jaranowski_1998}.

If L is the length of the detector and $\lambda$ the wavelength of gravitational
wave, in long wave approximation $\lambda \gg L$. The time delays of
wave propagating over the detector are negligible ($\Delta \approx $), so
gravitational wave field can be treated as being uniform all over the detector.
Let h be the relative change of arm length of detector when gravitational waves
passes by.
\begin{equation}
h(t) = \frac{1}{2} n_1.[\tilde{H}(t)n_1] - \frac{1}{2} n_2. [\tilde{H}(t)n_2]
\end{equation}
where $n_1=(1,0,0)$ and $n_2=(0,1,0)$ denotes a unit vectors parallel to the
detector arm, assuming $n_1 \perp n_2$. And $\tilde{H}$ is defined as:
\begin{equation}
\tilde{H}(t) = M(t)H(t)M(t)^T,
\end{equation}
\begin{equation*}
H(t)=
\begin{pmatrix}
h_+(t) & h_\times(t) 0\\
h_\times(t) & -h_+(t) & 0\\
0 & 0 & 0
\end{pmatrix}
\end{equation*}
where, $h_+$ and $h_\times$ are plus and cross polarization of \acp{GW}.
\begin{equation}
M= M_3 M_2 M_1
\end{equation}
where,
$M_1 \rightarrow \text{transformation matrix from wave to celestial sphere with
Euler angle($\alpha,\delta,\Psi$)}$\\
$M_2 \rightarrow \text{celestial coordinates to cardinal coordinates with Euler
angle}$\\
$M_3 \rightarrow \text{from cardinal to detector reference frame}$

\begin{equation}
M_1=
\begin{pmatrix}
\cos{-\Psi} & \sin{-\Psi}& 0\\
-\sin{-\Psi} & \cos{\Psi}& 0\\
0 & 0 & 1
\end{pmatrix}
\begin{pmatrix}
1 & 0 & 0\\
0 & \cos{(-\frac{\pi}{2}-\delta)} & \sin{(-\frac{\pi}{2}-\delta)}\\
0 & -\sin{(-\frac{\pi}{2}-\delta)} & cos{(-\frac{\pi}{2}-\delta)}
\end{pmatrix}
\begin{pmatrix}
\cos{(\frac{\pi}{2}-\alpha)} & \sin{(\frac{\pi}{2}-\alpha)} & 0\\
-\sin{(\frac{\pi}{2}-\alpha)} & \cos{(\frac{\pi}{2}-\alpha)} & 0\\
0 & 0 & 1 
\end{pmatrix}
\end{equation}
\begin{equation}
M_2=
\begin{pmatrix}
\sin{\lambda}\cos{(\phi_r+\Omega_rt)} & \sin{\lambda}\sin{(\phi_r+\Omega_rt)} 
& -cos{\lambda}\\
-sin{()\phi_r+\Omega_rt)} & \cos{(\phi_r+\Omega_rt)} & 0\\
\cos{\lambda}\cos{(\phi_r+\Omega_rt)}  & \cos{\lambda}\sin{(\phi_r+\Omega_rt)} 
& sin{\lambda}
\end{pmatrix}
\end{equation}

\begin{equation}
M_3=
\begin{pmatrix}
\cos{(\frac{\pi}{2}+\gamma)} & \sin{(\frac{\pi}{2}+\gamma)} & 0\\
-\sin{(\frac{\pi}{2}+\gamma)} & \cos{(\frac{\pi}{2}+\gamma)} & 0\\
0 & 0 & 1
\end{pmatrix}
\end{equation}

\begin{figure}[h!]
  \centering
  \begin{minipage}{0.4\textwidth}
    \includegraphics[width=\textwidth]{figure/wavedetector.jpg}
  \end{minipage}
  \hfill
  \begin{minipage}{0.6\textwidth}
    \includegraphics[width=\textwidth]{figure/wavecelestial.jpg}
  \end{minipage}
  \caption{Coordinate transformation from wave frame to detector frame. Image
   from~\cite{1996A&A...312..675B}}
  \label{fig:Wavetodetector}
\end{figure}

The response of the interferometer is given by:
\begin{equation}
h(t) = F_+(t) h_+(t) + F_\times(t) h_\times(t) 
\end{equation}

And $F_+$ and $F_\times$ are the antennae pattern. The antennae pattern are periodic
functions of time, and its period is one day. The antennae pattern are the
functions of right ascension($\alpha$), declination($\delta$) of the \acp{GW}
sources and polarization angle($\Psi$). 

The antennae pattern $F_+$ and $F_\times$ can be written as:
\begin{align*}
F_+(t) = a(t)cos(2\Psi) + b(t) sin(2\Psi)\\
F_\times(t) = b(t)cos(2\Psi) - a(t) sin(2\Psi)
\end{align*}
where,
\begin{align*}\label{a}
a(t) &=
\frac{1}{16}\sin{2\gamma}(3-\cos{2\lambda})(3-\cos{2\delta})\cos{[2(\alpha-\phi_r-\Omega_rt)]}\\
&-\frac{1}{4}\cos{2\gamma} \sin{\lambda} (3-\cos{2\delta})
\sin{[2(\alpha-\phi_r-\Omega_rt)]}
+\frac{1}{4}\sin{2\gamma} \sin{2\lambda} \sin{2\delta}
\cos{[\alpha-\phi_r-\Omega_r t]} \\
&-\frac{1}{2}\cos{2\gamma} \cos{\lambda} \sin{2\delta}
\sin{[\alpha-\phi_r-\Omega_r t]}
+\frac{3}{4} \sin{2\gamma} \cos^2{\lambda} \cos^2{\delta}
\end{align*}
\begin{align*}
b(t)& =\cos{2\gamma} sin{\lambda} \sin{\delta} \cos{[2(\alpha-\phi_r-\Omega_r
t)]} +
\frac{1}{4} \sin{2\gamma} (3-\cos{2\lambda}) \sin{\delta}
\sin{[\alpha-\phi_r-\Omega_rt]}\\
&+ \cos{2\gamma} \cos{\lambda} \cos{\delta} \cos{[\alpha-\phi_r-\Omega_r t]} +
\frac{1}{2}\sin{2\gamma} \sin{2\lambda} \cos{\delta}
\sin{[\alpha-\phi_r-\Omega_rt]}
\end{align*}
where,\\
$\lambda \rightarrow \text{latitude of detector}$\\
$\Omega_r \rightarrow \text{rotational angular velocity of earth}$\\
$\phi_r \rightarrow \text{phase which defines the position of the Earth in its
diurnal motion}$\\
$\phi_r + \Omega_r t \rightarrow$ local sidereal time of the detector[the
angle between local meridian and the vernal point].\\
$\gamma \rightarrow \text{orientation of the detector's arms with respect to
geographical direction.}$\\
This is the transformation using the rotation matrix from wave frame to the
detector frame. 

The gravitational waves signal can be written in terms of Doppler
parameter($\alpha, \delta, f, f_n $) and amplitude parameter($h_0, \iota, \psi,
\phi$):
\begin{equation}
h(t) = \sum_{i=1}^{4}A_{1i} h{1i}(t) + \sum_{i=1}^{4} A_{2i} h{2i}(t)
\end{equation}
The amplitudes $A_{1i}$ and $A_{2i}$ are given by:
\begin{align*}
A_{11} = h_0 \sin{2\theta}\left[\frac{1}{8} \sin{2\iota}\cos{2\psi}\cos{\phi_0}-
\frac{1}{4}\sin{\iota} \sin{2\psi}\sin{\psi_0}\right], \\
A_{12} = h_0 \sin{2\theta}\left[\frac{1}{4} \sin{iota}\cos{2\psi}\sin{\phi_0}+
\frac{1}{4}\sin{2\iota} \sin{2\psi}\cos{\psi_0}\right], \\
A_{13} = h_0 \sin{2\theta}\left[-\frac{1}{8} \sin{2\iota}\cos{2\psi}\sin{\phi_0}
-\frac{1}{4}\sin{\iota} \sin{2\psi}\cos{\psi_0}\right], \\
A_{14} = h_0 \sin{2\theta}\left[\frac{1}{4} \sin{2\iota}\cos{2\psi}\cos{\phi_0}-
\frac{1}{8}\sin{2\iota} \sin{2\psi}\sin{\psi_0}\right], \\
A_{21} = h_0 \sin^2{\theta}\left[\frac{1}{2}
(1+\cos^2{\iota})\cos{2\psi}\cos{2\phi_0}- \cos{\iota} \sin{2\psi}\sin{2\psi_0}\right], \\
A_{22} = h_0 \sin^2{\theta}\left[\frac{1}{2}
(1+\cos^2{\iota})\sin{2\psi}\cos{2\phi_0} + \cos{\iota} \cos{2\psi}\sin{2\psi_0}\right], \\
A_{23} = h_0 \sin^2{\theta}\left[-\frac{1}{2}
(1+\cos^2{\iota})\cos{2\psi}\sin{2\phi_0}- \cos{\iota} \sin{2\psi}\cos{2\psi_0}\right], \\
A_{24} = h_0 \sin^2{\theta}\left[-\frac{1}{2}
(1+\cos^2{\iota})\sin{2\psi}\sin{2\phi_0} + \cos{\iota} \cos{2\psi}\cos{2\psi_0}\right], \\
\end{align*}
And the dependent Doppler functions $h_{i}$ are:
\begin{equation}\label{h}
h_{1}=a(t) \cos{\phi(t)},\quad h_{2}=b(t) \cos{\phi(t)}\\ 
h_{l3}=a(t) \sin{\phi(t)},\quad h_{4}=b(t) \sin{\phi(t)} 
\end{equation}
where, a(t) and b(t) are defined in \ref{a}.



\subsection{Maximum Likelihood ratio}

The data from the gravitational waves interferometer are subject to various
noise sources and the signals($h(t)$) are buried under the noises($n(t)$). The challenging part
of \ac{GW} astronomy are to extract the weak signals from the random noises,
most of the computational power are needed to filter out the noises. The
detector data $x(t)$ can be written as :
\begin{equation}
x(t) = h(t)+n(t)
\end{equation}
The maximum likelihood ratio can be defined as ratio of the probability density function
of when signal is present to when signal is absent.
\begin{equation}
\Lambda = \frac{p(x|h+n)}{p(x|n)}
\end{equation}
The log likelihood function can be written as:
\begin{equation}
\ln{\Lambda}=(x|h)-\frac{1}{2}(h|h)
\end{equation}
where the scalar product is defined as:
\begin{equation}
(x|y)=4\mathcal{R} \int_{0}^{\infty}\frac{\tilde{x}(f)\tilde{y}^*(f)}{S_h(f)}df, 
\end{equation}
where, $\tilde{x}$ is the Fourier transform, $^*$ is the complex conjugate and 
$S_h$ is the power spectral density of the detector noise.

The signal $h(t,\vec{A},\vec{\lambda})$ depends linearly on the amplitudes parameter(A)
$h_0$, $\theta$, $\psi$, $\iota$ and $\phi$. The maximum value of the likelihood
function can be found by:
\begin{equation}
\frac{\partial\ln\Lambda}{\partial A_i}= 0, \quad i=1,....,4.  
\end{equation}

\begin{equation}
\sum_{j=1}^4M_{ij}A_j =(x||h_i),
\end{equation}
where M is a $4 \times 4$ matrix given by:
\begin{equation}
M_{ij} = (h_i|h_j)
\end{equation}
where,
\begin{equation}
(h_i||h_j)=\frac{2}{T_0}\int_{-T_0/2}^{T_0/2}h_i(t)h_j(t)dt, \quad
T_0 \rightarrow \text{observational time.}
\end{equation}
From Eqn.\ref{h} $h_1$ $\&$ $h_2$ $\propto$ $\cos{\phi(t)}$\quad and $h_3$ $\&$ $h_4$
$\propto$ $\sin{\phi(t)}$, so,
\begin{equation}
(h_1||h_3)=0, \quad (h_1||h_4)=0, \quad (h_2||h_3)=0, \quad (h_2||h_4)=0,  
\end{equation}
and
\begin{equation}
\begin{split}
(h_1||h_1)=(h_3||h_3)=\frac{1}{2}A,\\
(h_2||h_2)=(h_4||h_4)=\frac{1}{2}B,\\
(h_1||h_2)=(h_3||h_4)=\frac{1}{2}C,\\
\end{split}
\end{equation}
where, $A=(a(t)||a(t))$, \quad  $B=(b(t)||b(t))$, \quad  $C=(a(t)||b(t))$ 
\begin{equation*}
M=\frac{1}{2}
\begin{pmatrix}
A & C & 0 & 0\\
C & B & 0 & 0\\
0 & 0 & A & C\\
0 & 0 & C & B
\end{pmatrix}
\end{equation*}
The $\mathcal{F}$-statistic is the maximization of likelihood function over
$\vec{A}$. 
\begin{equation}
2\mathcal{F}(\vec(\lambda))=\sum_{i,j=1}^{4}[M^{-1}]_{ij}(x||h_i)(x||h_j)
\end{equation}
where,
\begin{equation*}                                                               
M^{-1}=\frac{2}{D}                                                                   
\begin{pmatrix}                                                                 
B & -C & 0 & 0\\                                                                 
-C & A & 0 & 0\\                                                                 
0 & 0 & B & -C\\                                                                 
0 & 0 & -C & A                                                                   
\end{pmatrix}                                                                   
\end{equation*} 
where, $D=AB-C^2$
\begin{equation}
2\mathcal{F}=\frac{2}{D}\left[B(x||h_1)^2 + A(x||h_2)^2 -2C(x||h_1)(x||h_2)+
B(x||h_3)^2 + A(x||h_4)^2 -2C(x||h_3)(x||h_4)\right]
\end{equation}
Since the 2$\mathcal{F}$ is maximized over the amplitude parameter, so it is the
function of remaining Doppler parameter($\vec{\lambda}\rightarrow \alpha,\delta,f^n$). The signal
template $h(t,\vec{A},\vec{\lambda})$ is not linear function of Doppler
parameter($\vec{\lambda}$), we need to find maximum 2$\mathcal{F}$ value over
$\vec{\lambda}$ analytically. 

First, we need to calculate the probability density of $2\mathcal{F}$ when
signal is present and when signal is absent. The detection of signal is claimed
when the $2\mathcal{F}$ beats the certain threshold level with given False Alarm
rate. 

Lets suppose we know $h_i$ i.e. the sky position and spin down parameters. When
the signals are absent $(x||h_i)$ = 0. To make the calculation easier, lets
assume the observational time is the integer multiple of one sidereal day, so $C
= 0$~\cite{Jaranowski_2000}. When the signal is present:
\begin{equation}
(x||h_1)= \frac{1}{2}AA_1, \quad (x||h_2) = \frac{1}{2}B A_2\\
(x||h_3)= \frac{1}{2}AA_3, \quad (x||h_4) = \frac{1}{2}B A_4
\end{equation}
Now, we can define $\mathcal{F}$-statistic as:
\begin{equation}
\mathcal{F} = \frac{1}{2}(z_1^2+z_2^2+z_3^2+z_4^2)
\end{equation}
where,
\begin{equation}
z_1=2\sqrt{\frac{T_0}{S_hA}}(x||h_i), \quad i=1,3
\end{equation}
\begin{equation}
z_1=2\sqrt{\frac{T_0}{S_hB}}(x||h_i), \quad i=2,4
\end{equation}
In absence of signal the $2\mathcal{F}$ is chi-squared distribution with four
degrees of freedom. And, when the signals are present, the $2\mathcal{F}$ is a
non-central chi-squared distribution with four degrees of freedom and non
centrality parameter is the optimal signal to noise ratio(d) defined as:
\begin{equation}
d^2=\frac{2}{S_h(f)}\int_0^{T_0} h^2(t)dt
\end{equation}

The probability density functions when signal is absent($p_0$)and when signal is
present($p_1$) is:
\begin{equation}
\begin{split}
p_0(\mathcal{F})&=\frac{\mathcal{F}^3}{6}exp{-\mathcal{F}},\\
p_1(d,\mathcal{F})&=\frac{2\mathcal{F}^3/2}{d^3}I_3(d\sqrt{2\mathcal{F}})exp{-\mathcal{F}-\frac{1}{2}d^2},
\end{split}
\end{equation}
where $I_3$ is the modified Bessel function of first kind and order 3.
The expectation value of 2$\mathcal{F}$ is 4+$d^2$.
\begin{figure}[h!]
	\includegraphics[width=\textwidth]{figure/chi2.png}
	\caption{In absence of signal the pdf will be $\chi^2$ distribution with
four degrees of freedom. In presence of signal the pdf will be non central
$\chi^2$ distribution where the non central parameter is the signal.}
\end{figure}

\subsection{Phases of gravitational waves}
The pulsars are spinning down by the emission of particles, \ac{EM} and \ac{GW}.
The frequency evolution are given by the Taylor expansion in the form:
\begin{equation}
f(t) = f_0 + \sum_{n=1}^{N} \frac{f_n}{n!}t^n
\end{equation}
where, $f_n$ are the spin down parameter. Usually, for less than one year of
search we don't need more than two spin down parameter. 
The frequency($f_0$) of gravitational waves from the source are Doppler modulated due to the
rotation of earth, so the apparent frequency($f_\prime$) is given by:
\begin{equation} 
f^\prime = f_0 \left(1+\frac{\vec{v}.\hat{r}}{c}\right)
\end{equation}
where, v is the relative velocity of source with respect to detector, $\hat{r}$
is the unit vector from the detector to the sources and c is the speed of light.

So, the frequency of gravitational waves($f_{gw}$) in Solar System Barycenter
(SSB) frame is~\cite{Krishnan_2004}:
\begin{equation}
f_{gw}(t) = f_0 + \sum_{n=1}^{N} \frac{f_n}{n!}\left(t - t_0 + \frac{\Delta
\vec{r}(t). \vec{n}}{c}\right)^n
\end{equation}
where, $t_0$ is a detector time in the start of observation, and we are
neglecting the proper motion of neutron stars. 

\section{Short time baseline Fourier Transforms}
The pulsars are slowly spinning down, and the frequency of gravitational waves
in the interferometer will be Doppler modulated due to the rotation of earth.
As, the frequency of \ac{GW} modulated, the match filtering the one chunk of
data for the whole observational time will be inconvenient. And, the
interferometer noises are also non stationary, so the interferometer data are 
divided into N segments and Fourier transformed. The computational cost of the
search increases linearly with the number of \acp{SFT} for a fixed observational
time. There are two main points accounted for making \acp{SFT},(i) the time of the
\acp{SFT} should be short enough for the detector noises to remain constant, (ii)
long enough for the signal power to stay within the frequency bin, as the
frequency change due to Doppler modulation and spin
down~\cite{Krishnan_2004,Abbott_2007}. 

We are deriving the maximum \ac{SFT} time such that the signal stays within half
of the frequency bin as shown in \citet{Krishnan_2004}. The frequency at the detector can be given by the Doppler formula:
\begin{equation}
\label{doppler}
f_d(t) - f_s(t) = f_s(t)\frac{\vec{v}(t).\hat{n}}{c}
\end{equation}
where, $f_s$ is the frequency at the Solar System Barycenter(SSB) frame and
$\vec{v}$ is the velocity of the detector.
The rate of frequency change due to Doppler effect can calculated by taking
derivative of Eqn \ref{doppler}.
\begin{equation}
\dot{f}_= \frac{f_s}{c}\frac{d\vec{v}}{dt}.\hat{n} \leq
\frac{f_s}{c}\left|\frac{d\vec{v}}{dt}\right|
\end{equation}
The term $dv/dt=acceleration$, and it can be written as $a=v^2/R$, where R is
the radius of earth. And we can also substitute $v=2\pi R/T$ where, T is the time
period of rotation of earth.
\begin{equation}\label{sfts1}
|\dot{f}|_{max}= \frac{4\pi^2R}{T}
\end{equation}
For the signal, to stay within the half of the frequency bin, it need to satisfy
the condition $|\dot{f}|T_{SFTs} < \frac{1}/{2T_{SFTs}}$, or it can be written as:
\begin{equation}\label{sfts2}
T_{SFTs}<\sqrt{\frac{1}{2|\dot{f}_{max}|}}
\end{equation}
Substituting \ref{sfts1} in \ref{sfts2}, we get,
\begin{equation}
T_{SFTs} < 18.5 \text{hours}\sqrt{\frac{1}{f_s}}
\end{equation}
So, for a pulsar emitting a \ac{GW} of frequency 100 Hz, the time of the
\ac{SFT} should be less than 1.85 hours. For our search, we are using \acp{SFT} of 30 minutes.


\section{Template Spacing}
In a matched filtering technique the shape of the signal that will the detected
by gravitational wave interferometer needs to be well known. The shape are
determined by the sky location, frequency and the frequency derivatives of the
sources. The intrinsic strain, polarization amplitude, inclination angle, and
initial phase of a signal are not known so these parameter are maximized. In a
directed search, the shape of the known is not well know due to the lack of
information on spin frequency of the sources. And in $r$-modes \ac{GW} search
from known pulsar, the spin frequency are known but we don't know the exact
$r$-mode frequency due to the unknown compactness of neutron stars. When the
exact waveform is not known, we need a multiple templates that possibly can
match with a signal. The template spacing should consider both computational
cost and the loss in signal to noise ratio. If the distance between templates are two far away
and the signal lies in between templates we will loose the significant amount of
\ac{SNR}. And, if the templates are too dense we will wasting a computational
power.  Lets match the signal $h(t,\vec{A},\vec{\lambda})$ with a little offset template
 $h(t,\vec{A},\vec{\lambda}+\Delta\vec{\lambda})$, the match is defined
as~\cite{Owen_1996}:
\begin{equation}
M(\vec{\lambda}+\Delta\vec{\lambda}) =
max<h(t,\vec{A},\vec{\lambda})|h(t,\vec{A},\vec{\lambda}+\Delta\vec{\lambda})>
\end{equation}
This can be expanded in power series about $\Delta\lambda=0$:
\begin{equation}\label{match}
M(\lambda,\Delta\lambda)\approx
1+\frac{1}{2}\left(\frac{\partial^2M}{\partial\Delta\lambda_i\partial\Delta\lambda_j}\right)_{\Delta\lambda^k=0}
\Delta\lambda^i\Delta\lambda^j
\end{equation}
We can define a metric 
\begin{equation}
g_{ij}(\lambda)=-\frac{1}{2}\left(\frac{\partial^2M}{\partial\Delta\lambda_i\partial\Delta\lambda_j}\right)_{\Delta\lambda^k=0}
\end{equation}
Now, Eqn~\ref{match} can be written as:
\begin{equation}
1-M=g_{ij} \Delta\lambda^i\Delta\lambda^j
\end{equation}
where 1-M is the mismatch between two templates.

We define a parameter minimal match($\mu$) as the maximum loss in signal to
noise ratio when the signal lies exactly in between two nearest templates. 
We can also define $\mu$ as a mismatch between the signal with Doppler
parameter($\vec{\lambda}$) with
expectation value $2\mathcal{F}(\vec{\lambda})$ with a template
$2\mathcal{F}(\vec{\lambda}')$. 
\begin{equation}
\mu(\vec{\lambda},\vec{\lambda}')=\frac{2\mathcal{F}(\vec{\lambda})-2\mathcal{F}(\vec{\lambda}')}{2\mathcal{F}(\vec{\lambda})}
\end{equation}

The template density for a search with known spin frequency(f) and spin
down($\dot{f},\ddot{f}$)
parameter of pulsar is given by:
\begin{equation}
N=\frac{\sqrt{det|g_{ij}|}}{2(1-\mu)}\int d\dot{f}\ddot{f}
\end{equation}
where, the metric($g_{ij}$) over the parameters f, $d\dot{f}$, $\ddot{f}$ is given
by~\cite{Wette_2008}:
\begin{equation}                                                                
g_{ij}=\frac{4\pi^2T^{i+j+2}(i+1)(j+1)}{(i+2)!(j+2)!(i+j+3)}             
\end{equation}\\                                                                
                                                                                
\text{i,j=0 for $f$, 1 for $\dot{f}$ and 2 for $\ddot{f}$}; $T           
=\text{Time of observation}$                                                              
\begin{equation} 
g_{ij} = \pi^2                                                 
\begin{pmatrix}
T^2/3 & T^3/6 & T^4/20 \\                                                       
T^3/6 & 4T^4/45 & T^5/36 \\                                                     
T^4/20 & T^5/36 & T^6/112                                                       
\end{pmatrix}
\end{equation} 

So, the number of templates depends on the duration of the search, and the
number of spin down parameters. The computational cost for a directed search
with known sky location but unknown $f,\dot{f},\ddot{f}$ will be $f*T^7$. But,
the sensitivity of the search only increases by $\sqrt{T}$.

\section{Hypothesis testing}                                                    
        When we analyze a data and claim a detection we need to perform certain
test on a sample of data. If our experiment results are from purely chance then
it is defined as Null hypothesis($H_o$) and if the results rejects the null
hypothesis we define as alternative hypothesis. The alternative hypothesis are
usually the one researcher looking for.  In our search if signal is absent it is
called null hypothesis and if signals are present we call it alternative
hypothesis.
	In statistical test there are two kind of error in the experimental
results. The type I error(False alarm probability) is when the experimenter
chooses $H_a$ when $H_o$ is true.  The type II error (False Dismissal
Probability) is when the researcher chooses $H_o$ when $H_a$ is true.  So, in
Type I error the test value is greater than the threshold significance level
when no signal are present. And, in type II error the test value is less than
the threshold when there are actually signal in the data.
As shown in figure the probability of False Alarm Rate is defined as:
\begin{equation}
P_{FAR}=P(2\mathcal{F}>2\mathcal{F}^*|H_o)=\int_{2\mathcal{F}^*}^\infty
pdf(2\mathcal{F}|H_o)d2\mathcal{F}
\end{equation}
And, the equation of False Dismissal Rate is given by:
\begin{equation}
P_{FDR}=P(2\mathcal{F}<2\mathcal{F}^*|H_a)=\int_0^{2\mathcal{F}^*}
pdf(2\mathcal{F}|H_a)d2\mathcal{F}
\end{equation}



\section{Expected value of $2\mathcal{F}$}
Assuming we have Gaussian noise and no signal, the probability distribution
(pdf) of a single value of $2\mathcal{F}$ is a chi-square distribution with 4
degrees of freedom.  Let's say that out of N templates, we get a largest value
denoted as $2\mathcal{F}^*$ and the probability of this happening is
$p(\chi^2_4;2\mathcal{F}^*)$. If $2\mathcal{F}^*$ is the largest value out of N
templates, then the other N-1 templates must be less than $2\mathcal{F}^*$ and
the probability of this happening is:
\begin{equation}
p(2\mathcal{F}\leq2\mathcal{F}^*)= [cdf(\chi^2_4;2\mathcal{F}^*)]^{N-1}
\end{equation} 

To get the probability that the value $2\mathcal{F}^*$  is this loudest
$2\mathcal{F}$ value, we have to multiply by N (because there are N different
possible orderings any of your N different templates could be the loudest. So,
probability that $2\mathcal{F}^*$ is the largest value of $2\mathcal{F}$ from a
collection of N values of 2F is~\cite{Abadie_2010, Wette:2009uea}:
\begin{equation}                                                                
p(N;2\mathcal{F}^*)=
Np(\chi^2_4;2\mathcal{F}^*)[cdf(\chi^2_4;2\mathcal{F}^*)]^{N-1} 
\end{equation}

\begin{figure}[bht!]                                                            
        \includegraphics[width=\textwidth]{figure/2F_max.png}                  
        \caption{Image: Santiago Caride. Probability as a function of
                 $2\mathcal{F}$. The
                 plot shows clearly the $2\mathcal{F}$ value is a function of number of  templates.}                             
        \label{fig:2F_max}                                                  
\end{figure} 
The  $2\mathcal{F}$ at which the probability is highest are the most likely
values to be the loudest $2\mathcal{F}$. The plot ~\ref{fig:2F_max} shows the
$2\mathcal{F}$ values for which the probability is equal to $0.05$ and $0.95$.
Any $2\mathcal{F}$ below the $0.05$ mark has only a $5\%$ chance of being the
loudest; same for any $2\mathcal{F}$ above the $0.95$ mark.

These two values then give you a confidence interval we can say that, $95\%$ of
the time, the loudest  $2\mathcal{F}$ will lie between these two values. If the
loudest  $2\mathcal{F}$ from your search is not inside this interval, then it is
very unlikely to be Gaussian noise. That could mean that we found a signal, or
that we found a source of non-Gaussian noise, like an instrumental line or a
detector artifact. We would then have to do more detailed follow up on our
individual outliers to figure out which of those is the case.

\section{Upper limit}
If we didn't detect any signal($2\mathcal{F}<2\mathcal{F}^*$) then we set up an
upper limit on the intrinsic strain($h_0$) of the signal that we are trying to
detect~\cite{Romano_2017}. To calculate the upper limit, we need the
$2\mathcal{F}$ value and confidence level of our search on specific templates. 
So, for a confidence level of $95\%$, we would have detected a signal of
$2\mathcal{F}<2\mathcal{F}^*$ at least $95\%$ of the time~\cite{Romano_2017}. In
our pipeline, we calculate the value of $h_0$ such that the false dismissal rate
is 0.05 as shown in table ~\ref{table:UL}.
\begin{table*}                                                                  
\begin{center}                                                                  
\begin{tabular}{|c|c|}                                                           
\hline
\textrm{Intrinsic strain($h_0$)} & \textrm{$FDR$} \\                         
\hline
$3.84636\times10^{-25}$ & 0\\
$9.615891\times10^{-26}$ & 0.21 \\
$1.14081\times10^{-25}$ & 0.0772\\
$1.1911\times10^{-25}$ & 0.0528\\
$1.9795\times10^{-25}$ & 0.0501\\
$1.19809\times10^{-25}$ & 0.0501\\
\hline
\end{tabular}                                                                   
\end{center}                                                                    
\caption{Calculation of 95\% confidence level upper limits on the intrinsic strain ($h_o$)}                                                  
\label{table:UL}                                                        
\end{table*}                   
The upper limit on $r$-mode gravitational wave search of Crab pulsar will be
discussed later in Chapter .
%%%%%%%%%%%%%%%%%%%%%%%%%%%%%%%%%%%%%%%%%%%%%%%%%%%%%%%%%%%%%%%%%%%%%%%%%%%%%%%%%%%%%%%%%
%%%%%%%%%%%%%%%%%%%%%%%%%%%%%%%%%%%%%%%%%%%%%%%%%%%%%%%%%%%%%%%%%%%%%%%%%%%%%%%%%%%%%%%%%
%%% 							END OF FIFTH CHAPTER									  %%%
%%%%%%%%%%%%%%%%%%%%%%%%%%%%%%%%%%%%%%%%%%%%%%%%%%%%%%%%%%%%%%%%%%%%%%%%%%%%%%%%%%%%%%%%%
%%%%%%%%%%%%%%%%%%%%%%%%%%%%%%%%%%%%%%%%%%%%%%%%%%%%%%%%%%%%%%%%%%%%%%%%%%%%%%%%%%%%%%%%%

\chapter{\textbf{Tables and Figures}}
\chapter{\textbf{Conclusion}}
This thesis presents the work on $r$-mode \ac{GW} searches from a Crab pulsar
for the \ac{LIGO} \ac{O1}and \ac{O2} run. We also showed the right frequency
parameter and estimated computational cost for the $r$-mode search. In chapter 1
we started discussing about some of the historic discovery of \acp{GW} by
\ac{LIGO} and Virgo interferometer. Then in chapter 2 we derived the Einstein
field equation and showed the perturbation in flat space-time travels at speed
of light known as gravitational waves. Then, we described the polarization of
\acp{GW} and energy radiated from \ac{GW} emission.  The principal of \acp{GW}
detector is based on the Michelson interferometer. The interferometer are
affected by various noise sources, we mentioned some origin of the noises and
the steps to extract signals that are buried in noise. We gave a details on
neutron star structure and emission of \ac{GW} radiation due to mountains on
neutron stars.

In chapter 3 we talked about the $r$-mode emission from a rotating neutron star.
R-modes are damped by the viscous mechanism but are unstable to \ac{GW}
radiation. We discussed the dependence of $r$-mode
instability window to the temperature and spin frequency of neutron star. We
later explained the $r$-mode frequency deviate from Newtonian case due to the
correction from general relativity and rapid rotation. 

Chapter 4 shows the correct frequency band and frequency derivatives to search
for $r$-modes. The right parameter is derived using the first and second order
correction in $r$-modes frequencies. The correction includes but not limited to
unknown compactness of neutron star and rapid rotation correction.  The chapter
also includes the spin down limit of selected pulsar that beats the LIGO noise
curve. Crab has the best spin down limit in terms of intrinsic strain and it can
be the first pulsar to search for $r$-mode \acp{GW}. 

In chapter 5 we discussed the method of extracting continuous wave signals from
the \ac{LIGO} noisy data. We started by the interferometer response to the
\ac{GW} signal and how the continuous wave signal are the functions of Doppler
and Amplitudes parameter. Then we gave a brief introduction of maximum
likelihood functions and how the signals are maximized to the unknown amplitude
parameters. We gave details of construction of \ac{SFT} and waveforms
from the interferometer data for the continuous waves. To claim the signal
detection, the signal should be above a certain confidence level. If there is a
signal in the data, the expected $\mathcal{F}-statistic$ value will be the non
central chi squared deviation, where the non central parameter will be
proportional to the signal. 

The next chapter is about the $r$-mode \ac{GW} search from the Crab pulsar. On
chapter 4 we gave a details why Crab is our first candidate. The Crab pulsar
timing has been constantly monitored by the Jodrell Bank Observatory. We
searched for a first two observing run of a Advanced \ac{LIGO}. Crab glitched in
the middle of the second run, so we divided the second \ac{LIGO} run into two
halves. We described the process of data analysis in our search that are
explained in detail in chapter 5. We did not find any evidence of \ac{GW} so, we
set up the upper limit for the three different searches.   

With the improvement of \ac{LIGO} sensitivity and computational power in few
years, the detection of continuous gravitational wave will be more probable. The
better understanding of $r$-mode instability and saturation amplitude will also
help to know the energy of the $r$-mode gravitational wave emission. Since, the
detection of continuous \ac{GW} will help us to understand the neutron star
structure, this will benefit not only the astronomer but nuclear and particle
physicist too.
%%%%%%%%%%%%%%%%%%%%%%%%%%%%%%%%%%%%%%%%%%%%%%%%%%%%%%%%%%%%%%%%%%%%%%%%%%%%%%%%%%%%%%%%%
%%%%%%%%%%%%%%%%%%%%%%%%%%%%%%%%%%%%%%%%%%%%%%%%%%%%%%%%%%%%%%%%%%%%%%%%%%%%%%%%%%%%%%%%%
%%% 						End of Fifth Chapter 											  %%%
%%%%%%%%%%%%%%%%%%%%%%%%%%%%%%%%%%%%%%%%%%%%%%%%%%%%%%%%%%%%%%%%%%%%%%%%%%%%%%%%%%%%%%%%%
%%%%%%%%%%%%%%%%%%%%%%%%%%%%%%%%%%%%%%%%%%%%%%%%%%%%%%%%%%%%%%%%%%%%%%%%%%%%%%%%%%%%%%%%%







%%%%%%%%%%%%%%%%%%%%%%%%%%%%%%%%%%%%%%%%%%%%%%
%Backmatter -- Bibliography, appendices, etc.%
%%%%%%%%%%%%%%%%%%%%%%%%%%%%%%%%%%%%%%%%%%%%%%
\backmatter


%%%%%%%%%%%%%%%%%%%%%%%%%%%%%%%%%%%%%%%%%%%%%%%%%%%%%%%%
%Bibliography:  Use BibTeX 							   %
%%%%%%%%%%%%%%%%%%%%%%%%%%%%%%%%%%%%%%%%%%%%%%%%%%%%%%%%
\bibliographystyle{chicago}
\addcontentsline{toc}{chapter}{\textbf{References}}
\bibliography{thesis}


%%%%%%%%%%%%%%%%%%%%%%%%%%%%%%%%%%%%%%%%%%%%%
%        APPENDIX A                         %
%%%%%%%%%%%%%%%%%%%%%%%%%%%%%%%%%%%%%%%%%%%%%
%\appendix
%\label{swpn}
%\addtocontents{toc}{\noindent\hyperref[swpn]{\textbf{Appendices}}}
%\addcontentsline{toc}{chapter}{A. \textbf{Double T Signature}}
%\chapter*{ } 	%Define an empty chapter title placeholder
%\vspace*{\fill}
%\begin{center}\textbf{Appendix A}\end{center}
%\begin{center}\textbf{Double T Signature}\end{center} %make sure the appendix title matches the title in the \addcontentsline portion above!
%\vspace*{\fill}
%
%
%\renewcommand\thefigure{A.\arabic{figure}} %this is to label the figure number using the Appendix name "A" instead of Chapter 6
%
%\afterpage{
%\begin{figure}
%\centering
%\includegraphics[width=\textwidth]{TTU_DblTalt_c2C.pdf}
%\caption{Double T signature}
%\label{fig:doubleT_sig}
%\end{figure}
%\clearpage
%}
%
%%%%%%%%%%%%%%%%%%%%%%%%%%%%%%%%%%%%%%%%%%%%%%
%%        APPENDIX B                         %
%%%%%%%%%%%%%%%%%%%%%%%%%%%%%%%%%%%%%%%%%%%%%%
%\addcontentsline{toc}{chapter}{B. \textbf{Another Double T}}
%\chapter*{\textbf{ }}
%\vspace*{\fill}
%\begin{center}\textbf{Appendix B}\end{center}
%\begin{center}\textbf{Another Double T}\end{center} %make sure the appendix title matches the title in the \addcontentsline portion above!
%\vspace*{\fill}
%
%\renewcommand\thefigure{B.\arabic{figure}}
%\setcounter{figure}{0}
%\afterpage{
%\begin{figure}
%\centering
%\includegraphics[width=\textwidth]{TTU_DblT_fl4Crvs.jpg}
%\caption{Another Double T}
%\label{fig:doubleT_sig2}
%\end{figure}
%\clearpage
%}
%


\end{document}




