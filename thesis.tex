%%%%%%%%%%%%%%%%%%%%%%%%%%%%%%%%%%%%%%%%%
%This documentclass loads the packages  %
%            setspace                   %
% and                                   %
%           fancyhdr.                   %
%You may have to                        %
%get these if your TeX distribution     %
%doesn't have them.                     %
%%%%%%%%%%%%%%%%%%%%%%%%%%%%%%%%%%%%%%%%%
\documentclass{ttuthes2007}
%%%%%%%%%%%%%%%%%%%%%%%%%%%%%%%%%%%%%%%%%%%%%%
%Include any other add-on  packages you need:%
%%%%%%%%%%%%%%%%%%%%%%%%%%%%%%%%%%%%%%%%%%%%%%
%\usepackage[utf8]{inputenc}
\usepackage[T1]{fontenc}
\usepackage{lmodern}
%\usepackage[labelfont=bf,labelsep=period]{caption}
\usepackage{multirow}
\usepackage{textcomp}
%\usepackage[colorlinks=true,citecolor=black,linkcolor=black]{hyperref}
\usepackage{afterpage}
\usepackage{pdflscape}
\usepackage{hhline}
\usepackage{enumitem}
\usepackage{amsmath,graphicx,bm,mathtools,cancel}
\usepackage{acronym,setspace,float,blindtext,hyperref,caption,siunitx,tikz} 
\usepackage[square,numbers]{natbib}
\newcommand{\colvec}[2][.8]{%
  \scalebox{#1}{%
    \renewcommand{\arraystretch}{.8}%
    $\begin{Bmatrix}#2\end{Bmatrix}$%
  }
}

\newcommand\tikzmark[2]{%                                                       
\tikz[remember picture,baseline] \node[above, outer sep=0pt, inner sep=0pt]     
(#1){\phantom{#2}};%                                                            
}                                                                               
\newcommand\link[2]{%                                                           
\begin{tikzpicture}[remember picture, overlay, >=stealth, shift={(0,0)}]        
  \draw[->] (#1) to (#2);                                                       
\end{tikzpicture}%                                                              
}        


%%%%%%%%%%%%%%%%%%%%%%%%%%%%%%%%%%
%EDIT  (Running head--  REQUIRED)%
%%%%%%%%%%%%%%%%%%%%%%%%%%%%%%%%%%
\rhead{\small Texas Tech University, \textit{Binod Rajbhandari}, December 2020}	%update your name and graduation date-year here

%%%%%%%%%%%%%%%%%%%%%%%%%%%%%%%%%%%%%%%%%%%%%%%
%Uncomment if the grad school doesn't like the%
%line under the  running head:                %
%%%%%%%%%%%%%%%%%%%%%%%%%%%%%%%%%%%%%%%%%%%%%%%
\renewcommand{\headrulewidth}{0pt}


%%%%%%%%%%%%%%%%%%%%%%%%%%%%%%%%%%%%%%%%%%%%%%%%%%
%Spacing -- Do you want double or one-and-a-half?%
%%%%%%%%%%%%%%%%%%%%%%%%%%%%%%%%%%%%%%%%%%%%%%%%%%
%\doublespacing
\onehalfspacing
%%%%%%%%%%%%%%%%%%%%%%%%%%%%%%%%%%%%%%%%%%%%%%%%%%%%%%%%%%%%%
%Leave the one you want uncommented.                        %
%In places where single-line-spacing is appropriate         %
%e.g, extended quotations, you can enclose the material     %
%in a singlespacing environment (with \begin{singlespacing} %
% ...  \end{singlespacing}                                  %
%%%%%%%%%%%%%%%%%%%%%%%%%%%%%%%%%%%%%%%%%%%%%%%%%%%%%%%%%%%%%


%%%%%%%%%%%%%%%%%%%%%%%%%%%%%%%%%%%%%%%%%%%%%%%%%%
%Other preamble stuff, e.g., theorem environments%
%or newcommands go here:                         %
% e.g.                                           %
%%%%%%%%%%%%%%%%%%%%%%%%%%%%%%%%%%%%%%%%%%%%%%%%%%
% \newtheorem{theorem}{Theorem}
% \newtheorem{proposition}[theorem]{proposition}
% \newtheorem{question}{Question}
% \newtheorem{conjecture}{Conjecture}




\begin{document}
\setlength{\parindent}{10ex}
%%%%%%%%%%%%%%%%%%%%%%%%%%%%%%%%%%%%%%%%%%%%%%%%%%%%%%%%
%TITLE PAGE -- Edit the spacing commands after each \\ %
% if necessary                                         %
%%%%%%%%%%%%%%%%%%%%%%%%%%%%%%%%%%%%%%%%%%%%%%%%%%%%%%%%
\begin{titlepage}
\vbox to  \textheight{
\begin{singlespacing}
\begin{center}
First Search for Gravitational Waves from R-modes of the Crab Pulsar\\[15pt]  %Edit
by\\[15pt]
Binod Rajbhandari,\\[15pt]   %Edit to put in your name and whatever degrees you already have
A Dissertation\\[15pt]   % or Thesis
In\\[15pt]
Physics\\[15pt]
Submitted to the Graduate Faculty\\
of Texas Tech University in\\
Partial Fulfillment of\\
the Requirements for\\
the Degree of\\[15pt]
Doctor of Philosophy\\[30pt]  %Edit (or Master of YYY)
%Approved\\[15pt]
Dr. Benjamin Owen\\
Chair of Committee\\[15pt] %Edit
Dr. Joseph D. Romano\\[15pt] %Edit
Dr. Alessandra Corsi\\[15pt] %Edit (add/remove names if you have more or fewer committee members)
Dr. David Ian Jones\\[15pt] %Edit
Dr. Mark Sheridan\\ %Edit (look up the info on the Graduate School's website)
Dean of the Graduate School\\[30pt]
December 2020      %Edit
\end{center}
\end{singlespacing}
\vfill}
\end{titlepage}
%%%%%%%%%%%%%%%%%%%
%End of title page%
%%%%%%%%%%%%%%%%%%%

%%%%%%%%%%%%%%%%%%%%%%%%%%%%%%%%%%%%%%%%%%%%%%%%%%%%%%%
%Copyright page -- delete or comment out if not needed%
%usage: \copyrightpage{year of appearance}{Name}      %
%%%%%%%%%%%%%%%%%%%%%%%%%%%%%%%%%%%%%%%%%%%%%%%%%%%%%%%
\copyrightpage{2020}{Binod Rajbhandari} %Name should be same as on title page
%%%%%%%%%%%%%%%%%%%%%%%%
%\end of copyright page%
%%%%%%%%%%%%%%%%%%%%%%%%

%%%%%%%%%%%%%%%%%%%%%%%
%Start of frontmatter %
%You need this:       %
%%%%%%%%%%%%%%%%%%%%%%%
\frontmatter


%%%%%%%%%%%%%%%%%%%%%%%%%%%%%
%Acknowledgements           %
%Comment out or delete      %
%if not  wanted             %
%%%%%%%%%%%%%%%%%%%%%%%%%%%%%
\chapter{\textbf{Acknowledgements}}
I am  deeply indebted to my research advisor Professor Benjamin Owen for
introducing me to the ground breaking research on Gravitational waves. Although
I was a stranger to topics on gravitational waves, your constant support,
guidance and encouragement made it possible to understand the dark side of
the universe.  I am extremely grateful to  Prof. Joseph
Romano, Prof. Alessandra Corsi, Prof. David Ian Jones and Prof. Dimitri
Volchenkov for providing their valuable time. 

I would like to thank the former Postdocs Dr. Ra Inta and Dr. Santiago Caride for
guiding me at the beginning phase of my research career. I miss all those talks
on musical acoustics with Ra Inta, and all the Symphony Orchestra concerts I
attended with Santiago Caride. I am also grateful to Prof. Walter Borst and
Prof. Joseph Romano for teaching me about the Physics of sound and music. I had
the best experience teaching the lab for the class. 

Thanks to Eric, Giammarco and Deven for being the nicest office mates. We indeed
had a memorable conversation touching various subjects. How can I forget our
awesome Astronomy friends Manuel, Liliana, Ramiro, Arvind, Paul, Kavitha, Dario,
Matteo. Many thanks to Debra, Joyce and Melanie for all the
administrative help. I am also grateful to my Houston friends Anuj, Shradha,
Niresh, Rajesh, Uchita and Rabin Ghaju for the welcome whenever I had to escape
Lubbock. And Thanks to Sueli, Daniel, Giovanni, Biren, Sanjit for being a great companion at Texas Tech. 

Finally, I must express my very profound gratitude to my parents Balkrishna and
Kamala Rajbhandari for all the support and encouragement throughout my life.
And, I am also grateful to my siblings Bibek and Kapila Rajbhandari for their
constant support during all the difficult times.
 

%%%%%%%%%%%%%%%%%%%%%%%%%
%End of acknowledgements%
%%%%%%%%%%%%%%%%%%%%%%%%%
\newpage
\chapter{Statements}
Chapter 4 of this thesis is work that was published as Caride et al., Phys.\
Rev.\ D \textbf{100}, 064013 (2019) and Chapter 6 is soon to be submitted for
publication. The published version of Chapter 4 was mostly written by Professor
Benjamin Owen; I gave comments on the presentation, corrected many formulae, and
plotted the figure. It describes the capabilities of code that I wrote under the
tutelage of Santiago Caride and Ra Inta. I wrote most of Chapter 6, with
comments and small pieces from Professor Owen. It describes data analysis that I
performed by myself. I made a significant contribution to another paper while I
was a member of the LIGO Scientific Collaboration (LSC), "Searches for
Gravitational Waves from Known Pulsars at Two Harmonics in 2015-2017 LIGO
Data"~\cite{Abbott1_2019}.  I noticed that the Crab, the most important pulsar
in that paper, had glitched halfway through the observation; and the LSC had to
redo the analysis and rewrite the paper based on that.

%%%%%%%%%%%%%%%%%%%
%Table of Contents%
%%%%%%%%%%%%%%%%%%%
\tableofcontents	%Leave this here (table will auto-update as you write new chapters/subsections/appendix)

%%%%%%%%%%%%%%%%%%%%%%%%%%%%%%%%%%%%%%%%%%%%%%%%%%
%Abstract -- Delete or comment out if not wanted:%
%%%%%%%%%%%%%%%%%%%%%%%%%%%%%%%%%%%%%%%%%%%%%%%%%%
\chapter{\textbf{Abstract}}
Neutron stars are the most dense form of matter, the density being comparable to the
density of an atomic nucleus. If the neutron star is rotating, then, just like
Rossby waves in Earth's atmosphere, motion of this fluid will be susceptible to the
Coriolis force and associated oscillations. These non-radial oscillations, known as $r$-modes, can be very
long lived---perhaps thousands of years. Because the fluid is so dense, these
$r$-modes may be viable sources of continuous gravitational waves.  R-modes are
current quadrupoles which oscillate at four thirds the star's spin frequency. The
modes are damped by viscosity, but can be unstable to gravitational radiation via
the Chandrasekhar-Friedman-Schutz instability. When relativistic corrections are
taken into consideration, the mode frequency can be 1.39 to 1.57 times the spin
frequency of the star, and the frequency derivative can be roughly estimated in
terms of the star's measured spin-down parameter. We show for the first time how
to construct searches over appropriate ranges of frequencies and spin-down
parameters to target $r$-modes from known pulsars. 

We present the first searches for gravitational waves from $r$-modes of the Crab
pulsar, coherently and separately integrating data from three stretches of the  
first two observing runs of Advanced LIGO using the $\mathcal{F}$-statistic.    
The second run was divided in two by a glitch of the pulsar roughly halfway     
through.  The frequencies and derivatives searched were based on radio          
measurements of the pulsar's spin-down parameters as described in Caride        
\textit{et al.,} Phys.\ Rev.\ D \textbf{100}, 064013 (2019).  We did not find   
any evidence of gravitational waves. The best 95\% confidence upper limits on   
the gravitational wave intrinsic strain were $1.6\times10^{-25}$ for the first  
run, $1.5\times10^{-25}$ for the first stretch of the second run, and           
$1.2\times10^{-25}$ for the second stretch of the second run. These are the     
first upper limits on gravitational waves from $r$-modes of a known pulsar to   
beat its spin-down limit, and they do so by more than an order of magnitude in  
amplitude or two orders of magnitude in luminosity.                             
                                          
%%%%%%%%%%%%%%%%%
%End of abstract%
%%%%%%%%%%%%%%%%%


%%%%%%%%%%%%%%%%%%%%%%%%%%%%%%%%%%%%%
%List of tables and list of figures %
%Delete or comment out if not needed%
%%%%%%%%%%%%%%%%%%%%%%%%%%%%%%%%%%%%%
\listoftables	%Leave this here. List will auto-update when you build a new table

\listoffigures	%Leave this here. List will auto-update when you insert new figures
%%%%%%%%%%%%%%%%%%%%%%%%%%%%%%%%%%%%
%End of lists of tables and figures%
%%%%%%%%%%%%%%%%%%%%%%%%%%%%%%%%%%%%
\newpage
\textbf{Acronyms}
\begin{acronym}[OCSVM]
\acro{NS}{Neutron Star}                                                      
\acro{EM}{electromagnetic wave}                                              
\acro{GW}{gravitational wave}                                                
\acro{LIGO}{Laser Interferometer Gravitational wave Observatory}             
\acro{O1}{first observing run}                                               
\acro{O2}{second observing run}                                              
\acro{SFT}{short fourier transform}                                          
\acro{SNR}{signal-to-noise-ratio}  
\acro{PSD}{power spectral density}
\acro{CW}{continuous wave}
\acro{GR}{General Relativity}
\acrodef{GWOSC}{Gravitational Wave Open Science Center}
\end{acronym}



%%%%%%%%%%%%%%%%%%%%%%%%
%MAIN PART OF  DOCUMENT%
%%%%%%%%%%%%%%%%%%%%%%%%

\mainmatter

%%%%%%%%%%%%%%%%%%%%%%%%%%%%%%%%%%%%%%%%%%%%%%%%%%%%%%%%%%%%%%%%%%%%%%%%%%%%%%%%%%%%%%%%%
%%%%%%%%%%%%%%%%%%%%%%%%%%%%%%%%%%%%%%%%%%%%%%%%%%%%%%%%%%%%%%%%%%%%%%%%%%%%%%%%%%%%%%%%%
%%% 							Chapter 1											  %%%
%%%%%%%%%%%%%%%%%%%%%%%%%%%%%%%%%%%%%%%%%%%%%%%%%%%%%%%%%%%%%%%%%%%%%%%%%%%%%%%%%%%%%%%%%
%%%%%%%%%%%%%%%%%%%%%%%%%%%%%%%%%%%%%%%%%%%%%%%%%%%%%%%%%%%%%%%%%%%%%%%%%%%%%%%%%%%%%%%%%

\chapter{\textbf{Introduction}}
    On September 14 2015, the two detectors of \ac{LIGO} detected a chirp signal
7\,ms apart ~\cite{Abbott_2016}. The signal frequency went from 35\,Hz to
250\,Hz in 200\,ms with a peak strain amplitude of $10^{-21}$.  The signal
matched the waveform of two binary black holes of mass 36\,$M_\odot$ and 29\,
$M_\odot$ spiraling into each other at 410\,Mpc (1.4 billion light years) to
form a remnant of 62\,$M_\odot$. The remaining 3\,$M_\odot$ was released as
gravitational waves. The false alarm rate of this event was 1 per
203,000\,years, which makes it of astrophysical origin. After some complicated
data analysis to rule out any terrestrial origin of the signal, the event was
announced on 11 February 2016 to the world. This discovery led to the 2017 Nobel
Prize in Physics being awarded to Kip Thorne, Rai Weiss, and Barry Barish.
\begin{figure}[bht!]                                                              
        \includegraphics[width=\textwidth]{figure/BBH.png}                          
	\caption{The first detection of \ac{GW} from a binary black hole
merger by two \ac{LIGO} detectors~\cite{Abbott_2016}.}
\end{figure}      

During the second observation run of Advanced LIGO and Virgo on August 17, 2017,
the network of three detectors observed a \ac{GW} signal from a binary
neutron star merger~\cite{Abbott_2017}. The signal was first detected marginally in the Virgo
detector and in Livingston after 22\,ms and later in Hanford in another 3 ms,
limiting the sky localization to 28 square degrees. The signal frequency went
from 30\,Hz to 2048\,Hz within a minute. The total mass of the merger remnant
was $2.74\,M_\odot$, where $0.025\,M_\odot$ was radiated as \ac{GW} energy at a
luminosity distance of 40\,Mpc (130 million light years). After 1.7\,s a
burst of gamma rays was detected by the Fermi telescope from the same
location~\cite{Abbott_2017b}.
Other \ac{EM} waves all across the spectrum were discovered with delays ranging
from few a hours for optical
light to a few weeks for radio waves~\cite{Abbott_2017a}. This was first cosmic event that
was seen in both gravitational waves and electromagnetic waves, starting the era
of multi-messenger astronomy. 
\begin{figure}[bhtp!] 
        \includegraphics[width=\textwidth]{figure/GW170817.png}
	\caption{The first detection of a binary neutron star merger (GW170817) by each \ac{LIGO} detector~\cite{Abbott_2017}. The figure shows spectrograms of GW170817. The signal direction was on the blind side of the VIRGO detector. This was important to predict the sky location of GW170817.}
        \label{GW170817}                                                             
\end{figure}      

The detection of \ac{GW} opened a new way to understand the universe. This not
only supported Einstein's General Theory of Relativity, but was the first
detection of such a high mass stellar black hole. The neutron star collision showed that the
heavier elements like uranium and gold are formed through $r$-process
nucleosynthesis, where a neutron get absorbed by a heavier nucleus such as
iron before beta decay can occur~\cite{Abbott_2017c}.


%%%%%%%%%%%%%%%%%%%%%%%%%%%%%%%%%%%%%%%%%%%%%%%%%%%%%%%%%%%%%%%%%%%%%%%%%%%%%%%%%%%%%%%%%
%%%%%%%%%%%%%%%%%%%%%%%%%%%%%%%%%%%%%%%%%%%%%%%%%%%%%%%%%%%%%%%%%%%%%%%%%%%%%%%%%%%%%%%%%
%%% 							Chapter 2											  %%%
%%%%%%%%%%%%%%%%%%%%%%%%%%%%%%%%%%%%%%%%%%%%%%%%%%%%%%%%%%%%%%%%%%%%%%%%%%%%%%%%%%%%%%%%%
%%%%%%%%%%%%%%%%%%%%%%%%%%%%%%%%%%%%%%%%%%%%%%%%%%%%%%%%%%%%%%%%%%%%%%%%%%%%%%%%%%%%%%%%%

\chapter{\textbf{Gravitational Waves}}
Compact objects such as black holes or neutron stars warp space-time. When
these objects move at relativistic speed, the changing warpage in space-time
propagating at the speed of light is known as gravitational
waves~\cite{thorne1995gravitational}. The detection of \acp{GW} has created a
new era in astronomy.  \acp{GW} can travel long distances without getting
absorbed or scattering whereas \ac{EM} waves get absorbed or scattered easily. So, the
universe that can be observed through \acp{GW} is invisible to \ac{EM} waves. 

\section{Theory of Gravitational Waves}
The General Theory of Relativity describes the equation of motion of a freely
falling particle as~\cite{carroll1997lecture, 2003gieg.book.....H}:
\begin{equation}
\frac{d^2x^\lambda}{d\tau^2}+\Gamma^\lambda_{\mu\nu}\frac{dx^\mu}{d\tau}\frac{dx^\nu}{d\tau}=0,
\end{equation}
where $\mu$ and $\nu$ are the space-time coordinates, the indices go from 0--3,
$\tau$ is the proper time and  $\Gamma^\lambda_{\mu\nu}$ is the affine connection:
\begin{equation}
\Gamma^\lambda_{\mu\nu} =\frac{\partial x^\lambda}{\partial
x^\alpha}\frac{\partial^2x^\alpha}{\partial x^\mu \partial x^\nu} ,
\end{equation}
The proper time obeys
\begin{equation}
d\tau^2=dt^2 - dx^2 = -g_{\mu\nu}dx^\mu dx^\nu,
\end{equation}
where $g_{\mu\nu}$ is the metric tensor that describes space-time curvature.
\subsection{Newtonian Limit}
	For a slowly moving particle 
\begin{equation}
\frac{dx^i}{d\tau}<< \frac{dt}{d\tau},\quad (c=1)
\end{equation}
and thus
\begin{equation}
\frac{d^2x^\mu}{d\tau^2}+\Gamma^\mu_{00}\left(\frac{dt}{d\tau}\right)^2
\cong 0
\end{equation}
For a stationary gravitational field, $\Gamma^\mu_{00}$ can be written as:
\begin{equation}
\Gamma^\mu_{00}= -\frac{1}{2} g^{\mu \nu} \frac{\partial g_{00}}{\partial x^\nu}
\end{equation}	
For a weak gravitational field, we can write the metric as Minkwoski plus a small
perturbation:
\begin{equation}
g_{\mu \nu}= \eta_{\mu \nu} + h_{\mu \nu},
\end{equation}
\begin{equation*}
\eta_{\mu\nu}=
 \begin{pmatrix}
    -1 & 0 & 0 & 0 \\
    0 & 1 & 0 & 0 \\
    0 & 0 & 1 & 0 \\
    0 & 0 & 0 & 1 
 \end{pmatrix}
\end{equation*}
\begin{equation}
\Gamma^\mu_{00}= -\frac{1}{2} g^{\mu \nu} \frac{\partial h_{00}}{\partial x^\nu},
\end{equation}	
\begin{equation}
\frac{d^2x^\mu}{d\tau^2}=\frac{1}{2} \eta^{\mu \nu} \frac{\partial h_{00}}{\partial x^\nu}
\left(\frac{dt}{d\tau}\right)^2
\end{equation}
For a stationary field $\partial_0h_{00}=0$ and the $\mu=0$ component of
the last equation becomes:
\begin{equation}
\frac{d^2t}{d\tau^2}=0,
\end{equation}
This means $dt/d\tau$ is constant. And

\begin{equation}                                                                
\frac{d^2x^i}{d\tau ^2}=\frac{1}{2} \left(\frac{dt}{d\tau}\right)^2 \partial _i
h_{00}
\end{equation} 
Dividing both sides by $\left(\frac{dt}{d\tau}\right)^2$, we get,
\begin{equation}\label{GR}
\frac{d^2x^i}{dt^2}=\frac{1}{2} \partial _i h_{00},
\end{equation} 
This gives us the Newtonian case,
\begin{equation} \label{Newton}
\frac{d^2\vec{x}}{dt^2}=-\nabla \Phi
\end{equation}
Comparing \ref{GR} and \ref{Newton},we get,
\begin{equation} \label{eq:14}
h_{00}=2\Phi
\end{equation}
And, comparing with the metric $g_{\mu \nu}= \eta_{\mu \nu} + h_{\mu \nu}$,
\begin{equation}\label{eq:15}
g_{00}=-(1+2\Phi),
\end{equation}
In Newtonian theory the gravitational potential obeys Poisson's
equation:
\begin{equation} \label{eq:16}
\nabla ^ 2\Phi = 4\pi G\rho,
\end{equation}
where $\rho$ is a
mass tensor. In Newtonian mechanics mass is the only source of gravitational
field but in general relativity we need a quantity known as the stress energy
density ($T_{\mu\nu}$). And the gravitational potential can be generalized to a
metric perturbation tensor ($h_{\mu\nu}$).
In the weak field limit, the rest energy density can be written as:
\begin{equation}\label{eq:17}
T_{00}=\rho,
\end{equation}
From \ref{eq:14}, \ref{eq:16} and \ref{eq:17}
\begin{equation} \label{eq:18}
\nabla ^2h_{00}=-8\pi GT_{00},
\end{equation}
For a vacuum the Ricci tensor $R_{\mu \nu} = 0$. But the equation with matter will be:
\begin{equation} \label{eq:19}
R_{\mu \nu} = 8 \pi GT_{\mu \nu},
\end{equation}
where
\begin{equation} \label{eq:22}
R_{\mu \nu} = \frac{\partial \Gamma ^\gamma _{\mu \nu}}{\partial x^\gamma} -
\frac{\partial \Gamma ^\gamma _{\mu \gamma}}{\partial x^\nu}+ \Gamma ^\gamma
_{\mu \nu}\Gamma ^\delta _{\gamma \delta}-\Gamma ^\gamma _{\mu \delta}\Gamma
^\delta_{\nu \gamma},
\end{equation}
In a local inertial frame, stress-energy is conserved in curved
space-time:
\begin{equation} \label{eq:20}
\nabla ^\mu T_{\mu \nu} =0,
\end{equation}
From \ref{eq:19} and \ref{eq:20},
\begin{equation}
\nabla ^\mu R_{\mu \nu} =0,
\end{equation}
In \ac{GR}, the Einstein field equation is written as:
\begin{equation}
R_{\mu\nu} -\frac{1}{2}g_{\mu\nu}=-\kappa T_{\mu\nu},
\end{equation}	
where $R_{\mu\nu}$ is the Ricci tensor, $g_{\mu\nu}$ is the metric tensor
curvature and $T_{\mu\nu}$ is the stress energy tensor. The LHS describes the
curvature of space-time and the RHS defines the energy density.\\

\subsection{Linearized Theory of Gravity}
The last two terms in \ref{eq:22} are quadratic in $h_{\mu \nu}$ so are negligible
in the linear approximation. The perturbation of the Ricci tensor is given by:
\begin{equation} \label{eq:23}
R_{\mu \nu} = \frac{\partial\Gamma ^\gamma _{\mu \nu}}{\partial 
x^\gamma} - \frac{\partial\Gamma ^\gamma _{\mu \gamma}}{\partial x^\nu}
\end{equation}
The first order perturbations in the connection are:
\begin{equation} \label{eq:24}
\Gamma ^\gamma _{\mu \nu}= \frac{1}{2} \eta^{\gamma \delta}\left(\frac{\partial
h_{\delta \mu }}{\partial x^\nu} + \frac{\partial h_{\delta \nu }}{\partial
x^\mu} -\frac{\partial  h_{\mu \nu }}{\partial x^\delta}\right),
\end{equation}
\begin{equation} \label{eq:25}
R_{\mu \nu} = \frac{1}{2}\left[ -\left(-\frac{\partial^2}{\partial
t^2}+\nabla ^2\right)h_{\mu\nu} + \partial_\mu V_\nu + \partial_\nu V_\mu
\right],
\end{equation}
\begin{equation} \label{eq:26}
V_\mu = \partial_\gamma h^\gamma_\mu - \frac{1}{2}\partial_\mu h^\gamma_\gamma,
\end{equation}
	If we choose $V_\mu=0$, similar to the Lorentz gauge condition in
electromagnetism, the linearized Einstein equation becomes:
\begin{equation} \label{eq:27}
\left(-\frac{\partial^2}{\partial t^2}+\nabla ^2\right)h_{\mu\nu} = 0 
\end{equation}

%We can define a transverse reverse of $h_{\mu \nu} 
%\begin{equation}
%\tilde h_{\mu \nu} \approx h_{\mu \nu} -1/2 \eta_{\mu \nu}h
%\end{equation}
Now, the gauge condition becomes:
\begin{equation}\label{eq:28}
\partial _\mu h^\mu _\lambda = 0,
\end{equation}
%The Einstein field equation in vacuum becomes:
%\begin{equation}\label{eq:29}
%\left(-\frac{\partial^2}{\partial t^2}+\nabla ^2\right)h_{\mu\nu} =0,
%\end{equation}


\subsection{Gravitational wave radiation}
The equation \ref{eq:27} has a plane wave solution given by:
\begin{equation} \label{eq:30}
h^{\mu \nu} = A^{\mu \nu}e^{ik_\sigma x^\sigma},
\end{equation}
Here $A^{\mu \nu}$ is a tensor which has information on
the polarization and amplitudes of the gravitational waves.
%  Apart from gauge
%transformation we cam use the coordinate transformation to simplify the $A_{\mu
%\nu}$
%\begin{equation}
%x^{'\mu}= x^\mu +x_i ^\mu(x)
%\end{equation}
%\begin{equation}
%\left(-\frac{\partial^2}{\partial t^2}+\nabla ^2\right)\xi_\mu(x) =0
%\end{equation}
Since $A_{\mu \nu}$ also satisfies the wave equation, the transformation will
eliminate any four components of $A_{\mu \nu}$;
\begin{equation}
\begin{aligned}
h_{ti}=0, \\
h^\nu_\nu =0,
\end{aligned}
\end{equation}
or $A_{ti}=0$ and $A_\nu ^\nu=0$. Using the gauge condition $V_\mu =0$,
\begin{equation}
\begin{aligned}
V_t=\frac{\partial h^t _t}{\partial t}=0,\\
V_i=\frac{\partial h^j _i}{\partial x^j}=0,
\end{aligned}
\end{equation}
From the above equation,
\begin{equation}                                                                
\begin{aligned} 
A_{tt}=0, \\
k_j A_{ij}=0
\end{aligned}                                                                   
\end{equation} 
This means the gravitational waves are transverse and the vector k
determines the direction of wave propagation. The time component vanishes and
since the waves are transverse $A_{zi}=0$ assuming $\vec{k}=(0,0,\omega)$. Now,
we are left with a $2 \times 2$ symmetric matrix in the xy-plane with zero trace. With coordinate transformation and gauge conditions, the linearized
Einstein equation can be written in terms of two dimensionless amplitudes
$h_+$ and $h_\times$ and two polarization vectors $\hat e_+$ and $\hat
e_\times$. Defining

\begin{equation*}                                                               
h_{\mu\nu}=                                                                  
 \begin{pmatrix}                                                                
    0 & 0 & 0 & 0 \\                                                           
    0 & h_+ & h_\times & 0 \\                                                            
    0 & h_\times & -h_+ & 0 \\                                                            
    0 & 0 & 0 & 1                                                               
 \end{pmatrix},                                                                  
\end{equation*} 
the metric $h_{\mu\nu}$ can be written as the sum of two polarization components.   
\begin{equation}
h_{\mu\nu}=h_+\hat e_{\mu\nu}^+ +h_\times \hat e_{\mu\nu}^\times,
\end{equation}
The polarization tensors are defined as:
\begin{equation}
\begin{aligned}
\hat e^+ = \hat e_x \otimes \hat e_x - \hat e_y \otimes \hat e_y, \\
\hat e^\times = \hat e_x \otimes \hat e_y + \hat e_y \otimes \hat e_x
\end{aligned}
\end{equation}
Or we can write the metric perturbation in matrix form:
\begin{equation*}                                                               
h_{\mu\nu}=     
 \begin{pmatrix}                                                                
    0 & 0 & 0 & 0 \\                                                            
    0 & 1 & 0 & 0 \\                                                            
    0 & 0 & -1 & 0 \\                                                            
    0 & 0 & 0 & 0                                                               
 \end{pmatrix}
h_+  
+
 \begin{pmatrix}                                                                
    0 & 0 & 0 & 0 \\                                                            
    0 & 0 & 1 & 0 \\                                                            
    0 & 1 & 0 & 0 \\                                                            
    0 & 0 & 0 & 0                                                               
 \end{pmatrix}
h_\times,                                                                  
\end{equation*} 

\begin{figure}[h!]                                                              
	\includegraphics[width=\textwidth]{figure/polarization.png}
	\caption{The figure shows the effect of passing a gravitational wave
through a ring of particles. The top figure shows the plus polarization and the
bottom shows the cross polarization. Figure from ~\cite{Schutz}.}                                                     
        \label{fig:polarization}
\end{figure}

	The effect of a passing gravitational wave is to compress and stretch
alternately in the transverse directions. Let us assume a ring of particles in
the xy
plane and a \ac{GW} traveling in the  z-direction. This will cause the
ring to contract along the x-axis at the same time it expands along the y-axis. These
contractions and expansions will oscillate as  gravitational waves pass. 

\subsection{Gravitational Wave Luminosity}
\ac{GW} energy cannot be localized perfectly.
However we can say that a certain amount of energy is contained in a region of
several wavelengths extent. The stress energy of \acp{GW} is given by:
\begin{equation}
T_{\mu\nu}^{GW} = \frac{1}{32\pi}\left\langle
h_{jk,\mu}^{TT}h_{jk,\nu}^{TT}\right\rangle,
\end{equation}
where <> means averaging over several wavelengths and TT means Transverse
Traceless.
For a wave propagating in the z-direction, 
%it has three non-zero components.
\begin{equation}
T_{00}^{GW}=\frac{1}{16\pi}\left\langle\dot{h}_+^2+\dot{h}_\times
^2\right\rangle
\end{equation}
If we suppose a \ac{GW} has a frequency $f$ and amplitude $h_+ = h_\times = h$,
then $\dot{h}_+^2 = \dot{h}_\times^2 = \frac{1}{2}(2\pi fh)^2$
S, and the energy density of the \acp{GW} is:
\begin{equation}
T_{00}^{GW} = \frac{\pi}{4}\frac{c^3}{G}f^2
h^2=0.3\left(\frac{f}{1\,\mathrm{kHz}}\right )^2
\left(\frac{h}{10^{-21}}\right )^2\,\mathrm{Wm^{-2}}
\end{equation}
\\
The spin down limit for the Crab strain amplitude is $h\approx 2.5 \times
10^{-24}$ for an $r$-mode frequency around 41\,Hz. So the total energy flux
radiated by the Crab pulsar through Earth due its loss of rotational energy will be $4.5\times10^{-9}\,\mathrm{Wm^{-2}}$.

\subsection{Generation of Gravitational Waves}
	The \ac{GW} is quadrupolar in nature as the mass (monopole) and
momentum (dipole) are conserved. In Electromagnetism we can have dipole
radiation as there can be positive and negative charges.
The Einstein quadrupole formula for \acp{GW} can be written as:
\begin{equation}\label{eg:strainamplitude}
h_{\mu\nu}= \frac{2}{r}\frac{G}{c^4}\ddot{I}_{\mu\nu}^{TT},
\end{equation}
where $I_{\mu\nu}$ is the quadrupole moment.
Using the Einstein quadrupole formula the \ac{GW} luminosity can be written
as:
\begin{equation}
\frac{dE}{dt} =
-\frac{1}{5}\frac{G}{c^5}\left\langle\dddot{I}_{\mu\nu}\dddot{I}_{\mu\nu}\right\rangle
\end{equation}
The expression for power radiated by a \ac{GW} is similar to the power radiated
by an electric dipole. There the expression is written as
$\frac{dE}{dt}=\frac{\mu_o}{6\pi}\left\langle \ddot{p}^2\right\rangle$, where p
is the electric dipole moment. Like we discussed earlier, for \acp{GW} the first
non vanishing term is the quadrupole whereas for \ac{EM} it is the dipole.

\section{Indirect evidence of Gravitational Waves}
	
Gravitational radiation will remove the energy from a binary system or a
rotating neutron star. Hulse and Taylor~\cite{1975ApJ...195L..51H} found a binary system with a
neutron star and a pulsar named PSR B1913+16 whose orbital period was decaying, which
can be explained by \acl{GW} emission. Both companions have a mass of $1.4
M_\odot$
with an orbital period of 8 hrs. The rate of orbital decay
$dP/dt\approx10^{-12}$\,$\mathrm{s}/\mathrm{s}$ 
is predicted by General relativity within 0.2\%~\cite{1975ApJ...195L..51H}. The disparity between predicted
and observed rates was due to the poor measurement of distance and proper motion
of the pulsar~\cite{Weisberg_2010}. This discovery led to
the 1993 Nobel Prize in Physics for Russell Hulse and Joseph Taylor.
\begin{figure}[bht!]
	\includegraphics[width=\textwidth]{figure/pulsar.png}
	\caption{The orbital decay of binary pulsar PSR B1913+16 due to the
loss of \ac{GW} energy. The image is from~\cite{Weisberg_2010}.}
\end{figure}  

\section{Gravitational Wave Detector}
The challenge of detecting \acp{GW} is that the amplitude of a \ac{GW} is a really small
 ($\approx10^{-18}$\,m) displacement, so the noise in the detector can easily mask the signal.
\ac{GW} detectors include resonant bars, interferometers, and pulsar
timing arrays. 

\subsection{Resonant bars}
The resonant bar works on a simple concept. When the frequency of a
\ac{GW} matches the bar the detector will resonate and ring like a bell. This
was the first \ac{GW} detector built by Joseph Weber, consisting of an
aluminum bar of 2 meters, and piezo-electric sensors. But the
noise of strain amplitude ($10^{-16}$) surpasses the strong \ac{GW}
signals ($10^{-21}$) by many orders of
magnitude and the GW frequency is only in principle detectable at the resonant
bar frequency. Weber claimed detection of \acp{GW} coming from the center of
 the galaxy, which led to the construction of similar detectors by other
physicists. But none of the
other detectors saw any signals and the claim most likely was a glitch in the
instrument.
\subsection{Interferometer}
\ac{GW} interferometry works on the simple principle
developed by Michelson and Morley and consists of a laser, beam splitter,
mirrors and a detector. The laser emits a monochromatic wave towards a beam
splitter which splits the beam into two perpendicular directions. The beam gets
reflected by the mirrors, passes through the beam splitter and meets at the
photo-detector. The two waves superimpose at the detector to give constructive or
destructive interference depending on the phase difference. When \acp{GW} passes
the interferometer the relative length between two arms oscillates changing the
intensity of light at photo-detector. Actually, the light reflects many times
between mirrors before they combine, making the effective length of the
interferometer more than its physical arm length.

There are two working \ac{LIGO} detectors, one in Livingston, Louisiana and one
in Hanford, Washington, 
separated by 3000\,km. Each arm of the interferometers is 4\,km in length. The
Virgo interferometer in Italy and KAGRA (3\,km arm length) in Japan are also
operational. It takes at least three detectors to determine the sky location of 
a short-duration \ac{GW} source, its intrinsic amplitude, polarization angle and 
distance~\cite{Schutz_2011}. But for long-lived \ac{CW} signals the Doppler effect
due to Earth's orbital and sidereal motion can help to locate the source. A
network of detectors will also build up \ac{SNR} as the
network SNR is the sum of individual detector SNR~\cite{Schutz_2011}.
\begin{equation}
\rho_N^2=\sum_{k=1}^N \rho_k^2 
\end{equation}

\subsubsection{Principle of Interferometer}
This section is based on the review article ~\citet{Sathyaprakash_2009}. The
\ac{GW} interferometer is based on the principle that passing of a \ac{GW} will
change the length of the arm of the interferometer. If there is a \ac{GW}
passing then the arrival time of reflected laser light will be different than when no
\ac{GW} passes. Let's assume a \ac{GW} is travelling in z-direction with +
polarization. Then the metric can be written as:
\begin{equation}
ds^2= -c^2dt^2+[1+h_+(t-\frac{z}{c})]dx^2+[1-h_+(t-\frac{z}{c})]dy^2+dz^2
\end{equation}
Lets assume a photon is emitted at time $t_{start}$ and travels
in the x-direction of the interferometer and returns back at time $t_{return}$. So,
the photon travels in a null world line with dy = dz = 0:
\begin{equation}
\left(\frac{dx}{dt}\right)^2=\frac{c^2}{1+h_+},
\end{equation}
For a small $h_+$,
\begin{equation}
\frac{dx}{dt}=\frac{c}{1+\frac{1}{2}h_+},
\end{equation}
Solving the above equation, such that  $t_{return}$ is the total time of one
complete trip for photon when reflected from a far end of the interferometer
arm. (for detail calculations see \cite{Schutz:1985jx}),
\begin{equation}
\frac{dt_{return}}{dt_{start}}=1+\frac{1}{2}[h_+(t_{start}+2L_x/c)-h_+(t_{start})],
\end{equation}
%Similarly for the interferometer y-arm,
%\begin{equation}
%\left(\frac{dy}{dt}\right)^2=\frac{c^2}{1-h_+}
%\end{equation}
%\begin{equation}
%\frac{dt_{return}}{dt_{start}}=1-\frac{1}{2}[h_+(t_{start}+2L_y/c)-h_+(t_{start})]
%\end{equation}
%The interferometer response is given by the difference in return time between
%the x-arm and the y-arm.

%\begin{equation}
%\frac{dt_{return}}{dt}=1+\frac{1}{2}{(1-\cos{\theta})h_+(t+2L)-(1+\cos{\theta})h_+(t)
%+ 2\cos{\theta} h_+[t+L(1-\cos{\theta})]}
%\end{equation}
For a small $L$ approximation compare to the wavelength ($\lambda$) of \ac{GW}
and assuming the
direction of \ac{GW} propagation makes an angle $\theta$ with the $z$-axis of
the interferometer~\cite{Sathyaprakash_2009}.
\begin{equation}
\frac{dt_{return}}{dt} =1+\sin^2{\theta} L \dot{h}_+ (t), \quad c=1, 
\end{equation}
For the x-arm, it can be written as:
\begin{equation}
\left(\frac{dt_{return}}{dt}\right)_{x-arm}=1+L\hat{e}_x.\dot{h}.\hat{e}_x,
\end{equation}
Similarly for the y-arm:
\begin{equation}                                                                
\left(\frac{dt_{return}}{dt}\right)_{y-arm}=1+L\hat{e}_y.\dot{h}.\hat{e}_y,  
\end{equation}  
The difference between the arrivals time of photons from the two arms is:
\begin{equation}
\left(\frac{d\delta
t_{return}}{dt}\right)=L(\hat{e}_x\otimes\hat{e}_x-(\hat{e}_y\otimes\hat{e}_y)\dot{h},
\end{equation}
Or the path difference between two arms as measured by the central observer's:
\begin{equation}
\delta t_{return}(t)=d:h,
\end{equation}
where $d=L(\hat{e}_x\otimes\hat{e}_x-(\hat{e}_y\otimes\hat{e}_y)$ and 
$d:h=d_{lm}h_{lm}$,
\\
This can be written in terms of the change in length between two arms:
\begin{equation}
\delta L(t)=\frac{1}{2}d:h,
\end{equation}
When a gravitational wave passes through a detector, one arm of the
interferometer gets stretched (lets say x-arm) by $\delta x=\frac{1}{2}h_+L$ and
the other arm gets squeezed by $\delta y=-\frac{1}{2}h_+L$.    

\subsubsection{Interferometer antenna pattern}
\ac{GW} astronomy is done with combinations of detectors around the globe. The
\ac{GW} polarization might not be same in all detector frames, so it is more
convenient to choose a polarization tensors in the same sky plane with respect
to \ac{GW} sources. Let $\hat\epsilon_+$ and $\hat\epsilon_\times$ be two
polarization tensors in the sky frame as shown in Figure \ref{fig:polarization}. Let
$\psi$ be the rotation angle from detector frame to the source frame, also known
as the polarization angle:
\begin{equation}
\begin{aligned}
\epsilon_+ = e_+\cos{2\psi} + e_\times \sin{2\psi}, \\
\epsilon_\times = -e_+\sin{2\psi} + e_\times \cos2{\psi}
\end{aligned}
\end{equation}
\begin{figure}[h!]
	\includegraphics[width=\textwidth]{figure/antennae.jpg}
	\caption{The left figure shows the basis vectors in the sky plane with
respect to the detector frame. The right figure shows the effect of rotation of
the basis vectors of the sky plane by angle $\psi$. Image from
~\cite{Sathyaprakash_2009}.}
\end{figure}  

Let $F_+$ and $F_\times$ be the antenna pattern functions on the sky frame
defined as:
\begin{equation}
F_+=d:e_+, \quad  F_\times=d:e_\times,
\end{equation}
where $d=L(\hat{e}_x\otimes\hat{e}_x- \hat{e}_y\otimes\hat{e}_y)$ and d:e =
$d_{ij}e^{ij}$,\\ 

The maximum value of $F_+$ and $F_\times$ is 1. We can define $\theta$ \&
$\phi$ to be spherical coordinates in the detector's reference frame. 
%We can write the antenna pattern in the detector frame by setting $\psi=0$ and
%using the rotation matrix
%\begin{equation*}                                                               
%R_\beta^\alpha=                                                                  
% \begin{pmatrix}                                                                
%    \cos\phi & \sin\phi & 0 \\                                                            
%    -\cos\theta \sin\phi & \cos\theta \cos\phi & \sin\theta \\                                                            
%    \sin\theta \sin\phi & -\sin\theta \cos\phi & \cos\theta                                                              
% \end{pmatrix}                                                                  
%\end{equation*}   
%
%For a plus polarization it is useful to define $t_{\alpha\beta}= R^{-1}e_+R$
%
%\begin{equation*}                                                               
%\begin{split}
%t_{\alpha\beta} & =                                                                  
% \begin{pmatrix}                                                                
%    \cos\phi & \sin\phi & 0 \\                                                            
%    -\cos\theta \sin\phi & \cos\theta \cos\phi & \sin\theta \\                                                            
%    \sin\theta \sin\phi & -\sin\theta \cos\phi & \cos\theta                                                              
% \end{pmatrix}^T                                                                  
%\begin{pmatrix}                                                                
%    1 & 0 & 0 \\                                                            
%    0 & -1 & 0 \\                                                            
%    0 & 0 & 0                                                               
% \end{pmatrix}       
% \begin{pmatrix}                                                                
%    \cos\phi & \sin\phi & 0 \\                                                     
%    -\cos\theta \sin\phi & \cos\theta \cos\phi & \sin\theta \\                        
%    \sin\theta \sin\phi & -\sin\theta \cos\phi & \cos\theta                           
% \end{pmatrix}\\  
%            &=
%     \begin{pmatrix}                                                                 
%    \cos^2\phi-\cos^2\theta\sin^2\phi & (1+\cos^2\theta)\sin\phi\cos\phi &
%\sin\theta\cos\theta\sin\phi \\                                                                
%    (1+\cos^2\theta)\sin\phi\cos\phi & \sin^2\phi-\cos^2\theta\cos^2\phi &
%-\sin\theta\cos\theta\cos\phi \\                                                               
%    \sin\theta\cos\theta\sin\phi & -\sin\theta\cos\theta\cos\phi & -\sin^2\theta                                                                   
% \end{pmatrix}     
%\end{split}     
%\end{equation*}   
%So, the antenna pattern for the plus polarization is $F_+$
%\begin{equation}
%\begin{split}
%F_+ & ==1/2t_{\alpha\beta}S_+\\
%& = (1+\cos^2\theta)\cos2\phi
%\end{split}
%\end{equation}
%Similarly for the cross polarization:
%\begin{equation}
%F_\times=\cos\theta\sin2\phi
%\end{equation}
In the source frame, the antenna pattern is given by:
\begin{align}
F_+=\frac{1}{2}(1+\cos^2\theta)\cos2\phi\cos2\psi-\cos\theta\sin2\phi\sin2\psi,\\
F_\times=\frac{1}{2}(1+\cos^2\theta)\cos2\phi\sin2\psi+\cos\theta\sin2\phi\cos2\psi
\end{align}

\subsubsection{Noises in Interferometer}

The strain amplitudes of recently detected binary black-hole mergers are of
order $h \approx10^{-21}$. When a \ac{GW} passes by, the change in arm length of
the interferometer is $10^{-18}$\,m, thousands of times smaller than the
diameter of proton. This small change in length due to \acp{GW} can be easily
drowned out by different sources of noise. The LIGO arms consist of tubes
evacuated to one trillionth of atmospheric pressure. This is important as even a
few molecules of gas will scatter the laser beam and the moving air hitting the
mirror will change the length travelled by the laser beam, mimicking a \ac{GW}
signal.  At frequencies below 10\,Hz the seismic noise including earthquakes and
other ground vibrations can move the mirrors, disturbing the \ac{GW} signal.
LIGO uses both active and passive damping to isolate any signal from seismic
noise.  Active damping includes sensors that track ground vibration and move the
test masses in and out of phase to cancel the noise. The passive damping
suspends the 40\,kg mirror by a four pendulum stack to isolate the mirror from
noise. Some other sources of noise are thermal noise (internal vibration and
heating of mirrors due to the laser) and shot noise (uncertainty in number of
quantized photons).  Other sources of noise are narrow spectral lines that might
imitate \acp{CW}.  Lines are produced by power mains, blinking LEDs, hardware
injections and resonance modes of mirror suspensions among others. Some of the
instrumental lines are shown in Figure \ref{fig:CWnoise}.
\begin{figure}[bht!]
	\includegraphics[width=\textwidth]{figure/CWnoise.png}
	\caption{The noise curves for the Livingston, Hanford and Virgo
interferometers during the \ac{O2} run. The spectral lines include known sources
such as 60\,Hz power line, blinking LEDs, calibration lines and violin modes.
Figure from~\cite{Abbott_2019}.}
	\label{fig:CWnoise}
\end{figure}

\section{Sources of continuous gravitational waves}
Continuous gravitational wave amplitudes are weaker than those of violent black
hole mergers. But continuous waves live for a longer time, so a longer
observational time is possible. The sources of continuous waves include rotating
neutron stars. A rotating neutron star will emit gravitational waves due to a
static solid deviation from axisymmetry (mass quadrupole) or from a fluid
oscillation inside the neutron star (current quadrupole). 

\subsection{Neutron Stars}
Neutron stars are the densest material objects in the universe. The density in
the inner core is around $10^{15}$\,$\mathrm{g/cm^3}$, which is 3 times nuclear
density. It drops to few $\mathrm{g/cm^3}$ at the crust. Neutron stars are
formed in Type II supernova explosions of stars above
$8\,M_\odot$~\cite{Lattimer_2004}. When the remnant core mass exceeds the
Chandrasekhar limit $(1.44\,M_\odot)$, then the electron degeneracy pressure
cannot withstand the gravitational pressure. Neutron star masses range from
1--2\,$M_\odot$ and radii from 10--14\,km, depending on the poorly known
equation of state of superdense matter. 
	
The neutron star structure can be divided into an outer and inner crust and an
outer and inner core. The atmosphere of a neutron star varies from a few cm
depth in hot neutron stars $(T\sim3\times10^6\,\mathrm{K})$ to a few millimeters
in cold ones $(T\sim3\times10^5\,\mathrm{K})$~\cite{Haensel:2007yy}. The
radiation from this atmosphere gives valuable information on the \ac{NS} surface
temperature, chemical composition, magnetic field, mass and radius. The outer
crust goes up to a density of $\rho=10^{11}\,\mathrm{g/cm^3}$ and contains
neutron rich nuclei in an degenerate electron gas~\cite{1976ApJ...208..550P}.
The inner crust can be up to a kilometer in depth. Its density might reaches
nuclear saturation density $(\rho_o=2.8\times10^{14}\,\mathrm{g/cm^3})$ and it
consists of electrons, free neutrons and neutron-rich atomic nuclei. The outer
core might be several kilometers deep and its density can reach up to
$5\times10^{14}\,\mathrm{g/cm^3}$.  It is primarily rich in neutrons. The inner
core's density might reach 10 times the nuclear saturation density.
At this extreme matter density it is hard to predict the composition. It might
still be dominated by neutrons, or it might feature heavier baryons (hyperons)
or free quarks. 

\subsection{Pulsars}
Pulsars are rotating neutron stars that produce a narrow \ac{EM} pulse that
acts like a lighthouse beam. When the narrow beam sweeps through the line of sight of
the earth, we see the pulsation. The first pulsar was discovered by Jocelyn Bell
in 1967~\cite{1968Natur.217..709H}. Pulsars are some of the most precise clocks in the
universe. Their spin periods increase slowly due to the loss of rotational
energy through particle winds, magnetospheric interactions and radiation
including \acp{GW}.

Pulsar's magnetic moment is not aligned with its rotational axis, so in an
inertial frame
the magnetic moment rotates with the pulsar. This time varying magnetic
moment emits magnetic dipole radiation. The power radiated by
a pulsar in magnetic dipole radiation is:
\begin{equation}
P= \frac{B^2R^6\Omega^4\sin{\alpha}^2}{6c^3},
\end{equation}
where $B$ is the dipole magnetic field, $R$ is the radius, $\Omega$ is the rotational
frequency, $\alpha$ is the angle between pulsar rotational axis and magnetic axis,
and $c$ is the speed of light.  

The energy loss of a pulsar can be estimated by the slowing down of a pulsar's
frequency. The rotational energy of a pulsar is written as $E=\frac{1}{2} I
\omega^2$. A
typical value of moment of inertia ($I$) of neutron star is of order
$10^{38}$\,$\mathrm{kg m^2}$. The
time derivative of the energy gives the rotational energy loss:
\begin{align}
\frac{dE}{dt} &= I\omega\dot{\omega}\\
&= 4\pi^2\nu\dot{\nu}
\end{align}
For the Crab pulsar with $\nu=29.65\,\mathrm{Hz}$ and $\dot{\nu}=
-3.68\times10^{-10}\,\mathrm{Hz/s}$, the rotational energy loss is
$4.31\times10^{31}\mathrm{W}$. Most of the Crab's rotational energy is lost in
powering the nebula through synchrotron and inverse Compton radiation.  Only
$1\%$ is lost in the observed pulsations due to electromagnetic
radiation~\cite{B_hler_2014}.
The characteristic age of a pulsar is given by:
\begin{equation}
t= \frac{1}{n-1}\frac{\nu}{\dot{\nu}}
\end{equation} 
Here $n$ is known as the braking index, given by $n=\nu\ddot{\nu}/\dot{\nu}^2$. 
The braking index value is $n=1$ for pulsar wind, $n=3$ for magnetic dipole
radiation, n=5 for mass quadrupole \ac{GW} radiation and $n=7$ for current
quadrupole radiation.

\section{Gravitational Waves from Neutron Stars}
An individual spinning neutron star can emit quasi-monochromatic \acp{GW} by
various mechanisms. The signals strength are weak but long lived. The frequency
of continuous \acp{GW} decreases slowly due to loss of rotational energy. The
detection of continuous \acp{GW} will give valuable information on the neutron star
equation of state.

\subsection{Mountains on neutron stars}
A non-axisymmetric rotating star can produce periodic \acp{GW} due to a time-varying mass quadrupole. Let us consider an asymmetric neutron star rotating
around its $z$-axis with a moment of inertia $I_{xx}\neq I_{yy}=I_{zz}$ and
ellipticity, $\epsilon=(I_{xx}-I_{yy})/I_{zz}$.

\begin{figure}[h!]                                                            
        \includegraphics[width=\textwidth]{figure/CW.png}                 
        \caption{Continuous gravitational wave emission due to a mountain on a
neutron star. Image: Ra Inta.}
        \label{fig:CW}                                                 
\end{figure}     
The principal moment of inertia needs to be transformed from the rotating frame to 
the inertial frame as \acp{GW} live in an inertial frame~\cite{PhysRevD.20.351}. 
We will use a rotation matrix $R$ for transformation.
$I_{inertial}= R^T I R$
where, 
\begin{equation}
R=
\begin{pmatrix}
\cos\theta & \sin\theta & 0 \\
-\sin\theta & \cos\theta & 0 \\
0 & 0 & 1
\end{pmatrix},
\end{equation}
$\theta=\Omega t$ and $\Omega$ is the rotational velocity of the neutron star.
\begin{equation}
\begin{split}
I_{inertial}&=
\begin{pmatrix}
(I_{xx}-I_{yy})(\cos{2\theta} -1) & \frac{1}{2}\sin{2\theta}(I_{xx}-I_{yy}) & 0 \\
\frac{1}{2}\sin{2\theta}(I_{xx}-I_{yy}) & (I_{xx}-I_{yy})(1-\cos{2\theta)} & 0  \\
0 & 0 & I_{zz}
\end{pmatrix}\\
&=(I_{xx}-I_{yy})
\begin{pmatrix}
(\cos2\theta -1) & \frac{1}{2}\sin{2\theta} & 0 \\
\frac{1}{2}\sin{2\theta} & (I_{xx}-I_{yy})(1-\cos{2\theta)} & 0  \\
0 & 0 & \frac{1}{\epsilon}
\end{pmatrix},
\end{split}
\end{equation}
where $I_{inertial}$ is the moment of inertia of the neutron star as seen from
the inertial frame.\\
The double derivative of moment of inertia is given by:
\begin{equation} \label{2ndMI}
\ddot{I}_{\mu\nu}=-16\pi ^2f^2(I_{xx}-I_{yy})
\begin{pmatrix}
\cos{(4\pi f t)} & 2\sin{(4\pi f t)} & 0 \\
2\sin{(4\pi f t)} & -\cos{(4\pi f t)} & 0 \\
0 & 0 & 0 \\
\end{pmatrix},
\end{equation}
The gravitational wave luminosity is:
\begin{equation}\label{GWLuminosity}
L= \frac{G}{5c^5}<\dddot{I}_{\mu\nu}\dddot{I}^{\mu\nu}>
\end{equation}
Taking one more derivative of \ref{2ndMI} \& plugging it in \ref{GWLuminosity}, we get,
\begin{equation}\label{Lgw}
L_{grav}= \frac{32G}{5c^5}\Omega ^6 I_{zz}^2\epsilon ^2,
\end{equation}
where $\epsilon =\frac{I_{xx}-I_{yy}}{I_{zz}}$ is defined as the ellipticity. When
the ellipticity is zero, there will not be any gravitational wave emission as there won't be any non-axisymmetric components.

The gravitational wave frequency from a non-axisymmetric neutron star will be
twice the spin frequency since the mass deformation will repeat after half a
revolution.

The strain amplitude of \ac{GW} is given by~\ref{eg:strainamplitude}:
\begin{equation}
h_{\mu\nu}=\frac{2G}{c^4d}\ddot{I}_{\mu\nu}
\end{equation}
Now, plugging the second time derivative of the moment of inertia in the above equation
\begin{equation} \label{3rdMI}
h_{\mu\nu}=-\frac{32\pi^2f^2 \epsilon I_{zz}G}{dc^4}
\begin{pmatrix}
\cos(4\pi f t)(1+\cos^2{i}) & 2\sin{(4\pi f t)}\cos{i} & 0 \\
2\sin{(4\pi f t)}\cos{i} & -\cos{(4\pi f t)}(1+\cos ^2i) & 0 \\
0 & 0 & 0 
\end{pmatrix},
\end{equation}
where $i$ is the inclination angle between the spin axis and
the inertial observer. When $i=90^\circ$, the strain amplitude will be:
\begin{equation}
h_0=\frac{4\pi ^2Gf^2 \epsilon I_{zz}}{c^4d},
\end{equation}

\subsection{$r$-modes}
The other kind of \ac{GW} emission is the non-radial oscillation of fluid
inside a neutron star; in particular the $r$-modes. These modes are unstable to
gravitational radiation, so they may live for a long time in rapidly rotating neutron
stars. These oscillating modes carry away rotational energy from a neutron star,
slowing down its spin frequency. The $r$-modes amplitude may grow to a large value just
after the formation of a neutron star. This might explain the
slowing down of spin frequencies of neutron stars that are born spinning near the Keplerian
limit. The detection of $r$-modes will help to
understand the interior and composition of a neutron star. It may even give the
unknown spin frequency of neutron stars where the narrow beam of light is not
pointing towards earth. The theory of $r$-modes will be discussed in detail in
Chapter 3.
\begin{figure}
\centering
	\includegraphics[width=0.4\textwidth]{figure/333.png}
	\caption{$r$-mode pattern of an oscillating neutron star. In the
non-perturbed star the rings are at constant latitude. Image: Chad
Hanna/ Ben Owen.}
\end{figure}

The \ac{GW} luminosity due to $r$-mode emission can be written in geometrized
units ($c$=$G$=1) as~\cite{Owen_2010}:
\begin{equation}\label{eq:modeenergy}
\frac{dE}{dt}=\frac{4}{25}M R^3 \alpha \omega^8\tilde{J},
\end{equation}
where $M$, $R$, $\omega$, are the mass, radius, angular velocity of the neutron
star and $\alpha$ is the amplitude of $r$-modes. Here $\tilde{J}=1.635\times10^{-2}$ and
for a typical star defined as~\cite{Owen:1998xg}:
\begin{equation}
\tilde{J}=\frac{1}{MR^3}\int_0^R\rho r^6dr.
\end{equation}
The rotational kinetic energy of a neutron star can be written as:
$E=\frac{1}{2}I_{zz}\omega^2$. So, taking the time derivative of energy will give:
\begin{equation}\label{eq:rotenergy}
\frac{dE}{dt}= I_{zz}\omega\dot{\omega},
\end{equation}
Combining Eqn ~\ref{eq:modeenergy} and \ref{eq:rotenergy}.
\begin{equation}\label{eq:braking}
\dot{\omega}=\frac{4}{25 I_{zz}}M R^3 \alpha\tilde{J} \omega^7, 
\end{equation}
Let us define $4/(25 I_{zz})M R^3 \alpha\tilde{J}=K$, so $\dot{\omega}=K
\omega^7$. 
Taking the derivative of \ref{eq:braking},
\begin{equation}
\ddot{\omega}= 7K\omega^6\dot{\omega},
\end{equation}
Subsituting $K=\dot{\omega}/\omega^7$,
\begin{equation}\label{eq:modebraking}
\frac{\omega\ddot{\omega}}{\dot{\omega}^2}=7,
\end{equation}
The lhs of  ~\ref{eq:modebraking} is defined as the braking index ($n$), and the
braking index of a pulsar spinning down purely by $r$-modes is 7. 

\subsubsection{$r$-mode amplitude}
The energy of the $r$-mode is proportional to the dimensionless amplitude
parameter $\alpha$. The amplitude parameter is roughly equal to the ratio of
velocity perturbation ($\delta v$) to the rotational velocity at the
equator~\cite{Ian}. The
intrinsic strain amplitude due to a $r$-mode \ac{GW} can be written
as~\cite{Owen_2010}:
\begin{equation*}                                                               
h_0 =\sqrt{\frac{8\pi}{5}}\frac{G}{c^5}r^{-1}\omega^3 \alpha M R^3 \tilde{J}      
\end{equation*}                                                                 
\text{Inverting}                                                                
\begin{equation*}                                                               
\alpha = \sqrt{\frac{5}{8\pi}}\frac{c^5}{G} r \frac{1}{\omega^3} h_0 \frac{1}{M   
R^3 \tilde{J}},                                                                  
\end{equation*}
we obtain                                                                 
\begin{equation*}                                                               
\alpha=\sqrt{\frac{5}{8\pi}}\frac{c^5}{G}\left(\frac{1}{M                       
R^3 \tilde{J}}\right)\left(\frac{1\,\mathrm{kpc}\times 10^{-24}}{(2\pi \times
100)^3}\right)
\left(\frac{h_o}{10^{-24}}\right)\left(\frac{r}{1\,\mathrm{kpc}}\right)\left(\frac{100\,\mathrm{Hz}}{f}\right)^3
\end{equation*}                                                                 
Here we converted $\omega=2\pi f$, and $r$ is the distance of neutron star from
the gravitational wave detector. For fiudicial \text{$M=1.4\,M_\odot$,
$R=11.7$\,$\mathrm{km}$
\& $\tilde{J}=0.0164$}, we have                         
\begin{equation} 
\alpha
=0.028\left(\frac{h_o}{10^{-24}}\right)\left(\frac{r}{1\,\mathrm{kpc}}\right)\left(\frac{100\,\mathrm{Hz}}{f}\right)^3
\end{equation}           
%%%%%%%%%%%%%%%%%%%%%%%%%%%%%%%%%%%%%%%%%%%%%%%%%%%%%%%%%%%%%%%%%%%%%%%%%%%%%%%%%%%%%%%%%
%%%%%%%%%%%%%%%%%%%%%%%%%%%%%%%%%%%%%%%%%%%%%%%%%%%%%%%%%%%%%%%%%%%%%%%%%%%%%%%%%%%%%%%%%
%%% 							Chapter 2											  %%%
%%%%%%%%%%%%%%%%%%%%%%%%%%%%%%%%%%%%%%%%%%%%%%%%%%%%%%%%%%%%%%%%%%%%%%%%%%%%%%%%%%%%%%%%%
%%%%%%%%%%%%%%%%%%%%%%%%%%%%%%%%%%%%%%%%%%%%%%%%%%%%%%%%%%%%%%%%%%%%%%%%%%%%%%%%%%%%%%%%%


\chapter{\textbf{R-modes}}

\section{Introduction}
R-modes have been a subject of interest for many physicists as the oscillation
of fluids inside neutron stars will help in understanding the interiors of
neutron stars and their equation of state. In the Newtonian slow motion
approximation $r$-modes are the current quadrupoles where the dominant restoring
force is the Coriolis force. The velocity perturbation is given
by~\cite{Owen:1998xg}:
\begin{equation}
\delta{\upsilon_j}=\alpha\Omega R(r/R)^ l Y_j^{B,l,l}e^{i \omega t},
\end{equation}
where $\Omega$ is the rotational velocity of the neutron star and $\vec{Y}_{lm}^B$ is a magnetic type vector spherical harmonic given by:
\begin{equation}
\vec{Y}_{lm}^B=[l(l+1)]^{-1/2} r \vec{\nabla} \times (r \vec{\nabla}Y_{lm})
\end{equation}
It is similar to long wavelength Rossby waves in Earth's atmosphere or oceans
which are responsible for heat circulation. R-modes are monochromatic waves but
the frequency decreases slowly due to the spinning down of the neutron star. The
$l=m=2$ mode couples to gravitational radiation, so these are the dominant
modes.
\begin{figure}[h!]
  \centering
  \begin{minipage}[b]{0.5\textwidth}
    \includegraphics[width=\textwidth]{figure/rmodes.png}
  \end{minipage}
  \hfill
  \begin{minipage}[b]{0.5\textwidth}
    \includegraphics[width=\textwidth]{figure/rmodes1.png}
  \end{minipage}
\caption{Top figure: Top and side view of $l = m = 2$ $r$-mode oscillation.
Bottom figure: The four patterns flow backward with angular velocity
$-1/3\Omega$. The left path shows the motion of individual fluid elements.
Figure from ~\cite{lindblom2001neutron}.}
\end{figure}
 Viscous mechanisms damp the mode but a mechanism known as the
Chandrasekhar-Friedman-Schutz (CFS) instability
~\cite{PhysRevLett.24.611}\cite{1978ApJ...222..281F} drives the $r$-modes due to
gravitational wave emission. The CFS instability can be understood as follows,
on a non rotating star, gravitational waves release positive angular momentum
from a forward moving mode and negative angular momentum from a backward moving
mode, damping both modes. For a rotating star, backward moving modes are dragged
forward as viewed from an inertial frame emitting positive angular momentum. But
in a rotating frame, the $r$-mode has negative angular momentum, so the
gravitational wave increases the amplitude of the modes instead of damping it.
This discrepancy between a rotating and an inertial frame is known as the CFS
instability. This instability causes the amplitude of the modes to
grow~\cite{Owen_2000}. 

\section{Viscous damping}
The $r$-mode amplitude grows exponentially and spins down the pulsar abruptly
unless some non-linear mechanism stops the growth~\cite{PhysRevD.65.084039}. But millisecond
pulsars exist, and that shows $r$-modes saturate at some amplitude. So there
needs to be a damping mechanism that saturates the mode amplitude. The fluid
inside neutron stars is subject to different viscous mechanisms, damping the
$r$-modes. The viscosity depends on the temperature of the stars.

Shear viscosity has a short timescale at low temperature that can be
calculated by neutron-neutron scattering cross sections~\cite{Owen_2000}. The dissipation
of mode energy is determined by the transport of momentum and energy due to
particle scattering in the neutron star~\cite{1987ApJ...314..234C}. When
the temperature falls below the superfluid transition temperature, the neutrons
and protons pair into superfluid states and don't contribute dissipation due to
momentum transport~\cite{1987ApJ...314..234C}. So, only electron-electron
scattering will able to transport momentum in a superfluid state. The
dissipation time scales due to shear viscosity from neutron-neutron and
electron-electron scattering are~\cite{ANDERSSON_2001}:
\begin{align*}
t_{sv(nn)}&=6.6\times10^7
\,s\left(\frac{M}{1.4\,M_\odot}\right)^{-5/4}\left(\frac{R}{10\,\mathrm{km}}\right)^{23/4}\left(\frac{T}{10^9\,\mathrm{K}}\right)^2,\\
t_{sv(ee)}&=2.2\times10^7\,s\left(\frac{M}{1.4\,M_\odot}\right)^{-1}\left(\frac{R}{10\,\mathrm{km}}\right)^5\left(\frac{T}{10^9\,\mathrm{K}}\right)^2,
\end{align*}
For an $n=1$ polytrope star, $t_{sv(nn)}=6.7\times 10^7\,\mathrm{s}$ and
$t_{sv(ee)}=2.2\times 10^7\,\mathrm{s}$ . So neutron stars are more viscous in the superfluid
state than the normal fluid state, which is different from the generally known
superfluidity of terrestrial materials~\cite{1987ApJ...314..234C}.

Bulk viscosity arises from the compression and rarefaction of the fluid
disturbing beta equilibrium ($p+e^- \leftrightarrow n + \nu_e$) from a direct
Urca process~\cite{Owen_2000}. In direct Urca process thermally excited neutrons
undergo beta decay to produce protons, electrons and
antineutrinos~\cite{PhysRevLett.66.2701}.  The proton and electron undergo
inverse beta decay to release neutron and neutrino.  Neutrinos carry away energy
from the star. Bulk viscosity dominates at higher temperature
($T>10^9\,\mathrm{K}$) but for $T > 10^{12} \,\mathrm{K}$, the star becomes
opaque to neutrinos. So an $r$-mode amplitude might grow for a new born neutron
star after a few minutes and release rotational energy through gravitational
radiation. At lower temperature there won't be enough thermally excited nucleons
to undergo the Urca process and the neutron star are cooled by modfied Urca
process which are a million times slower than direct Urca
process~\cite{PhysRevLett.66.2701}.

The bulk viscosity timescale for an $n$ = 1 polytrope is~\cite{ANDERSSON_2001}:
\begin{equation}
t_{bv}=2.4 \times 10^{10}\,s
\left(\frac{M}{1.4\,M_\odot}\right)^{-1}\left(\frac{R}{10\,\mathrm{km}}\right)^5\left(\frac{T}{10^9\,\mathrm{K}}\right)^{-6}\left(\frac{\nu}{1000\,\mathrm{Hz}}\right)^2
\end{equation}
So the shear viscosity dissipation decreases
with temperature while the bulk viscosity increases with temperature.

Gravitational radiation makes the $r$-mode unstable in rotating stars while
viscosity damps the mode. The $r$-mode growth time scale needs to be fast enough not to be damped by the
viscosity. The $l=m=2$ mode has the shortest timescale and higher multipole
modes have
significantly longer ones~\cite{ANDERSSON_2001}. The timescale of
\ac{GW} radiation of the $r$-mode is~\cite{ANDERSSON_2001}:
\begin{equation}
t_{gw}=-47\,\mathrm{s}                                                       
\left(\frac{M}{1.4M_\odot}\right)^{-1}\left(\frac{R}{10\,\mathrm{km}}\right)^{-4}
\left(\frac{\nu}{1000\,\mathrm{Hz}}\right)^6,
\end{equation}
where the negative sign means the mode is unstable.

\section{$r$-mode instability window}
The total growth timescale can be written as~\cite{Owen_2000}:
\begin{equation}
\frac{1}{t}=-\frac{1}{2E}\frac{dE}{dt}=\frac{1}{t_{gw}}+\sum_v \frac{1}{t_v},
\end{equation}
where $t_v$ are the dissipative timescales.
The mode is stable when $t>0$ and unstable when $t<0$. 
The instability of a mode is defined by a critical frequency such that
$\frac{1}{t_{gw}}+\sum_v \frac{1}{t_v}= 0$.
The gravitational radiation timescale depends on the spin frequency, and the viscous
timescale depends on temperature. 

The width of the instability window increases with the spin frequency of the
neutron star~\ref{fig:instability}. In new born neutron stars the $r$-mode
amplitude grows exponentially until it reaches a saturation point or the wave
breaks into daughter modes when non-linear hydrodynamic processes come into
effect~\cite{Owen_1998}. Some neutron stars are born at the Keplerian limit, and
lose most of their rotational energy due to unknown reasons. R-modes might
explain the spinning down of some young neutron stars into their present low
frequency. The evolution of a neutron star also depends on the saturation
amplitude and the spin frequency at which the $r$-mode enters the instability
window. A neutron star with saturation amplitude $\alpha$=1 can spin down in
12.3\,years with final temperature of $12.6 \times 10^8\,\mathrm{K}$ whereas
with $\alpha=10^{-4}$ can spin-down a pulsar in $6.11 \times 10^7$\,years with
temperature of $2 \times 10^8\,\mathrm{K}$~\cite{Alford_2014}. Some authors
claim that the $r$-modes saturation amplitude is well below
unity~\cite{Gressman_2002}. The $r$-mode can split into daughter modes whose
amplitudes grow exponentially if the parent mode exceeds a certain threshold,
where the daughter mode takes energy from the parent mode \cite{Arras_2003}.
\begin{figure}[bht!]                                                            
        \includegraphics[width=\textwidth]{figure/rmodesamplitude.png}                         
	\caption{Left figure: The spin and temperature  evolution of a $1.4 M_\odot$ neutron star
for different saturation amplitude and initial spin frequency entering the
instability window. The dotted curve shows the instability window. Right figure:
The spin evolution with respect to time. Figure from ~\cite{Alford_2014}.}
        \label{fig:instability}                                                     
\end{figure}  

\section{R-modes frequency}
A neutron star's interior is composed of neutron rich fluid. In a rotating star,
the liquid surface is subject to the Coriolis force and has modes similar to
Rossby waves. The non-radial oscillation known as $r$-modes is unstable to
gravitational radiation. They might live for a long time even in the presence of
viscosity. The time dependent Lagrangian displacement vector of a mode
propagating in the azimuthal direction can be written as~\cite{ANDERSSON_2001}:
\begin{equation}\label{modesvector}
\vec{\xi}= \vec{\xi} e^{i(m\varphi+\omega_r t)},
\end{equation}
where $m$ is the azimuthal quantum number, $\omega_r$ is the frequency of modes in
the rotating frame.
The inertial frame is related to the rotating frame by the equation:
\begin{equation}\label{inertialconvt}
\left(\frac{d}{dt}\right)_i=\left(\frac{\partial}{\partial
t}\right)_r+\vec{u}\cdot\nabla=\partial_t+ \Omega
\partial_\varphi,
\end{equation}
Plugging \ref{modesvector} into \ref{inertialconvt}, we get
\begin{equation}
\omega_i=\omega_r - m\Omega
\end{equation}
The frequency of an $r$-mode in the rotating frame is:
\begin{equation}\label{rotatingframe}
\omega_r = \frac{2m\Omega}{l(l+1)},
\end{equation}

For $l=m=2$, the frequency of $r$-mode in rotating frame $\omega_r=\frac{2}{3}\Omega$, and plugging this into \ref{inertialconvt},
$\omega_i = -\frac{4}{3}\Omega$.
The negative sign is the discrepancy of mode as seen in the rotating and
inertial frames and this discrepancy causes the mode's amplitude to grow instead
of damping from gravitational radiation. 

\subsection{Factors affecting $r$-mode frequency}
The $r$-mode frequency will deviate from the Newtonian case when corrections due
general relativity, rotation, core-crust coupling and other factors are considered. 

\subsubsection{General Relativity}
The relativistic $r$-mode frequency is dependent on the compactness of the star which is the ratio of mass to radius ($M/R$). The compactness can be parametrized as:
%\begin{equation}\label{compactness}
%\frac{M}{R}=\left(\frac{M}{1.4M_\odot}\right)\left(\frac{10km}{R}\right)\left(\frac{1.4M_\odot}{10^4m}\right)
%\end{equation} 
%One solar mass in geometrized units(c=G=1) is equal to 1474. So, plugging the
%numbers in \ref{compactness} gives:
\begin{equation}
\frac{M}{R}=0.21\left(\frac{M}{1.4M_\odot}\right)\left(\frac{10\,\mathrm{km}}{R}\right),
\end{equation} 
%The maximum compactness is given by~\cite{LATTIMER_2007}:
%\begin{equation}
%R \geq 2.83GM/c^2
%\end{equation}
%In geometrized units, the above equation will be $M/R \leq 0.35$
%\citet{LATTIMER_2007} gives the probable minimum mass of neutron
%star($1M_\odot$) will have radius of 14.5 km, making $M/R=0.103$.
% They have used the lowest mass as $1.02M_\odot$ and maximum mass of
%neutron star as $2.76M_\odot$. 

\citet{Idrisy_2015} have derived a conservative limit of compactness $0.11 \leq
M/R \leq 0.31$. They have used the lowest mass as 1.02\,$M_\odot$ and maximum
mass of a neutron star as 2.76\,$M_\odot$. The frequency of $r$-modes in relation to
compactness($M/R$) and spin frequency($\nu$) is~\cite{Idrisy_2015}: 

\begin{equation}
f=(-1.379 + 0.079(M/R) - 2.25(M/R)^2)\nu,
\end{equation}  
Plugging in the range of compactness (0.11--0.31), the $r$-mode frequency,
$f=(1.39-1.57)\nu$.
In other words, the effect of general relativity will increase the $r$-mode frequency in an inertial frame by a few percent.

\subsubsection{Rotation}
The second order rotational correction is important for rotating neutron stars for finding the range of $r$-mode frequency.
For a  Newtonian star (Newtonian gravity without \ac{GR} corrections), the second order formula for the $r$-mode frequency is given by 
\cite{Lindblom_1999}
\begin{equation}
f/\nu=\kappa_0 - 2 + \kappa_2 \Omega^2/\pi G \rho_0,
\end{equation}
where $\kappa_0=\omega_r/\Omega$, $\kappa_2$ is a dimensionless parameter,
and $\rho$ is the average mass density of the non-rotating star.
As $\kappa_0=2/3$, the above equation is:
\begin{equation}
f/\nu = -4/3 + \kappa_2 \Omega^2/\pi G \rho_0
\end{equation}
The expansion parameter is
\begin{equation}
\frac{\Omega^2}{\pi G \rho_0}=0.145 \left(\frac{\nu}{716\,\mathrm{Hz}}\right)^2
\left(\frac{R}{10\,\mathrm{km}}\right)^3 \left(\frac{1.4M_\odot}{M}\right)
\end{equation}
The correction
will be around 8\% for the fastest rotating neutron stars. So, the rapidly rotation correction will be less than the general relativity correction but it can be
significant for a pulsar rotating near the mass shedding limit. It can also have
a large effect on the uncertainty in $\dot{f}$ for data analysis, as we shall
see in Chapter \ref{Chapter:4}.

\subsubsection{Crust core coupling}
The solid crust is formed when a neutron star cools below
$10^{10}$\,K. The friction between the neutron fluid and the solid crust can damp
the $r$-mode. If the $r$-mode amplitude is high enough it can heat the crust-core
interface and melt the crust~\cite{Lindblom_2000}. The resulting heating can compete with the
energy
removed by the thermal
conduction and neutrino emission at the core-crust boundary. The elastic restoring
force on the crust is less than the Coriolis force, so for a rapidly rotating neutron
star the crust will oscillate much like a liquid~\cite{Levin_2001}. When the spin frequency increases, the $r$-mode
frequency will rise to meet the elastic crust modes, causing an avoided
crossing~\cite{Levin_2001}. The avoided crossing is
similar to the mode splitting in coupled pendulum by varying the length of one
pendulum. 

%%%%%%%%%%%%%%%%%%%%%%%%%%%%%%%%%%%%%%%%%%%%%%%%%%%%%%%%%%%%%%%%%%%%%%%%%%%%%%%%%%%%%%%%%
%%%%%%%%%%%%%%%%%%%%%%%%%%%%%%%%%%%%%%%%%%%%%%%%%%%%%%%%%%%%%%%%%%%%%%%%%%%%%%%%%%%%%%%%%
%%% 							END OF THIRD CHAPTER								  %%%
%%%%%%%%%%%%%%%%%%%%%%%%%%%%%%%%%%%%%%%%%%%%%%%%%%%%%%%%%%%%%%%%%%%%%%%%%%%%%%%%%%%%%%%%%
%%%%%%%%%%%%%%%%%%%%%%%%%%%%%%%%%%%%%%%%%%%%%%%%%%%%%%%%%%%%%%%%%%%%%%%%%%%%%%%%%%%%%%%%%

\chapter{\textbf{How to search for gravitational waves from $r$-modes of known
pulsars}}\label{Chapter:4}
\section{Introduction}
In terms of strain, the most sensitive searches for \acp{GW} are those for
continuous waves, signals emitted by spinning neutron stars which need not be
in binaries.
The most sensitive searches for continuous \acp{GW} are those for known
pulsars, for which a timing solution derived from \ac{EM} observations
allows coherent integration of years of data~\cite[and references
therein]{Riles:2017evm}.
Most known pulsar searches (most recently~\cite{Authors:2019ztc}) have
targeted \ac{GW} frequencies precisely double (or occasionally equal to) the
observed spin frequency of each pulsar, based on electromagnetic pulse timing
and assuming an emission model of a ``mountain''---a mass quadrupole rotating
with the star.
Some known pulsar searches (most recently~\cite{Abbott:2019bed}) have instead
targeted narrow frequency bands of a fraction of a~Hz, losing some sensitivity
and costing more than fully targeted search, but allowing for some
uncertainties in the \ac{EM} timing parameters and in the physics such as the
possibility of free precession.

Neutron stars might also emit \acp{GW} via $r$-modes, rotation-dominated
quasi-normal modes driven unstable by gravitational radiation with frequencies
roughly 4/3 the spin frequency of the star~\cite[and references
therein]{Paschalidis:2016vmz}.
There are many uncertainties in the damping mechanisms that compete with the
instability and in the amplitudes attainable by $r$-modes due to nonlinear
hydrodynamics and other saturation mechanisms.
Within those uncertainties, $r$-modes might be oscillating at relatively low
amplitudes in fast spinning young pulsars up to several thousand years after
their birth, in rapidly accreting neutron stars, and in millisecond pulsars
perhaps a long time after accretion stops~\cite[and references
therein]{Glampedakis:2017nqy}.

So far \ac{GW} searches have only set upper limits on $r$-modes in broad band
(hundreds to thousands of~Hz) searches for non-pulsing neutron stars (most
recently~\cite{Abbott:2018qee}), but searches could be done for $r$-modes from
known pulsars too.
The main issue is determining the frequency band to search.
The uncertainty in $r$-mode frequency for a known pulsar is typically a
few~Hz~\cite{Idrisy:2014qca}, broader than previous narrow band pulsar
searches but not as broad as previous $r$-mode searches.
For long integrations, which are the most sensitive searches, some thought
needs to be given to spin-down parameters as well.

The main caveat is that $r$-modes might not truly be unstable (damping might
beat driving) in all or most neutron stars once all damping mechanisms are
taken into account.
Another caveat is that $r$-modes might be unstable but saturate at amplitudes
too small to be detectable with present and near-future detectors.
Predictions of saturation amplitudes~\cite{Arras:2002dw} are indeed too small
to detect for most pulsars and present detectors~\cite{Owen_2010}.
But calculations of saturation amplitude might be wrong and \ac{GW} detectors
are improving.
Also, it takes some time to develop and refine a \ac{GW} search, so it is
worthwhile to start now.

In this article we describe how to perform searches for \acp{GW} from
$r$-modes of known pulsars using minimal adaptations of existing code.
In particular the directed search pipeline used most recently in
Ref.~\cite{Abbott:2018qee}, based on the implementation of the
$\mathcal{F}$-statistic in LALSuite~\cite{LALSuite}, is easily adaptable for
this purpose.
We present a method for choosing a search parameter space and estimate
computational costs and sensitivities using that method.
(The parameter space was partially estimated once before~\cite{Ian}.)
We show that interesting searches of data from LIGO's \ac{O1} and \ac{O2} are
feasible already, and that future searches of more sensitive data sets will be
even more interesting.
Along the way we highlight the main issues affecting realistic observations,
which leads naturally to a list of suggestions for future work both by
theorists and by data analysts.

\section{Assumptions}

\subsection{Physics}

We assume that the \ac{GW} frequency evolution $f(t)$ in the reference frame
of the solar system barycenter is
\begin{equation}
\label{ft}
f(t) = f\left( t_0 \right) + \dot f\left( t_0 \right) \left( t-t_0 \right) +
\frac{1}{2} \ddot f\left( t_0 \right) \left( t-t_0 \right)^2,
\end{equation}
where $t_0$ is some reference time (often the beginning of the observation),
dots indicate time derivatives, and we shall use a simple $f$ to indicate
$f(t_0)$ from now on.
Physically, Eq.~(\ref{ft}) assumes that the signal frequency does not change
too fast (such as from glitches or higher derivatives) or too erratically
(such as from timing noise).
Timing noise is unlikely to be an issue for the integration times (a year or
less) considered here~\cite{Ashton:2014qya}.
Glitches can be avoided by checking \ac{EM} observations.
As we shall see below, higher derivatives are not a problem, and for some
searches even the second derivative is not needed.

Throughout we consider the lowest order (current quadrupole) $r$-mode, since
it is the fastest driven by \ac{GW} emission and the least damped by most
forms of viscosity.

We shall make frequent use of the ``spin-down limits''~\cite{Owen_2010} on
intrinsic \ac{GW} strain~\cite{Jaranowski:1998qm}
\begin{equation}
h_0^\mathrm{sd} \simeq 1.6\times10^{-24} \left( \frac{\mbox{1 kpc}}{r} \right)
\left( \frac{\left|\dot f\right|} {10^{-10}\mbox{ Hz/s}} \frac{\mbox{100 Hz}}
{f} \right)^{1/2}
\label{h0sd}
\end{equation}
(where $r$ is the distance to the pulsar) and on the $r$-mode amplitude
parameter~\cite{LMoO:1998prl}
\begin{equation}
\alpha_\mathrm{sd} \simeq 0.033 \left( \frac{\mbox{100 Hz}} {f} \right)^{7/2}
\left( \frac{|\dot{f}|} {10^{-10}\mbox{ Hz\,s}^{-1}} \right)^{1/2}.
\end{equation}
These correspond to the assumption that all of the observed $\dot f$ is due to
\ac{GW} emission via $r$-modes.
That is clearly an unrealistic assumption, especially when considering the
observed values of second derivatives too; but these limits serve as useful
milestones for search sensitivity.
These numerical forms of the limits assume certain neutron star structure
parameters and also assume that the ratio of \ac{GW} frequency to spin
frequency is 4/3, so they are uncertain by a factor of two or
so~\cite{Owen_2010}.
While they should not be taken too literally, these spin-down limits give a
rough idea of which \ac{GW} searches are most interesting.

There are many debates on the growth and damping timescales of the $r$-modes,
and on the saturation amplitude (which determines the long term strength of
\ac{GW} emission).
We largely bypass them, though we note that $\alpha$ is predicted to saturate
at order $10^{-4}$ or lower~\cite{Arras:2002dw}.
Theoretical uncertainties in these quantities are great and might best be
resolved by observations which do not rely too much on theoretical guidance to
pick targets.
Attempts at relatively model independent predictions or connections to
electromagnetic observations sometimes favor or disfavor one pulsar or
another.
Alford and Schwenzer~\cite{Alford:2012yn} argue that $r$-modes in young
neutron stars spinning down shut off above 60~Hz under a wide variety of
conditions, making PSR~J0537\textminus6910 the best candidate.
They also propose that the braking index (see below) could be as low as four
in some $r$-mode dominated pulsars rather than seven as implied for
constant~$\alpha$ evolution~\cite{Owen:1998xg}.
Certainly J0537\textminus6910 has the best (lowest)
$\alpha_\mathrm{sd}$ for all young pulsars~\cite{Owen_2010} (order
$10^{-1}$), and there are hints that the braking index between its frequent
glitches might be seven~\cite{Andersson:2017fow}.
There are also arguments that millisecond pulsars will emit \acp{GW} from
$r$-modes for a long time, but only at small amplitudes even if they have high
spin-down limits~\cite{Bondarescu:2013xwa, Alford:2014pxa}.
And temperature observations of some pulsars might indicate small $r$-mode
amplitudes due to constraints on viscous heating~\cite{Schwenzer:2016tkf}.
But as observers we note that the universe holds many surprises, and we
consider searches for any pulsar with an attainable spin-down limit.

An $r$-mode emits \acp{GW} at the mode's oscillation frequency $f$ in an
inertial frame of reference.
This frequency is a function of star's spin frequency $\nu$ of the
form~\cite[e.g.]{Yoshida:2004gk}
\begin{equation}
\label{fnu}
f/\nu = A - B \left( \nu / \nu_K \right)^2 + O(\nu)^4,
\end{equation}
where we have chosen the signs so that $A>0$ and $B>0.$
(The sign of the second term is generic to retrograde modes, such as all
$r$-modes, and is due to the effects of gravitational redshift and dragging of
inertial frames on the Coriolis force~\cite{Paschalidis:2016vmz}.)
Both parameters depend on the (usually unknown) mass $M$ and radius $R$ of the
neutron star and the still uncertain equation of state, and both are
dimensionless numbers of order unity.
The rapid rotation calculation of $r$-mode frequencies in
Ref.~\cite{Yoshida:2004gk} indicates that the $O(\nu)^4$ remainder in
Eq.~(\ref{fnu}) is negligible except (for some stars) when the star is
spinning almost at its Kepler frequency $\nu_K,$ the frequency at which
centrifugal force tears it apart.
Most pulsars, including those we find most interesting for this analysis, spin
at much less than the 716~Hz of the fastest observed
frequency~\cite{Manchester:2004bp}, so we will neglect the $O(\nu)^4$
remainder in Eq.~(\ref{fnu}).
We also assume that any backbending (nonmonotonic $f$ vs.\ $\nu$) due to a
possible phase transition~\cite{Glendenning:1997fy} occurs at much higher
frequencies and does not appreciably affect Eq.~(\ref{fnu}).

In Eq.~(\ref{fnu}) we neglect many physical effects which should have only a
small effect on the ranges of $A$ and $B.$
We assume that any differential rotation has a negligible effect on the mode
frequency.
In practice this seems likely to be true, even though the $r$-mode tends to
generate a small amount of differential rotation analogous to Stokes
drift~\cite{Friedman:2017wfi}.
We assume that the magnetic field's direct effect (through restoring force) on
the $r$-mode frequency is small.
Several studies (most recently~\cite{Jasiulek:2016epr}) indicate that this is
true for the relatively low magnetic fields of pulsars spinning in the LIGO
band.
While superfluidity generally has only a tiny effect on the mode
frequencies~\cite{Lindblom:1999wi}, it does split each normal fluid $r$-mode
into two where the neutron and proton fluids are co-moving or
counter-moving~\cite{Andersson:2001bz}; and the latter mode has an additional
restoring force due to entrainment of the two fluids.
These modes in general have slightly different frequencies, but the
counter-moving modes tend to be much more damped by mutual friction.
Thus we can assume that the \ac{GW} emission is almost all through co-moving
modes, which have frequencies almost identical to normal fluid modes.

More serious is the issue of avoided crossings between $r$-modes and other
modes.
For example, Levin and Ushomirsky~\cite{Levin:2000vq} pointed out that, as a
star spins down, the $r$-mode and a torsional mode of the solid crust
(vibrating at of order 100~Hz in a nonrotating star) will swap identities.
More sophisticated calculations~\cite[e.g.]{Glampedakis:2006ap} support this
idea.
A similar issue arises with the coupling between the $r$-modes and the buoyant
force responsible for $g$-modes~\cite{Kantor:2017xuo}.
In either case, in the vicinity of the other mode's frequency the avoided
crossing introduces large errors into the simple $r$-mode frequency dependence
posited in Eq.~(\ref{fnu}).
We shall neglect this, and assume that the pulsars we consider have $\nu$
safely away from the major avoided crossings.
In a search for many pulsars, most of them are likely to satisfy this
assumption; but it is a potentially serious issue worthy of more research.

The other potentially major effect is the coherence time of the $r$-modes.
If mode-mode coupling calculations such as Ref.~\cite{Brink:2004kt} are
correct, a saturated $r$-mode exists in rough equilibrium with two ``daughter
modes'' but may occasionally undergo abrupt phase shifts~\cite{Ira}.
A similar situation could hold if one attempts to take advantage of the
relatively broad band of frequencies to search for $r$-modes from a pulsar
which does not have \ac{EM} timing contemporary with a \ac{GW} data run---the
pulsar could have glitched during the \ac{GW} run, introducing a phase error
at a random time.
Such phase errors would reduce the sensitivity of coherent data analysis
methods described below~\cite{Ashton:2017wui}.

\subsection{Data analysis}

We assume a search method based on coherent integration using a minimal
adaptation of the code used in several broad band directed searches, most
recently Ref.~\cite{Abbott:2018qee}.
This code implements the multi-interferometer
$\mathcal{F}$-statistic~\cite{Jaranowski:1998qm, Cutler:2005hc}, which
combines matched filters in such a way as to account for the daily changes of
the interferometers' beam patterns as the Earth rotates.
This particular code implements the ``demodulated'' $\mathcal{F}$-statistic
first used in Ref.~\cite{Abbott:2003yq}.
A ``resampled'' implementation of the
$\mathcal{F}$-statistic~\cite{Patel:2009qe} could speed up computations
considerably, although it would present some difficulties in dividing up the
parameter space.
The $\mathcal{F}$-statistic also quickly maximizes signal-to-noise ratio over
the (typically unknown) angles describing the orientation of the neutron
star's spin axis.
In Gaussian noise, $2\mathcal{F}$ is drawn from a $\chi^2$ distribution with
four degrees of freedom.
In the presence of a signal, the $\chi^2$ is noncentral and the power
signal-to-noise ratio (if large) is approximately $\mathcal{F}/2.$

For some pulsars a wind nebula indicates the orientation of the star's spin
axis, and in that case a similar statistic called the
$\mathcal{G}$-statistic~\cite{Jaranowski:2010rn} can achieve slightly better
sensitivity.
Since the $\mathcal{G}$-statistic is not included in the implementation of the
$\mathcal{F}$-statistic in LALSuite~\cite{LALSuite} used in
Ref.~\cite{Abbott:2018qee}, we do not consider it further here; but it is a
natural avenue of future improvement.

We assume that the \ac{GW} search can use a single sky position.
Pulsar positions are typically known to sub-arcsecond
precision~\cite{Manchester:2004bp} and the sky resolution of a directed
continuous \ac{GW} search is two orders of magnitude less precise at the
frequencies of most pulsars~\cite[e.g.]{Abbott:2018qee}.

We do not assume we have a coherent \ac{EM} pulsar timing solution throughout
the \ac{GW} observation.
Narrow band searches for pulsars such as~\cite{Abbott:2019bed} are already
broad enough to extrapolate \ac{EM} timing from old observations, and since
our search bands turn out to be broader they are even more robust.
The main worry is whether there is a glitch during the \ac{GW} observation, in
which case it will effectively cut the integration time and reduce the
signal-to-noise ratio~\cite{Ashton:2017wui}.

We do assume that, in cases where the pulsar is a component of a binary, the
binary's orbital parameters are known well enough to avoid requiring a search
over them.
This is true for most binary pulsars except those in low mass x-ray binaries.
Since those are accreting and therefore the spin can undergo random walks, we
neglect them for our present purposes.

\section{Parameter space}

Under the assumptions stated above, the parameter space of a search consists
of ranges of $f,$ $\dot f,$ and possibly $\ddot f.$
With some uncertainties, these can be calculated as functions of the
\ac{EM}-observed spin $\nu$ and spin-down parameters $\dot\nu$ and $\ddot\nu.$

\subsection{General expressions}

Assuming that $A$ and $B$ do not change appreciably with time, the time
derivatives of Eq.~(\ref{fnu}) yield
\begin{eqnarray}
\label{fdot}
\dot f / \dot \nu &=& A - 3B \left( \nu/\nu_K \right)^2,
\\
\label{fddot}
\ddot f / \ddot \nu &=& A - \left( 3 + 6/n \right) B \left( \nu/\nu_K
\right)^2,
\end{eqnarray}
where $n = \nu \ddot \nu / \dot \nu^2$ is the braking index of the pulsar.
Thus an observation of $(\nu, \dot\nu, \ddot\nu)$ and calculations of $\nu_K$
and the ranges of $(A,B)$ determine the ranges of $(f, \dot f, \ddot f)$ to be
searched---in principle.
In practice $\ddot\nu$ tends to have significant uncertainties, affecting the
choice of $\ddot f$ as discussed below.

To get the frequency range of a search we insert the ranges of $A$ and $B$
into Eq.~(\ref{fnu}) and---for now---assume that $\nu_K$ is known to obtain
\begin{eqnarray}
(f/\nu)_{\min} &=& A_{\min} - B_{\max} \left( \nu/\nu_K \right)^2,
\\
(f/\nu)_{\max} &=& A_{\max} - B_{\min} \left( \nu/\nu_K \right)^2.
\end{eqnarray}

To determine the range of $\dot{f}$ for a given $f,$ note that
Eqs.~(\ref{fnu}) and~(\ref{fdot}) can be combined to write
\begin{equation}
\dot f / \dot \nu = f / \nu - 2B \left( \nu/\nu_K \right)^2.
\end{equation}
Then we simply have the range
\begin{eqnarray}
\left( \dot{f} / \dot\nu \right)_{\min} &=& f / \nu - 2B_{\max} \left(
\nu/\nu_K \right)^2,
\\
\left( \dot{f} / \dot\nu \right)_{\max} &=& f / \nu - 2B_{\min} \left(
\nu/\nu_K \right)^2.
\end{eqnarray}

Note that Eqs.~(\ref{fnu}) and~(\ref{fdot}) combined determine $A$ and
$B/\nu_K^2$ as
\begin{eqnarray}
A &=& \left( 3f / \nu - \dot{f} / \dot{\nu} \right) /2,
\\
B/\nu_K^2 &=& \left( f / \nu - \dot{f} / \dot{\nu} \right) / \left( 2\nu^2
\right).
\end{eqnarray}
These relations can be used for parameter estimation from a \ac{GW} detection:
Once $f$ and $\dot f$ are known, we find $A$ and $B/\nu_K^2,$ which in turn
can yield information on $M$ and $R$ and the equation of
state~\cite{Yoshida:2004gk, Idrisy:2014qca}.
The equations for $A$ and $B$ also can be used to write
\begin{equation}
\ddot{f} / \ddot{\nu} = \dot{f} / \dot{\nu} - (3/n) \left( f / \nu - \dot{f} /
\dot{\nu} \right),
\end{equation}
which in principle uniquely determines $\ddot f$ in terms of $f,$ $\dot f,$
and \ac{EM}-observed quantities.

In practice, $\ddot\nu$ (or equivalently $n$) measurements are available only
for a few pulsars; and even then they may have large errors when measured over
short baselines.
For example, the monthly fits to $\ddot\nu$ for the Crab pulsar provided by
Jodrell Bank can vary by a factor of a few from the long term average and can
even change sign~\cite{Crab2015}.
It is not clear how much of this timing noise is due to magnetospheric effects
and how much is due to a genuine fluctuating torque on the star.
Since the $r$-mode frequency is determined mainly by the Coriolis force, we
are interested in the latter but not the former.
However at the moment we wish to be cautious in our choice of parameter space.
As a practical matter, current codes including that used in
Ref.~\cite{Abbott:2018qee} cut the parameter space into computing batch jobs
in a way such that the range of $\ddot f$ can depend on $f$ but not on $\dot
f.$
Plugging in the full range of $\dot f$ we can get
\begin{eqnarray}
\left( \ddot f / \ddot\nu \right)_{\min} &=& f/\nu - 2(1+3/n) B_{\max} \left(
\nu/\nu_K \right)^2,
\\
\left( \ddot f / \ddot\nu \right)_{\max} &=& f/\nu - 2(1+3/n) B_{\min} \left(
\nu/\nu_K \right)^2
\end{eqnarray}
as functions of $f.$
To err on the safe side by covering more parameter space, we can take the
minimum $\ddot f$ as zero.
Since our ``safe side'' $B_{\min}$ vanishes (see below), the maximum $\ddot f$
can be taken to be simply $\ddot\nu f/\nu,$ using the highest $\ddot\nu$
observed during the \ac{GW} integration.

At the moment these overly broad parameter ranges are not a concern, because
\ac{O1} searches are computationally cheap and even \ac{O2} searches are not
extravagant (see below).
These parameter ranges could be refined later for longer searches, when
computational cost is more of an issue, for example by calculating the range
of $B$ and exploring its consequences.

\subsection{Numerical ranges of parameters}

The range of $A$ is fairly well known.
The most recent calculation~\cite{Idrisy:2014qca} used the general
relativistic slow rotation approximation~\cite{Lockitch:2000aa,
Lockitch:2002sy} to compute $A$ for a variety of neutron-star equations of
state, obtaining 1.39 $\le A \le$ 1.57 depending almost purely on $M/R.$
Since that calculation was published, the big new constraint on the neutron
star equation of state is the lack of a large tidal effect in the binary
neutron-star merger GW170817~\cite{TheLIGOScientific:2017qsa}.
This disfavors large radii and low $A.$
But for the rest of this paper, to be conservative (cover a wide range of
parameters), we shall use the $A_{\min}$ and $A_{\max}$ quoted above.

The range of $B$ is less well known than the range of $A.$
The best general relativistic calculation~\cite{Yoshida:2004gk} drops the slow
rotation approximation, but adds the Cowling approximation (neglecting the
metric perturbation) and gives numbers only for two equations of state and two
$M/R$ values.
And the equations of state are polytropes rather than realistic equations of
state with the adiabatic index varying depending on the density.
The errors due to the Cowling approximation can be estimated as a few percent,
which is not of too much concern here; but the uncertainty from the stellar
models is more serious.

We estimate the range of $B$ from Ref.~\cite{Yoshida:2004gk} as follows:
Their Eq.~(15) gives $B\left( \nu/\nu_K \right)^2$ as 1.23--1.95 times the
ratio of kinetic to potential energy for the four stellar models considered.
For our purposes the most interesting model is their model $c,$ a polytrope of
adiabatic index~2 and $M/R=0.1$ which yields the number 1.95 (and for which
the slow rotation approximation is very accurate all the way to the Kepler
frequency).
In Fig.~2 of Ref.~\cite{Yoshida:2004gk} the sequence for model $c$ terminates
at a kinetic-to-potential energy ratio of 0.1, and this termination point
corresponds to $\nu=\nu_K,$ the ``Kepler frequency'' or maximum spin frequency
of the star.

Hence we can write $B \simeq 0.195$ for this stellar model, which should set a
safe upper limit on $B_{\max}$ for the following reasons:
The results of Ref.~\cite{Yoshida:2004gk} show that $B$ increases for smaller
$M/R$ and for lower adiabatic index (higher polytropic index).
The value $M/R=0.1$ for their model $c$ is smaller than post-GW170817 bounds,
indicating we are safe there.
An adiabatic index of~2 also errs on the safe side, since piecewise polytropic
fits to realistic equations of state~\cite{Read:2008iy} yield higher indices.
The bound on $B_{\min}$ is less clear without detailed calculations, but
$B_{\min}=0$ is well beyond the range quoted and should be safe.

Last we consider $\nu_K.$
The fastest observed spin frequency for a neutron star is about
716~Hz~\cite[and references therein]{Paschalidis:2016vmz}.
The Kepler frequency is expected to scale roughly as its Newtonian dependence
$M^{1/2} R^{-3/2},$ even in general relativity~\cite[and references
therein]{Paschalidis:2016vmz}.
Neutron star radii are roughly constant for a given equation of state, while
reliable mass measurements range over almost a factor of
two~\cite[and references therein]{Ozel:2016oaf}.
To err on the safe side (high $B_{\max}$), we assume that the 716~Hz pulsar is
on the high end of the mass range and our pulsar is on the low end so that it
has a lower Kepler frequency.
Then we can safely take $\nu_K = \mbox{716 Hz}/\sqrt{2} \simeq 506$~Hz, erring
on the safe side by assuming a possible factor of two difference in mass.

To summarize, we recommend for the moment a broad parameter space with ranges
\begin{eqnarray}
\label{range0}
f_{\min} = \nu \left( A_{\min} - B_{\max} \frac{\nu^2} {\nu_K^2} \right),
&\quad&
f_{\max} = \nu\, A_{\max},
\\
\label{range1}
\dot f_{\min} = -\dot\nu \left( \frac{f}{\nu} - 2B_{\max} \frac{\nu^2}
{\nu_K^2} \right),
&\quad&
\dot f_{\max} = -\dot\nu \frac{f}{\nu},
\\
\label{range2}
\ddot f_{\min} = 0,
&\quad&
\ddot f_{\max} = \ddot\nu \frac{f}{\nu},
\end{eqnarray}
where $\ddot\nu$ is the maximum value consistent with \ac{EM} observations,
and the other parameters are $A_{\min} = 1.39,$ $A_{\max} = 1.57,$ $B_{\max} =
0.195,$ and $\nu_K = 506$~Hz.

\section{Computational cost}

We rely heavily on the search parameter space metric~\cite{Wette:2008hg}
\begin{equation}
g_{ij} = \pi^2 \left(
\begin{array}{ccc}
T^2/3 & T^3/6 & T^4/20 \\
T^3/6 & 4T^4/45 & T^5/36 \\
T^4/20 & T^5/36 & T^6/112
\end{array}
\right)
\end{equation}
where the indices are labeled in the order $(f, \dot f, \ddot f)$ and also can
be labeled 0, 1, 2.
Here $T$ is the time from beginning to end of the integration.
This metric controls the density and placement of templates (parameter values
of matched filters) by relating coordinate distances (parameter differences)
to loss of signal-to-noise ratio~\cite{Owen_1996}.
The mismatch $g_{ij} \Delta \lambda^i \Delta \lambda^j$ between two signals
with parameters displaced by $\Delta \lambda^i$ is the fractional loss in
optimal power signal-to-noise ratio due to filtering one with the parameters
of the other.
In continuous \ac{GW} searches template banks are often constructed so that
the worst case mismatch between any signal and the nearest template is 0.2.
We calculate metric components neglecting the amplitude modulation of the
$\mathcal{F}$-statistic and including only phase terms, an approximation which
works well for integrations of many days~\cite{Prix:2006wm}.

To determine which pulsars require $\ddot f,$ we compute $g_{22} \ddot
f_{\max}^2$ to determine the maximum mismatch due to neglecting $\ddot f.$
(Note that this does not allow for the possible mitigating effect of varying
$f$ and $\dot f$ somewhat, and thus it is a conservative estimate.)
If the mismatch is comparable to or greater than 0.2, $\ddot f$ is needed.
First we evaluate this criterion using $\ddot\nu$ values taken from the ATNF
catalogue~\cite{Manchester:2004bp}.
In many cases these values are unknown or are known to be contaminated by
timing noise (such as when they are negative).
However the values of $\nu$ and $\dot\nu$ are typically well measured.
Hence we also check the need for $\ddot f$ using $\ddot\nu = 7 \dot\nu^2 /
\nu$ (implied by a braking index of 7), and if either this mismatch or the one
using the observed $\ddot\nu$ satisfies the criterion we consider a search
over $\ddot f$ to be necessary.

We consider three values of $T:$
First $1.12\times10^7$~s and $2.32\times10^7$~s, the lengths of \ac{O1} and
\ac{O2} respectively, then one year or $3.15\times10^7$~s which might be
characteristic of future LIGO runs~\cite{Aasi:2013wya}.
We only consider pulsars with $f_{\max}$ greater than 10~Hz since, even at
design sensitivity, LIGO noise increases rapidly below that frequency and the
$r$-mode amplitudes required to emit at the spin-down limit become enormous.
We find that for \ac{O1} the Crab, Vela, and several others need $\ddot f;$
while for \ac{O2} and a one-year integration most pulsars with measured
$\ddot\nu$ and several without it need $\ddot f.$

We test the need of the third frequency derivative using $g_{33} = \pi^2 T^8 /
2025$~\cite{Wette:2008hg}, the third derivative of $\nu$ from the ATNF
catalogue~\cite{Manchester:2004bp} when it is given, and the $n=7$ value of
$91 \dot\nu^3 / \nu^2$ for the third derivative when it is not given.
(The numerical factor 91 is $n(2n-1)$ in general, and can be obtained by
differentiating the definition of the braking index.)
For \ac{O1} no pulsar needs a third derivative.
For \ac{O2} no pulsar with a spin-down limit above the noise (see below) needs
a third derivative.
For a one year integration the Crab (alone of the pulsars detectable at the
spin-down limit) is on the edge of needing a third derivative.
Since this is a conservative estimate and the need can be mitigated by
slightly shortening $T$ and our focus here is on how to adapt current data
analysis codes, which do not include the third derivative, we do not address
third derivatives further here.

Under the assumption that two frequency derivatives are needed and three are
not, the proper volume of the parameter space $\sqrt{g} \int df\, d\dot f\,
d\ddot f$ integrated over the ranges in Eq.~(\ref{range0})--(\ref{range2}) is
approximately
\begin{equation}
\label{propvol}
\sqrt{g} \nu \left| \dot\nu \right| \ddot\nu B_{\max} \left( \nu/\nu_K
\right)^2 \left[ A_{\max}^2 - A_{\min}^2 \right].
\end{equation}
(Here we have dropped the $B$ terms in $f_{\min}$ and $f_{\max},$ since they
are small corrections, and $g$ indicates the determinant of the metric.)
To estimate the number of templates, we divide by the proper volume per
template~\cite{Owen_1996},
\begin{equation}
V = \left( 2 \sqrt{\mu/3} \right)^3 \simeq 0.138
\end{equation}
for a three-dimensional template bank mismatch $\mu$ of 0.2.
In cases where $\ddot f$ is not needed, these expressions need to be modified,
but those cases are so computationally cheap that they are not an issue.
In practice the number of templates is modified from these estimates by the
vagaries of the actual template placement code, typically dominated by the
problem of covering the edges of long narrow stretches of parameter space.
Our tests with LALSuite~\cite{LALSuite} show that a real search might use
three times as many templates as these ideal numbers.
Since this factor can vary for each search, we do not include it further or
attempt to estimate the numbers too precisely.

To get values for the number of templates, we use $\sqrt{g} \simeq T^6 \times
8.41\times10^{-3}$ where $T$ is measured in seconds.
We find that for \ac{O1} using ATNF values of $\ddot\nu,$ the Crab requires
$6\times10^9$ templates and the others generally require one or more orders of
magnitude fewer than the Crab.
Taking catalogue numbers at face value, the exception is J0537\textminus6910
which requires $1.5\times10^{10}$ templates.
Using the maximum $\ddot f$ derived from a braking index of 7, the Crab
triples to about $2\times10^{10}$ templates.
Using ATNF spin-downs, for \ac{O2} the Crab and PSR~J0537\textminus6910
require of order $5\times10^{11}$ and $1\times10^{12}$ templates, and for a
one year integration they require $3\times10^{12}$ and $7\times10^{12}.$
The latter number is the same as the first directed search for an isolated
neutron star~\cite{Abadie:2010hv}, whose cost was modest by the standards of
continuous \ac{GW} data analysis.

Again, these numbers are likely to be larger in reality due to the template
placement algorithms in LALSuite~\cite{LALSuite}.
And, although we make rough blanket statements here, for a real search each
pulsar needs some investigation into timing noise and glitches.
For example, while PSR~J0537\textminus6910 might be a very interesting pulsar
to search, it is known to glitch frequently and because it is visible only in
x-rays it is important to maintain satellite timing~\cite{Andersson:2017fow}.
Even with timing, if this pulsar glitches in the middle of a \ac{GW} observing
run, the run will need to be divided into segments.
For another example, since in \ac{O2} the noise performance of the Hanford
interferometer was usually significantly worse than the Livingston
interferometer at the frequencies of most pulsars with high spin-down limits,
some pulsar searches might use only data from Livingston with little loss in
sensitivity.

To convert template numbers to computational cost, we run a piece of the
latest directed search code used in~\cite{Abbott:2018qee} to estimate the
computational cost per \ac{SFT} of 30 minutes of data per template.
On the LIGO-Caltech cluster Broadwell and Skylake benchmarking nodes the cost
is generally somewhat less than 50~ns per \ac{SFT} per template, depending on
network activity and disk throughput.
For \ac{O1} the number of \acp{SFT} was about $6\times10^3.$
For \ac{O2} the number is about double, but we still use $7\times10^3$
assuming a search which does not integrate data from the Hanford
interferometer because its noise is significantly worse than that at
Livingston for the low frequencies considered here.
For a one year search of future data we assume two interferometers at a duty
cycle of 70\% each, comparable to the most stable past operation of the
interferometers, resulting in $2.5\times10^4$ \acp{SFT}.

Under these assumptions the cost of a Crab search is of order 500 core-hours,
$5\times10^4,$ or $1\times10^6$ for \ac{O1}, \ac{O2}, or one year
respectively.
This indicates that searching all pulsars for \ac{O1} and \ac{O2} is not a
computational problem, even though the number of templates is likely to be
larger in reality.
The one-year figure for the Crab is comparable to the total power used in a
bundle of recent directed searches~\cite{Abbott:2018qee}.
It is not outrageously expensive, but indicates that soon it will be desirable
to reduce the costs through a combination of theory and data analysis
innovations.

The density of templates per unit frequency is useful in estimating
sensitivity (below) and in load balancing the code.
Derived similarly to the proper volume~(\ref{propvol}) but omitting the $\int
df$ and dividing by the volume per template $V,$ this density is approximately
\begin{equation}
2\sqrt{g} B_{\max} \left( \nu/\nu_K \right)^2 \left| \dot\nu \right| \ddot\nu
f/\nu / V.
\end{equation}
For \ac{O1} at the maximum end of the frequency ranges, this takes the values of
$1.1\times10^9$ and $1.4\times10^9$\,Hz$^{-1}$ for the Crab and J0537
respectively.
For \ac{O2} the corresponding numbers are about $9\times10^{10}$ and
$1.1\times10^{11},$ and for a one year integration they are about
$5\times10^{11}$ and $7\times10^{11}.$

\section{Sensitivity}

\begin{figure*}
\includegraphics[angle=270,width=\linewidth]{figure/rmeth1}
\caption{
\label{fig1}
Spin-down limits for interesting pulsars (horizontal lines) and sensitivity
estimates (other curves), both in terms of intrinsic strain vs.\ \ac{GW}
frequency.
Spin-down limits are taken from Eq.~(\ref{h0sd}) in the text.
Sensitivity estimates are taken from Eq.~(\ref{h0Td}) and the paragraph
containing it.
}
\end{figure*}

\begin{figure*}
\includegraphics[angle=270,width=\linewidth]{figure/rmeth2}
\caption{
\label{fig2}
Same as the previous figure, for higher frequencies.
There are fewer pulsars here, but the spin-down limits on $r$-mode amplitude
are generally closer to predictions of saturation amplitude.
}
\end{figure*}

We express the sensitivity of each search in terms of upper limits on $h_0$
that can be placed in the absence of a detection.
This is slightly pessimistic---the upper limits are conservative by design and
it is plausible that a somewhat fainter signal could be detected---but it
facilitates comparison with published upper limits from previous searches for
continuous \acp{GW}.
The precise definition of $h_0^\mathrm{UL}$ we use is the same as for instance
in Ref.~\cite{Abbott:2018qee}.
It is a 95\% confidence limit on a population of injected signals with fixed
$h_0$ but varying frequency (within a small band), frequency derivatives, and
angles of inclination and polarization.

The sensitivity of a search of data from a single detector with stationary
noise can be expressed as~\cite{Wette:2011eu}
\begin{equation}
\label{h0Td}
h_0^\mathrm{UL} = \frac{5}{2} \hat\rho \sqrt{ \frac{S_h} {T_d} },
\end{equation}
where $S_h$ is the strain noise \ac{PSD} and $T_d$ is the amount of data.
(In general $T_d$ is less than the integration span $T$ times the number of
interferometers due to maintenance, earthquakes, and so on.)
For multiple detectors or non-stationary noise the \ac{PSD} in
Eq.~(\ref{h0Td}) is replaced by a weighted sum~\cite{Jaranowski:1998qm,
Cutler:2005hc}.
For observations of many days at most sky locations, the sum is very close to
the harmonic mean of noise \acp{PSD}, so we will use the harmonic mean when we
give numbers later.
The statistical factor $\hat\rho$ is iteratively estimated to sufficient
precision using the method of Wette~\cite{Wette:2011eu}, using the template
densities above and assuming that the upper limits are placed on 0.1\,Hz
frequency bands.
(This upper limit band might be chosen differently for different searches, but
its effect on sensitivity is negligible.)
The factor $5\hat\rho/2$ ranges about 33--38 for the searches considered here,
comparable to the factor for directed searches~\cite{Abbott:2018qee} and about
triple the factor for exact timing searches of known
pulsars~\cite{Authors:2019ztc}.
As an alternative, one can write this in terms of sensitivity depth, or the
square root of the \ac{PSD} divided by $h_0$ as in
Ref.~\cite{Dreissigacker:2018afk}.
For the searches considered here, the sensitivity depth is on the order of
110--170\,Hz$^{-1/2}.$
This is comparable to values achieved~\cite{Dreissigacker:2018afk} with narrow
band pulsar searches such as Ref.~\cite{Abbott:2019bed}, or somewhat better
due to longer integration times.
We consider noise \acp{PSD} for \ac{O1}~\cite{H1O1, L1O1}, \ac{O2}~\cite{H1O2,
L1O2}, Advanced LIGO design~\cite{Design}, and the recently funded A+
design~\cite{A+}.
Hence we show four sensitivities:
\ac{O1}, \ac{O2}, and one year integrations at Advanced LIGO and A+ design.

In Figs.~\ref{fig1} and~\ref{fig2} we plot our sensitivity measure vs.\
frequency for the four cases mentioned above, superposed on a set of spin-down
limits for known pulsars from the ATNF catalogue~\cite{Manchester:2004bp}.
Most of the pulsars plotted have already been searched for \acp{GW} at
$f=2\nu$ (and some also at $f=\nu$) in previous LIGO and Virgo papers based on
exact pulsar timing solutions~\cite{Authors:2019ztc}.
Most of the pulsars whose spin-down limits are accessible with existing data
are young and energetic (and sometimes glitchy) like the Crab, and most are
shown in Fig.~\ref{fig1}.
As with known-timing searches, the Crab is the first spin-down limit to become
accessible (already in \ac{O1}), and several more including Vela soon follow.
For later noise curves some middle-aged pulsars (in Fig.~\ref{fig1}) and some
recycled millisecond (in Fig.~\ref{fig2}) pulsars become accessible.
Most notable in Fig.~\ref{fig2} are J0537\textminus6910 and
J0437\textminus4715.
Due to its frequency and proximity to Earth, the latter has
$\alpha_\mathrm{sd}$ of order $10^{-5}$---much lower, and hence more feasible,
than the other accessible pulsars, although $h_0^\mathrm{sd}$ indicates this
pulsar will require at least A+ to detect.

Not all of these pulsars are timed concurrently with LIGO-Virgo observing
runs.
Since the searches proposed here cover broad frequency bands, the uncertainty
in frequency and spin-down parameters is not an issue---unlike the $\nu$ and
$2\nu$ searches.
Our proposed searches do suffer in sensitivity if a pulsar glitches during the
integration, though; and they cannot account for the fluctuating torques
likely in accreting systems.
The glitch issue means that frequent x-ray timing of J0537\textminus6910 will
be important for future \ac{GW} observing runs~\cite{Andersson:2017fow}.

\section{Discussion}

We have shown that searches for continuous \acp{GW} from $r$-modes of known
pulsars can beat the spin-down limits on some pulsars in existing data for
reasonable computational costs.
Although the $r$-mode amplitudes required for detection in such data are
higher than predicted by theory, this work serves as a starting point for
future improvements.
Spin-down limits for many more pulsars will be attainable in the next few
years.

Part of our goal is to point out what theory work could be most important to
help observations.
It is crucial to get the range of mode frequencies and spin-down parameters
right, and helpful to narrow the range down and reduce costs.
Relating frequencies and spin-down parameters more precisely to neutron star
properties will also help measure the latter once a signal is detected.

The most important feature of the search is the $r$-mode frequency range.
Avoided crossings such as that with $t$-modes in the crust could widen the
parameter ranges of some pulsars well beyond what we consider here.
Some pulsars could be undetectable without addressing the avoided crossings
problem.
Updated ranges of the $A$ and $B$ parameters of Eq.~(\ref{fnu}) would also
help in terms of narrowing the parameter space and hence reducing the
computational costs, which will grow to be substantial in coming years.

On the computational side, a pipeline that divides parameter space in a way
suitable for use of the resampled $\mathcal{F}$-statistic~\cite{Patel:2009qe}
would allow for significantly reduced computational cost for long
integrations.

More estimates of saturation amplitudes would be helpful.
This is a very difficult problem, and essentially has been addressed only by
one approach~\cite{Arras:2002dw}.

It would also help to be sure of the coherence time.
If saturated $r$-modes in equilibrium with other modes occasionally experience
phase jumps, this would render long coherent integrations of \ac{GW} data
problematic.

The coherence time issue leads into future work for data analysis:
Accretion and glitches can also introduce issues which encourage development
of alternatives to the straightforward coherent integrations considered here.
Adaptations of semi-coherent techniques developed for other
searches~\cite{Sun:2017zge, Suvorova:2017dpm, Dergachev:2011pd,
Ashton:2018qth} could be fruitful for $r$-modes from known pulsars too.
For those pulsars with a known inclination angle, upgrading from the
$\mathcal{F}$-statistic to the $\mathcal{G}$-statistic will also improve
sensitivity.

\section*{acknowledgments}

We are grateful to the continuous waves search group of the LIGO Scientific
Collaboration, particularly Ian Jones and Karl Wette, for helpful discussions.
This work was supported by NSF grant PHY-1607673.
This paper has been assigned document number LIGO-P1900173.













%%%%%%%%%%%%%%%%%%%%%%%%%%%%%%%%%%%%%%%%%%%%%%%%%%%%%%%%%%%%%%%%%%%%%%%%%%%%%%%%%%%%%%%%%
%%%%%%%%%%%%%%%%%%%%%%%%%%%%%%%%%%%%%%%%%%%%%%%%%%%%%%%%%%%%%%%%%%%%%%%%%%%%%%%%%%%%%%%%%
%%% 							END OF Fourth CHAPTER								  %%%
%%%%%%%%%%%%%%%%%%%%%%%%%%%%%%%%%%%%%%%%%%%%%%%%%%%%%%%%%%%%%%%%%%%%%%%%%%%%%%%%%%%%%%%%%
%%%%%%%%%%%%%%%%%%%%%%%%%%%%%%%%%%%%%%%%%%%%%%%%%%%%%%%%%%%%%%%%%%%%%%%%%%%%%%%%%%%%%%%%%

\chapter{\textbf{Data Analysis}}
\section{Introduction}
Searches for continuous \acp{GW} can be divided into different categories,
according to the information on the sources. Pulsars like the Crab are routinely
monitored in radio, so we know the spin frequency and sky coordinates. The
neutron star radiates \ac{GW} at twice the spin frequency due to the mass
quadrupole. A search with a fully known waveform is known as a targeted search.
They are computationally cheaper and more sensitive than others. And other type
of search is the directed search where the sky coordinates are known but not the
spin frequency. The $r$-mode frequency is uncertain over a small band when
relativistic and other corrections are considered. Supernova remnants like
Cassiopeia A have neutron stars that are not seen emitting pulses, probably
because the narrow beam of light doesn't sweep towards the earth. So, a large
frequency band is required to search for \ac{GW} from Cassiopeia A which
increases the computational cost. The type of search for unknown neutron stars
is known as an all sky search. There are around $10^8$ neutron stars in the
Milky Way but only 2000 are observed. Some of the neutron stars that are not
observed in \ac{EM} radiation can be detected in \ac{GW}. One of the all sky
searches was done for the whole sky from 20 to 1922 Hz~\cite{Abbott_2019a}. Due
to computational limitations, the all-sky search have short observational times
so they are less sensitive than other types of continuous \ac{GW} searches.

\begin{table}[ht]                                                               
\centering                                                                                                                                      
\begin{tabular}{lcccc}                                                          
\hline                                                                          
\hline
Type of Search&Sky Location&frequency&Computational Cost&Sensitivity\\[15pt]          
\hline                                                                          
Targeted&Known&Known&\tikzmark{a}{}&\tikzmark{c}{}\\[10pt]                            
%Narrowband&Known&Known&&\\                                                      
Directed&Known&Unknown&&\\[10pt]                                                      
All Sky&Unknown&Unknown&\tikzmark{b}{}&\tikzmark{d}{}\\                         
\hline
\hline                                                                          
\end{tabular}                                                                   
\link{a}{b}\link{d}{c}                                                          
\caption{Different types of continuous gravitational wave search.}
\label{tab}                                                                    
\end{table}         

In this chapter we will discuss the construction of a signal in
\acp{CW}, the computational cost of the search by looking at the algorithm to
generate a template bank. We will also talk about how we claim a detection of real
signals and estimate the sensitivity of our search.

\section{Signal Model} 
We here derive the model of the waveform considering both Doppler and
amplitude modulation of the signal. This will be a summary of the data analysis
paper by \citet{Jaranowski_1998}.

If $L$ is the length of the detector and $\lambda$ the wavelength of gravitational
wave, in the long wave approximation $\lambda \gg L$. The time delays of
waves propagating over the detector are negligible, so the
gravitational wave field can be treated as being uniform all over the detector.
Let $h$ be the relative change of arm length of the detector when a gravitational wave
passes by. Then
\begin{equation}
h(t) = \frac{1}{2} n_1\cdot[\tilde{H}(t)n_1] - \frac{1}{2} n_2\cdot [\tilde{H}(t)n_2],
\end{equation}
where $n_1=(1,0,0)$ and $n_2=(0,1,0)$ denote unit vectors parallel to the
detector arm, assuming $n_1 \perp n_2$. And $\tilde{H}$ is defined as:
\begin{equation}
\tilde{H}(t) = M(t)H(t)M(t)^T,
\end{equation}
\begin{equation*}
H(t)=
\begin{pmatrix}
h_+(t) & h_\times(t) & 0\\
h_\times(t) & -h_+(t) & 0\\
0 & 0 & 0
\end{pmatrix},
\end{equation*}
where, $h_+$ and $h_\times$ are plus and cross polarization of \acp{GW}. Also
\begin{equation}
M= M_3 M_2 M_1,
\end{equation}
where
$M_1$ is the transformation matrix from wave to celestial sphere with
Euler angle ($\alpha,\delta,\psi$).\\
$M_2$ is the transformation from celestial coordinates to cardinal
coordinates ($\mathrm{NSEW}$).\\
$M_3$ is the transformation from cardinal to detector reference frame.

\begin{equation}
\scriptsize
M_1=
\begin{pmatrix}
\cos{-\Psi} & \sin{-\psi}& 0\\
-\sin{-\Psi} & \cos{\psi}& 0\\
0 & 0 & 1
\end{pmatrix}
\begin{pmatrix}
1 & 0 & 0\\
0 & \cos{(-\frac{\pi}{2}-\delta)} & \sin{(-\frac{\pi}{2}-\delta)}\\
0 & -\sin{(-\frac{\pi}{2}-\delta)} & \cos{(-\frac{\pi}{2}-\delta)}
\end{pmatrix}
\begin{pmatrix}
\cos{(\frac{\pi}{2}-\alpha)} & \sin{(\frac{\pi}{2}-\alpha)} & 0\\
-\sin{(\frac{\pi}{2}-\alpha)} & \cos{(\frac{\pi}{2}-\alpha)} & 0\\
0 & 0 & 1 
\end{pmatrix}
\end{equation}

\begin{equation}
\scriptsize
M_2=
\begin{pmatrix}
\sin{\lambda}\cos{(\phi_r+\Omega_rt)} & \sin{\lambda}\sin{(\phi_r+\Omega_rt)} 
& -\cos{\lambda}\\
-\sin{()\phi_r+\Omega_rt)} & \cos{(\phi_r+\Omega_rt)} & 0\\
\cos{\lambda}\cos{(\phi_r+\Omega_rt)}  & \cos{\lambda}\sin{(\phi_r+\Omega_rt)} 
& \sin{\lambda}
\end{pmatrix}
\end{equation}

\begin{equation}
\scriptsize
M_3=
\begin{pmatrix}
\cos{(\frac{\pi}{2}+\gamma)} & \sin{(\frac{\pi}{2}+\gamma)} & 0\\
-\sin{(\frac{\pi}{2}+\gamma)} & \cos{(\frac{\pi}{2}+\gamma)} & 0\\
0 & 0 & 1
\end{pmatrix}
\end{equation}

\begin{figure}[h!]
  \centering
  \begin{minipage}{0.4\textwidth}
    \includegraphics[width=\textwidth]{figure/wavedetector.jpg}
  \end{minipage}
  \hfill
  \begin{minipage}{0.6\textwidth}
    \includegraphics[width=\textwidth]{figure/wavecelestial.jpg}
  \end{minipage}
  \caption{Coordinate transformation from wave frame to detector frame. Image
   from~\cite{1996A&A...312..675B}.}
  \label{fig:Wavetodetector}
\end{figure}

$\lambda \rightarrow \text{latitude of detector}$\\
$\Omega_r \rightarrow \text{rotational angular velocity of Earth}$\\
$\phi_r \rightarrow \text{phase which defines the position of the Earth in its
diurnal motion}$\\
$\phi_r + \Omega_r t \rightarrow$ local sidereal time of the detector [the
angle between local meridian and the vernal point].\\
$\gamma \rightarrow \text{orientation of the detector's arms with respect to
local north.}$\\
This is the transformation using the rotation matrix from wave frame to the
detector frame. 

The response of the interferometer is given by:
\begin{equation}
h(t) = F_+(t) h_+(t) + F_\times(t) h_\times(t) 
\end{equation}
where $F_+$ and $F_\times$ are the antenna patterns. The antenna patterns are periodic
functions of time with period one day. The antenna patterns are 
functions of right ascension ($\alpha$), declinationi ($\delta$) of the \acp{GW}
sources and polarization angle ($\Psi$). 

The antenna patterns $F_+$ and $F_\times$ can be written as:
\begin{align*}
F_+(t) = a(t)cos(2\psi) + b(t) sin(2\psi),\\
F_\times(t) = b(t)cos(2\psi) - a(t) sin(2\psi),
\end{align*}
\begin{align}
\small
\begin{split}\label{eq:a}
a(t) &=
\frac{1}{16}\sin{2\gamma}(3-\cos{2\lambda})(3-\cos{2\delta})\cos{[2(\alpha-\phi_r-\Omega_rt)]}\\
&-\frac{1}{4}\cos{2\gamma} \sin{\lambda} (3-\cos{2\delta})
\sin{[2(\alpha-\phi_r-\Omega_rt)]}\\
&+\frac{1}{4}\sin{2\gamma} \sin{2\lambda} \sin{2\delta}
\cos{[\alpha-\phi_r-\Omega_r t]} \\
&-\frac{1}{2}\cos{2\gamma} \cos{\lambda} \sin{2\delta}
\sin{[\alpha-\phi_r-\Omega_r t]}
+\frac{3}{4} \sin{2\gamma} \cos^2{\lambda} \cos^2{\delta}
\end{split}\\
\begin{split}\label{eq:b}
b(t)& =\cos{2\gamma} \sin{\lambda} \sin{\delta} \cos{[2(\alpha-\phi_r-\Omega_r
t)]}\\ 
&+\frac{1}{4} \sin{2\gamma} (3-\cos{2\lambda}) \sin{\delta}
\sin{[\alpha-\phi_r-\Omega_rt]}\\
&+ \cos{2\gamma} \cos{\lambda} \cos{\delta} \cos{[\alpha-\phi_r-\Omega_r t]}\\
&+\frac{1}{2}\sin{2\gamma} \sin{2\lambda} \cos{\delta}
\sin{[\alpha-\phi_r-\Omega_rt]},
\end{split}
\end{align}

The gravitational wave signal can be written in terms of Doppler
parameters ($\alpha, \delta, f, f_n $) and amplitude parameters ($h_0, \iota, \psi,
\phi$):
\begin{equation}
h(t) = \sum_{i=1}^{4}A_{1i} h_{1i}(t) + \sum_{i=1}^{4} A_{2i} h_{2i}(t),
\end{equation}
The amplitudes $A_{1i}$ and $A_{2i}$ are given by:
\begin{align*}
A_{11} = h_0 \sin{2\theta}\left[\frac{1}{8} \sin{2\iota}\cos{2\psi}\cos{\phi_0}-
\frac{1}{4}\sin{\iota} \sin{2\psi}\sin{\psi_0}\right], \\
A_{12} = h_0 \sin{2\theta}\left[\frac{1}{4} \sin{\iota}\cos{2\psi}\sin{\phi_0}+
\frac{1}{4}\sin{2\iota} \sin{2\psi}\cos{\psi_0}\right], \\
A_{13} = h_0 \sin{2\theta}\left[-\frac{1}{8} \sin{2\iota}\cos{2\psi}\sin{\phi_0}
-\frac{1}{4}\sin{\iota} \sin{2\psi}\cos{\psi_0}\right], \\
A_{14} = h_0 \sin{2\theta}\left[\frac{1}{4} \sin{2\iota}\cos{2\psi}\cos{\phi_0}-
\frac{1}{8}\sin{2\iota} \sin{2\psi}\sin{\psi_0}\right], \\
A_{21} = h_0 \sin^2{\theta}\left[\frac{1}{2}
(1+\cos^2{\iota})\cos{2\psi}\cos{2\phi_0}- \cos{\iota} \sin{2\psi}\sin{2\psi_0}\right], \\
A_{22} = h_0 \sin^2{\theta}\left[\frac{1}{2}
(1+\cos^2{\iota})\sin{2\psi}\cos{2\phi_0} + \cos{\iota} \cos{2\psi}\sin{2\psi_0}\right], \\
A_{23} = h_0 \sin^2{\theta}\left[-\frac{1}{2}
(1+\cos^2{\iota})\cos{2\psi}\sin{2\phi_0}- \cos{\iota} \sin{2\psi}\cos{2\psi_0}\right], \\
A_{24} = h_0 \sin^2{\theta}\left[-\frac{1}{2}
(1+\cos^2{\iota})\sin{2\psi}\sin{2\phi_0} + \cos{\iota} \cos{2\psi}\cos{2\psi_0}\right]\\
\end{align*}
And the time dependent Doppler functions $h_{i}$ are:
\begin{align}
\begin{split}\label{h}
h_{1}=a(t) \cos{\phi(t)},\quad h_{2}=b(t) \cos{\phi(t)},\\ 
h_{3}=a(t) \sin{\phi(t)},\quad h_{4}=b(t) \sin{\phi(t)}, 
\end{split}
\end{align}
where, a(t) and b(t) are defined in \ref{eq:a} \& \ref{eq:b}.



\subsection{Maximum Likelihood ratio}

The data from the gravitational wave interferometers are subject to various
noise sources and the signals $h(t)$ are buried under the noise $n(t)$. The challenging part
of \ac{GW} astronomy is to extract the weak signals from the random noise.
Most of the computational power is needed to filter out the noise. The
detector data $x(t)$ can be written as :
\begin{equation}
x(t) = h(t)+n(t),
\end{equation}
The maximum likelihood ratio can be defined as the ratio of the probability density function
of when a signal is present to when a signal is absent,
\begin{equation}\label{eqn:lhood}
\Lambda = \frac{p(x|h+n)}{p(x|n)}
\end{equation}
Here the scalar product is defined as:
\begin{equation}
(x|y)=4\mathcal{R} \int_{0}^{\infty}\frac{\tilde{x}(f)\tilde{y}^*(f)}{S_h(f)}df, 
\end{equation}
where $\tilde{x}$ is the Fourier transform, $^*$ is the complex conjugate and 
$S_h$ is the power spectral density of the detector noise.
Assuming the noise is a zero mean, stationary, Gaussian random process of
variance unity, the probability density function of noise will be:
\begin{align*}
p(x|n) & = \frac{1}{2\pi}\exp(-(x-0)^2/(2*1))\\
&= \frac{1}{2\pi} \exp(-\frac{1}{2}(x|x))
\end{align*}
Similarly, if the data contains a signal $h$, now the mean will be $h$.
\begin{equation}
p(x|h+n)=\frac{1}{2\pi} \exp(-\frac{1}{2}(x-h|x-h))
\end{equation}
Now, the likelihood ratio~\ref{eqn:lhood} can be written as:
\begin{equation}
\Lambda= \frac{\exp(-\frac{1}{2}(x-h|x-h))}{\exp(-\frac{1}{2}(x|x)}
\end{equation}
Taking the log of likelihood ratio:
\begin{align*}
\ln{\Lambda} & =-\frac{1}{2}(x-h|x-h) + \frac{1}{2}(x|x)\\
& = \cancel{-\frac{1}{2}(x|x)}
+(x|h)-\frac{1}{2}(h|h)+\cancel{\frac{1}{2}(x|x)},
\end{align*}
the log likelihood function can be written as:
\begin{equation}
\ln{\Lambda}=(x|h)-\frac{1}{2}(h|h)
\end{equation}
The signal $h(t,\vec{A},\vec{\lambda})$ depends linearly on the amplitude
parameters ($\vec{A}$) $h_0$, $\theta$, $\psi$, $\iota$ and $\phi$. The maximum
value of the likelihood function can be found by:
\begin{equation}
\frac{\partial\ln\Lambda}{\partial A_i}= 0, \quad i=1,....,4,  
\end{equation}

\begin{equation}
\sum_{j=1}^4M_{ij}A_j =(x|h_i),
\end{equation}
where M is a $4 \times 4$ matrix given by:
\begin{equation}
M_{ij} = (h_i|h_j),
\end{equation}
and where
\begin{equation}
(h_i|h_j)=\frac{2}{T_0}\int_{-T_0/2}^{T_0/2}h_i(t)h_j(t)dt, \quad
T_0 \rightarrow \text{observational time.}
\end{equation}
From Eqn.\ref{h}\quad $h_1$ $\&$ $h_2$ $\propto$ $\cos{\phi(t)}$\quad \& \quad $h_3$ $\&$ $h_4$
$\propto$ $\sin{\phi(t)}$, so,
\begin{equation}
(h_1|h_3)=0, \quad (h_1|h_4)=0, \quad (h_2|h_3)=0, \quad (h_2|h_4)=0,  
\end{equation}
and
\begin{equation}
\begin{split}
(h_1|h_1)=(h_3|h_3)=\frac{1}{2}A,\\
(h_2|h_2)=(h_4|h_4)=\frac{1}{2}B,\\
(h_1|h_2)=(h_3|h_4)=\frac{1}{2}C,\\
\end{split}
\end{equation}
where  $A=(a(t)|a(t))$, \quad  $B=(b(t)|b(t))$, \quad  $C=(a(t)|b(t))$ and 
\begin{equation*}
M=\frac{1}{2}
\begin{pmatrix}
A & C & 0 & 0\\
C & B & 0 & 0\\
0 & 0 & A & C\\
0 & 0 & C & B
\end{pmatrix},
\end{equation*}
The $\mathcal{F}$-statistic is the maximization of the likelihood function over
$\vec{A}$, given by 
\begin{equation}
2\mathcal{F}(\vec{\lambda})=\sum_{i,j=1}^{4}[M^{-1}]_{ij}(x|h_i)(x|h_j),
\end{equation}
where
\begin{equation*}                                                               
M^{-1}=\frac{2}{D}                                                                   
\begin{pmatrix}                                                                 
B & -C & 0 & 0\\                                                                 
-C & A & 0 & 0\\                                                                 
0 & 0 & B & -C\\                                                                 
0 & 0 & -C & A                                                                   
\end{pmatrix}                                                                   
\end{equation*} 
and $D=AB-C^2$. Thus
\begin{equation}
\small
2\mathcal{F}=\frac{2}{D}\left[B(x|h_1)^2 + A(x|h_2)^2 -2C(x|h_1)(x|h_2)+
B(x|h_3)^2 + A(x|h_4)^2 -2C(x|h_3)(x|h_4)\right],
\end{equation}
Since 2$\mathcal{F}$ is maximized over the amplitude parameter, it is a
function of remaining Doppler parameter ($\vec{\lambda}\rightarrow \alpha,\delta,f^n$). The signal
template $h(t,\vec{A},\vec{\lambda})$ is not a linear function of Doppler
parameter ($\vec{\lambda}$), so we need to find the maximum 2$\mathcal{F}$ value over
$\vec{\lambda}$ numerically. 

First, we need to calculate the probability density of $2\mathcal{F}$ when a
signal is present and when a signal is absent. The detection of a signal is claimed
when $2\mathcal{F}$ beats the certain threshold level with a given false alarm
rate. 

Lets suppose we know $h_i$, i.e. the sky position and spin down parameters. When
the signals are absent $(x|h_i)$ = 0. To make the calculation easier, let's
assume the observational time is an integer multiple of one sidereal day, so $C
= 0$~\cite{Jaranowski_2000}. When the signal is present:
\begin{align*}
(x|h_1)= \frac{1}{2}AA_1, \quad (x|h_2) = \frac{1}{2}B A_2,\\
(x|h_3)= \frac{1}{2}AA_3, \quad (x|h_4) = \frac{1}{2}B A_4
\end{align*}
Now, we can write the $\mathcal{F}$-statistic as:
\begin{equation}
\mathcal{F} = \frac{1}{2}(z_1^2+z_2^2+z_3^2+z_4^2),
\end{equation}
where,
\begin{equation}
z_i=2\sqrt{\frac{T_0}{S_hA}}(x|h_i), \quad i=1,3,
\end{equation}
\begin{equation}
z_i=2\sqrt{\frac{T_0}{S_hB}}(x|h_i), \quad i=2,4
\end{equation}
In the absence of a signal $2\mathcal{F}$ is drawn from a chi-squared distribution with four
degrees of freedom. And when the signals are present, $2\mathcal{F}$ is
drawn form a
non-central chi-squared distribution with four degrees of freedom and the non
centrality parameter is the optimal signal to noise ratio ($\rho$) defined as:
\begin{equation}
\rho^2=\frac{2}{S_h(f)}\int_0^{T_0} h^2(t)dt,
\end{equation}

The probability density functions when a signal is absent ($p_0$) and when a signal is present ($p_1$) are:
\begin{equation}
\begin{split}
p_0(\mathcal{F})&=\frac{\mathcal{F}^3}{6}exp{-\mathcal{F}},\\
p_1(\rho,\mathcal{F})&=\frac{2\mathcal{F}^3/2}{\rho^3}I_3(\rho\sqrt{2\mathcal{F}})exp{-\mathcal{F}-\frac{1}{2}\rho^2},
\end{split}
\end{equation}
where $I_3$ is the modified Bessel function of the first kind and order 3.
The expectation value of 2$\mathcal{F}$ is 4+$\rho^2$.
\begin{figure}[h!]
	\includegraphics[width=\textwidth]{figure/chi2.png}
	\caption{In the absence of signal the pdf will be a $\chi^2$ distribution with
four degrees of freedom. In the presence of signal the pdf will be a non central
$\chi^2$ distribution where the non central parameter is the signal to noise
ratio.}
\end{figure}

\subsection{Matched Filtering technique}
Matched filtering is an optimal technique of maximizing the signal to noise ratio by
correlating the detector data with a known waveform~\cite{Sathyaprakash_2009}.
The detector output $x(t)$ contains a signal $h(t)$ and noise $n(t)$, and let
$q(t)$ be
a template of known shape. Let $c$ be the correlation between the template and
detector output:
\begin{equation}\label{eq:matchfil1}
c(\tau) = \int_{-\infty}^{\infty} x(t)q(t+\tau)dt,
\end{equation}
where $\tau$ is the time delay between the template and the detector data.
The time series data can be converted into the frequency domain using the Fourier
transform. 
\begin{equation}\label{eq:matchfil2}
x(t)=\int_{-\infty}^{\infty} \tilde{x}(f)e^{-i2\pi f t} df\\
\end{equation}
\begin{equation}\label{eq:matchfil3}
\begin{split}
q(t+\tau) &=\int_{-\infty}^{\infty} \tilde{q}(f^{'})e^{-i2\pi f^{'}(t+\tau)}
df^{'}\\
 & =\int_{-\infty}^{\infty} \tilde{q}^{*}(f^{'})e^{2i\pi f^{'}(t+\tau)} df^{'}.
\end{split}
\end{equation}
Substituting equation ~\ref{eq:matchfil2} \& \ref{eq:matchfil2} into \ref{eq:matchfil1},

\begin{align*}
c(\tau) &= \int_{-\infty}^{\infty} dt \int_{-\infty}^{\infty}
df\int_{-\infty}^{\infty} df^{'} \tilde{x}(f)\tilde{q}^*(f^{'}) e^{-i2\pi
ft}e^{i2\pi f^{'}(t+\tau)}\\
  & = \int_{-\infty}^{\infty} df \int_{-\infty}^{\infty} df^{'}
\tilde{x}(f)\tilde{q}^*(f^{'})\underbrace{\int_{-\infty}^{\infty} dt e^{-i 2\pi
(f-f^{'})t}}_{\delta(f-f^{'})}
e^{i 2\pi f^{'}\tau}\\
 & = \int_{-\infty}^{\infty}df\tilde{x}(f)
\int_{-\infty}^{\infty} df^{'}\delta(f-f^{'})\tilde{q}^*(f^{'})e^{i 2\pi
f^{'}\tau}\\
 & = \int_{-\infty}^{\infty}df\tilde{x}(f)\tilde{q}^{*}(f)e^{i 2\pi f\tau}.
\end{align*}
The \ac{SNR} is defined as $\frac{\mu}{\sigma}$, where $\mu=\langle c \rangle$
is the mean and $\sigma$ is the standard deviation of the correlation,

\begin{equation}\label{eq:sign}
\begin{split}
\mu & = \langle\int_{-\infty}^{\infty}\tilde{q}^{*}(f)\tilde{x}(f)\rangle\\
&=\int_{-\infty}^{\infty}\tilde{q}^{*}(f)\tilde{h}(f),
\end{split}
\end{equation}
\begin{equation}\label{eq:noise1}
\begin{split}
\sigma^{2} & =<\int_{-\infty}^{\infty}\tilde{q}(f)\tilde{n}^{*}(f)df
\int_{-\infty}^{\infty} \tilde{q}^{*}(f^{'}) \tilde{n}(f^{'})df^{'}>\\
       &= \int_{-\infty}^{\infty}df|\tilde{q}(f)|^{2}
\int_{-\infty}^{\infty}\underbrace{\tilde{n}^{*}(f)
\tilde{n}(f^{'})}_{S_{n}(f)\delta(f - f^{'})}df^{'}\\
      &=\int_{-\infty}^{\infty}df|\tilde{q}(f)|^{2} S_{n}(f),
\end{split}
\end{equation}
From \ref{eq:sign} and \ref{eq:noise1}, we can get the \ac{SNR}.
\begin{equation}
\rho=\frac{\int_{-\infty}^{\infty}\tilde{q}^{*}(f)\tilde{h}(f)}{(\int_{-\infty}^{\infty}df|\tilde{q}(f)|^2
S_{n}(f))^{1/2}}
\end{equation}
The optimal signal to noise ratio is defined as:
\begin{equation}
\rho_{opt}^2 = \langle h,h \rangle
=2\left[\int_{0}^{\infty}\frac{|\tilde{h}(f)|^2}{S_{h}(f)}df\right]
\end{equation}
So, the optimal \ac{SNR} is the total energy of the signal weighted down by the
\ac{PSD} of the noise. So, if the \ac{PSD} of the noise at some frequency bin is higher,
the \ac{SNR} will be lower.


\subsection{Phases of gravitational waves}
Pulsars are spinning down by the emission of particles, \ac{EM} waves and \acp{GW}.
The frequency evolution is given by the Taylor expansion in the form:
\begin{equation}
f(t) = f_0 + \sum_{n=1}^{N} \frac{f_n}{n!}t^n,
\end{equation}
where $f_n$ are the spin down parameters. Usually, for less than one year of
search we don't need more than two spin down parameters. 
The frequency ($f_0$) of gravitational waves from the source is Doppler modulated due to the
rotation of Earth, so the apparent frequency ($f^\prime$) is given by:
\begin{equation} 
f^\prime = f_0 \left(1+\frac{\vec{v}\cdot\hat{r}}{c}\right)
\end{equation}
where $v$ is the relative velocity of the source with respect to the detector, $\hat{r}$
is the unit vector from the detector to the source and $c$ is the speed of light.

So the frequency of gravitational waves ($f_{gw}$) in the Solar System Barycenter
(SSB) frame is~\cite{Krishnan_2004}:
\begin{equation}
f_{gw}(t) = f_0 + \sum_{n=1}^{N} \frac{f_n}{n!}\left(t - t_0 + \frac{\Delta
\vec{r}(t)\cdot \vec{n}}{c}\right)^n,
\end{equation}
where $t_0$ is detector time at the start of the observation, and we are
neglecting the proper motion of neutron stars. 

\section{Short time baseline Fourier Transforms}
Pulsars are slowly spinning down, and the frequency of gravitational waves
in the interferometer will be Doppler modulated due to the rotation of earth.
Since the frequency of \acp{GW} is modulated, match filtering one chunk of
data for the whole observational time will be inconvenient. And the
interferometer noise is also non stationary, so the interferometer data is 
divided into $N$ segments and Fourier transformed. The computational cost of the
search increases linearly with the number of \acp{SFT} for a fixed observational
time. There are two main points accounted for in making \acp{SFT}, (i) the
duration of the
\acp{SFT} should be short enough so Doppler and spindown stay in one bin, (ii)
long enough for the noise to have a well-averaged floor~\cite{Krishnan_2004,Abbott_2007}. 

We derive the maximum \ac{SFT} duration such that the signal stays within half
a frequency bin~\cite{Krishnan_2004}. The frequency at the detector can be given by the Doppler formula:
\begin{equation}
\label{doppler}
f_d(t) - f_s(t) = f_s(t)\frac{\vec{v}(t)\cdot\hat{n}}{c},
\end{equation}
where $f_s$ is the frequency at the Solar System Barycenter (SSB) frame and
$\vec{v}$ is the velocity of the detector.
The rate of frequency change due to the Doppler effect can be calculated by
taking the derivative of Eqn \ref{doppler},
\begin{equation}
\dot{f}_= \frac{f_s}{c}\frac{d\vec{v}}{dt}\cdot\hat{n} \leq
\frac{f_s}{c}\left|\frac{d\vec{v}}{dt}\right|.
\end{equation}
The term $dv/dt$=acceleration, and it can be written as $a=v^2/R$, where $R$ is
the radius of earth. And we can also substitute $v=2\pi R/T$ where, $T$ is the time
period of rotation of earth to find
\begin{equation}\label{sfts1}
|\dot{f}|_{max}= \frac{4\pi^2R}{T}
\end{equation}
For the signal to stay within half of a frequency bin it needs to satisfy
the condition $|\dot{f}|T_{SFTs} < \frac{1}{2T_{SFTs}}$, or it can be written as:
\begin{equation}\label{sfts2}
T_{SFTs}<\sqrt{\frac{1}{2|\dot{f}_{max}|}}.
\end{equation}
Substituting \ref{sfts1} into \ref{sfts2}, we get,
\begin{equation}
T_{SFTs} < 18.5\,\mathrm{hours} \sqrt{\frac{1\,\mathrm{Hz}}{f_s}}.
\end{equation}
So for a pulsar emitting a \ac{GW} of frequency $100\,\mathrm{Hz}$, the duration of the
\ac{SFT} should be less than $1.85\,\mathrm{hours}$. The typical duration of \acp{SFT} in
\ac{LIGO} search is 30 minutes.


\section{Template Spacing}
In matched filtering the shape of the signal needs to be well known. The shape
is
determined by the sky location, frequency and the frequency derivatives of the
sources. The intrinsic strain, polarization amplitude, inclination angle, and
initial phase of a signal are not known so these are maximized quickly and
analytically by the $F$-statistic. In a
directed search, some parameters are not well known due to the lack of
information on the spin frequency of the source. For a \ac{GW} search for
$r$-modes
from a known pulsar, the spin frequency is known but we don't know the exact
$r$-mode frequency due to the unknown compactness of the neutron star. When the
exact waveform is not known, we need multiple templates that possibly can
match with a signal. The template spacing in parameters should consider both computational
cost and the loss in signal to noise ratio. If the distance between templates is
too far away
and the signal lies in between templates we will lose a significant amount of
\ac{SNR}. And, if the templates are too dense we will waste computational
power.  Let's match the signal $h(t,\vec{A},\vec{\lambda})$ with a slightly offset template
 $h(t,\vec{A},\vec{\lambda}+\Delta\vec{\lambda})$. The match is defined
as~\cite{Owen_1996}:
\begin{equation}
M(\vec{\lambda}+\Delta\vec{\lambda}) =
\mathrm{max}_{\vec{A}}\left\langle
h(t,\vec{A},\vec{\lambda})|h(t,\vec{A},\vec{\lambda}+\Delta\vec{\lambda})\right\rangle,
\end{equation}
This can be expanded in a power series about $\Delta\lambda=0$:
\begin{equation}\label{match}
M(\lambda,\Delta\lambda)\approx
1+\frac{1}{2}\left(\frac{\partial^2M}{\partial\Delta\lambda_i\partial\Delta\lambda_j}\right)_{\Delta\lambda^k=0}
\Delta\lambda^i\Delta\lambda^j
\end{equation}
We can define a metric 
\begin{equation}
g_{ij}(\lambda)=-\frac{1}{2}\left(\frac{\partial^2M}{\partial\Delta\lambda_i\partial\Delta\lambda_j}\right)_{\Delta\lambda^k=0}
\end{equation}
Now, Eqn~\ref{match} can be written as:
\begin{equation}
1-M=g_{ij} \Delta\lambda^i\Delta\lambda^j,
\end{equation}
where $1-M$ is the mismatch between two templates.

We define the parameter minimal match ($\mu$) as the maximum loss in signal to
noise ratio when the signal lies exactly in between two nearby templates. 
\section{Hypothesis testing}                                                    
        When we analyze a data and claim a detection we need to perform a certain
test on a sample of data. If our experimental results are from pure chance then
it is defined as the null hypothesis ($H_o$) and if the results rejects the null
hypothesis we define it as the alternative hypothesis. The alternative
hypothesis is
usually the one the researcher is looking for.  In our search if a signal is absent it is
called the null hypothesis and if signals are present we call it the alternative
hypothesis.
	In statistical tests there are two kinds of error in the
results. A type I error (False alarm probability) is when the experimenter
chooses $H_a$ when $H_o$ is true. A type II error (False Dismissal
Probability) is when the researcher chooses $H_o$ when $H_a$ is true.  So, in
a Type I error the test value is greater than the threshold significance level
when no signal is present. And, in type II error the test value is less than
the threshold when there are actually signals in the data.
The probability of False Alarm Rate is defined as:
\begin{equation}
P_{FAR}=P(2\mathcal{F}>2\mathcal{F}^*|H_o)=\int_{2\mathcal{F}^*}^\infty
pdf(2\mathcal{F}|H_o)d2\mathcal{F}.
\end{equation}
And, the equation of False Dismissal Rate is given by:
\begin{equation}
P_{FDR}=P(2\mathcal{F}<2\mathcal{F}^*|H_a)=\int_{-\infty}^{2\mathcal{F}^*}
pdf(2\mathcal{F}|H_a)d2\mathcal{F}.
\end{equation}



\section{Expected value of $2\mathcal{F}$}
Assuming we have Gaussian noise and no signal, the probability distribution
(pdf) of a single value of $2\mathcal{F}$ is drawn from a chi-squared
distribution with 4 degrees of freedom.  Let's say that out of $N$ templates, we
get a largest value denoted as $2\mathcal{F}^*$. The probability that a single
value of $2\mathcal{F}^*$ is largest is given by $p(\chi^2_4;2\mathcal{F}^*)$.
If $2\mathcal{F}^*$ is the largest value out of $N$ templates, then the other
$N$-1 templates must be less than $2\mathcal{F}^*$ and the probability of this
happening is:
\begin{equation}
p(2\mathcal{F}\leq2\mathcal{F}^*)=
[\mathrm{cdf}(\chi^2_4;2\mathcal{F}^*)]^{N-1}.
\end{equation} 
The probability that $2\mathcal{F}^*$ is the largest value of $2\mathcal{F}$
from a collection of $N$ values of $2\mathcal{F}$ is~\cite{Abadie_2010, Wette:2009uea}:
\begin{equation}                                                                
p(N;2\mathcal{F}^*)=
Np(\chi^2_4;2\mathcal{F}^*)[\mathrm{cdf}(\chi^2_4;2\mathcal{F}^*)]^{N-1}. 
\end{equation}

\begin{figure}[bht!]                                                            
        \includegraphics[width=\textwidth]{figure/2F_max.png}                  
	\caption{Probability as a function of $2\mathcal{F}$. The plot shows
clearly the $2\mathcal{F}$ value is a function of number of  templates. Image:
Santiago Caride.}                             
        \label{fig:2F_max}                                                  
\end{figure} 
These values then give a confidence interval; we can say that $95\%$ of the
time, the loudest  $2\mathcal{F}$ will be less than the threshold value of
$2\mathcal{F}$. If the loudest  $2\mathcal{F}$ from the search has probability
of less than 5\% of being drawn from chi squared distribution, then it is very
unlikely to be Gaussian noise. That could mean that we found a signal, or that
we found a source of non-Gaussian noise, like an instrumental line or a detector
artifact. We would then have to do more detailed follow up on our individual
outliers to figure out which of those is the case.

\section{Veto process}
Interferometer data is contaminated by different noise sources. Some of the
noise sources are from the instruments of the interferometer itself (power supply, mirror
suspension, blinking LEDs, fans....). These noises occur at specific
frequencies and might act as a signal of continuous gravitational waves or they
can even mask the signals. These noises known as instrumental lines need to be
properly identified and removed from the searches. We use the so called Fscan
veto to remove candidates in frequency bands with instrumental lines. An Fscan
is a normalized spectrogram of the SFTs and normalizes the SFTs by scaling the
power to the running median over 51 frequency bins~\cite{Aasi_2015}. If the
noise is stationary, Gaussian and smooth then the power is drawn from a $\chi^2$
distribution. If the Fscan power deviates from the $\chi^2$ distribution, then
it is non stationary Gaussian noise, spectral lines or both~\cite{Aasi_2015}.

The other veto process is the interferometer veto. If the $2\mathcal{F}$ value
of any one of the detectors is greater than the joint $2\mathcal{F}$ value of
all the detectors, then such a $2\mathcal{F}$ value will be discarded. 

\section{Upper limit}
If we didn't detect any signal ($2\mathcal{F}<2\mathcal{F}^*$) then we set an
upper limit on the intrinsic strain ($h_0$) of the signal that we are trying to
detect~\cite{Romano_2017}. To calculate the upper limit, we need the
$2\mathcal{F}$ value and confidence level of our search on specific templates. 
So, for a confidence level of $95\%$, we would have detected a signal of
$2\mathcal{F}>2\mathcal{F}^*$ at least $95\%$ of the time~\cite{Romano_2017}. In
our pipeline, we calculate the value of $h_0$ such that the false dismissal rate
is 0.05.
The upper limits on $r$-mode gravitational waves from the Crab pulsar will be
discussed later in Chapter ~\ref{chapter:6}.
%%%%%%%%%%%%%%%%%%%%%%%%%%%%%%%%%%%%%%%%%%%%%%%%%%%%%%%%%%%%%%%%%%%%%%%%%%%%%%%%%%%%%%%%%
%%%%%%%%%%%%%%%%%%%%%%%%%%%%%%%%%%%%%%%%%%%%%%%%%%%%%%%%%%%%%%%%%%%%%%%%%%%%%%%%%%%%%%%%%
%%% 							END OF FIFTH CHAPTER									  %%%
%%%%%%%%%%%%%%%%%%%%%%%%%%%%%%%%%%%%%%%%%%%%%%%%%%%%%%%%%%%%%%%%%%%%%%%%%%%%%%%%%%%%%%%%%
%%%%%%%%%%%%%%%%%%%%%%%%%%%%%%%%%%%%%%%%%%%%%%%%%%%%%%%%%%%%%%%%%%%%%%%%%%%%%%%%%%%%%%%%%

\chapter{\textbf{First searches for gravitational waves from $r$-modes of the
Crab pulsar}}\label{chapter:6}

\section{Introduction}

Rapidly rotating neutron stars might be detectable emitters of long lived
quasi-monochromatic radiation known as continuous
\acp{GW}~\cite{Glampedakis2018}. The emission mechanism for continuous waves
could be a nonaxisymmetric mass quadrupole or a current-quadrupolar $r$-mode.
Hence the detection of continuous \acp{GW} might help reveal the underlying
properties of neutron star interiors. The \ac{GW} frequencies of many pulsars
lie in the most sensitive band of \ac{LIGO}, so these rapidly rotating
neutron stars are attractive targets for continuous \ac{GW} searches.

The $r$-modes, whose frequencies are mainly determined by the Coriolis force,
are unstable to \ac{GW} emission~\cite{Andersson_1998,Friedman_1998} even
allowing for various damping mechanisms. Hence they might amplify and sustain
themselves, and might be the most interesting possibility for continuous
\acp{GW}. $R$-modes might play an important role in the spin-downs of the
fastest young neutron stars~\cite{Owen_1998} and in the regulation of spin
periods of some older accreting neutron stars. Comparison of the \ac{GW}
frequency to the spin frequency determined from timing radio or x-ray pulses
could measure the compactness of a pulsar.

Some \ac{GW} searches, starting with Ref.~\cite{Abadie_2010}, have set upper
limits on $r$-mode \ac{GW} emission. However most of these have been broad band
searches for neutron stars not previously known as pulsars. The searches
themselves did not take any extra steps to account for $r$-mode rather than
mass-quadrupole emission; rather the results could be interpreted in terms of
$r$-modes~\cite{Owen_2010}.

Caride \textit{et al.}~\cite{PhysRevD.100.064013} showed that dedicated searches for
$r$-modes from known pulsars are feasible. The key is to search the right range
of frequencies and frequency derivatives, which are significantly different from
the more often considered case of mass-quadrupole emission. Not surprisingly,
as with other \ac{GW} searches for pulsars, the Crab is the first prospect to
beat the spin-down limit using \ac{LIGO} data. The spin-down limit assumes that
all the rotational energy is lost in the form of \acp{GW}, and represents a
milestone a search must beat to have a chance of detection. Recently Fesik and
Papa~\cite{Fesik2020} first published an $r$-mode pulsar search similar to that
proposed by Caride \textit{et al.}~\cite{PhysRevD.100.064013} for another pulsar which
is interesting for different reasons; but they did not beat its spin-down limit.
The Crab is a relatively nearby pulsar with one of the fastest known spin-down
rates, and thus it has one of the highest spin-down limits
($2.5\times10^{-24}$). Its rotational frequency changes at the rate
$-3.69\times10^{-10}$\,Hz/s~\cite{JodrellBankObservatory}. The Crab's pulse
timing is constantly measured by \ac{EM} observations, so the spin frequency
evolution of the Crab is well known during the LIGO \ac{O1} and \ac{O2}
observation runs.

For various reasons we know that the Crab is not emitting \ac{GW} at or near the
spin-down limit. Observations of the nebula indicate that most of the
rotational energy is lost powering the nebula via synchrotron and inverse
Compton radiation from the pulsar wind, and a few percent is lost in the narrow
light beam~\cite{B_hler_2014}. Recent \ac{GW} searches~\cite{Abbott_2019, O3}
have concluded that less than 0.01\% of the Crab's rotational energy loss is
through mass-quadrupole gravitational radiation. The braking index $n$ of the
Crab (the logarithmic derivative of its spin-down with respect to frequency) is
2.519.  This is closer to the $n=3$ expected for magnetic dipole
radiation~\cite{Lyne_2014} than to the $n=7$ expected for $r$-mode
emission~\cite{Owen_1998}. Such a low braking index is another indicator that
\ac{GW} emission is a small fraction of the spin-down limit. How small has not
been quantified in a model-independent way or for $r$-modes. But
Palomba~\cite{Palomba2000} found that, for a class of reasonable mass-quadrupole
models, the measured braking index of the Crab means it is emitting \ac{GW} at
least a factor of a few below the spin-down limit. We take this to suggest that
any $r$-mode \ac{GW} signal from the Crab must be at least a factor of a few (in
strain) below the spin-down limit.

Nevertheless, Caride \textit{et al.}~\cite{PhysRevD.100.064013} showed that a search of
the Crab with interesting sensitivity is feasible, and we confirm this.
We performed searches for the Crab in \ac{O1} and \ac{O2} data (the publicly
available LIGO data sets) using the matched filtering-based technique known as
the $\mathcal{F}$-statistic.
While we did not find any evidence of a \ac{GW} signal, we were able to set
upper limits beating the spin-down limit by a significant amount over a wide
parameter space.

\section{R-modes search method} 

Our search is based on a matched filtering technique  known as the 
$\mathcal{F}$-statistic. Developed by ~\citet{Jaranowski_1998} for a single interferometer and by
~\citet{PhysRevD.72.063006} for multiple interferometers, it is a statistical procedure
for the detection of continuous gravitational waves. The
$\mathcal{F}$-statistic accounts for the Doppler and amplitude modulation due
to the daily rotation and orbital motion of Earth in a computationally efficient
manner. In the presence of a signal,
$2\mathcal{F}$ is a non-central chi squared distribution with four degrees
of freedom and the non-central parameter is proportional to the signal. The
$\mathcal{F}$-statistic is the log of the likelihood function maximized over the
unknown strain, phase constant, inclination angle and polarization angle. 
The main issue, as described by Caride \textit{et al.}~\cite{PhysRevD.100.064013}, is to
find the ranges of frequencies and frequency derivatives to search.

Pulsars are slowly spinning down due to \ac{GW} and other losses of rotational
energy. The evolution of the rotational frequency $\nu$ of a spinning down neutron star in the
frame of the solar system barycenter is approximated by: 
\begin{equation}
 \nu(t) = \nu\left( t_0 \right) + \dot{\nu}\left( t_0 \right) \left( t-t_0
\right) + \frac{1}{2} \ddot{\nu}\left( t_0 \right) \left( t-t_0 \right)^2,
\end{equation} 
where $t_0$ is a reference time, often the start time of the observation.
Generally $\dddot{\nu}$ is not needed for a less than a year of integration
time~\cite{PhysRevD.100.064013}. The frequency evolution is precisely known from
electromagnetic observations. It might depart from the above approximation due
to timing noises or glitches. Glitches are sudden increases in spin frequency
followed by exponential recovery to the pre-glitch
frequency~\cite{Espinoza_2011}.  Since the Crab glitched during O2, we divided
the search into two roughly equal stretches. More on the Crab pulsar timing for
our search will be discussed in section~\ref{timing}.  Timing noise is residual
phase wandering of pulses relative to the normal spin down model.  Timing noise
will deviate \ac{GW} phases from Taylor series for time scales of 1 year and
longer~\cite{Jones_2004}. Assuming that the \ac{GW} timing noise is similar to the
one observed in EM pulses, the mismatch of the Crab ephemeris during the LIGO S5 run
from the model with no timing noise is less than
$1\%$~\cite{PhysRevD.91.062009}.  So, for all our searches (four months of
data), timing noises should not significantly mismatch the templates from the
signal.

For a Newtonian star with spin frequency $\nu$, the $r$-mode \ac{GW} frequency
is approximately $f=\frac{4}{3}\nu$. The frequency ratio deviates from $4/3$
when corrections due to general relativity, rapid rotation, superfluidity,
magnetic fields and core-crust coupling are considered~\cite{Idrisy:2014qca}.
For fast rotating neutron stars, the elastic restoring force on the crust
couples with the Coriolis restoring force on the $r$-modes resulting in avoided
crossings~\cite{Levin_2001}. These are small frequency bands where the simple
relation between $f$ and $\nu$ is drastically altered as modes change identities.
Away from avoided crossings, the $r$-mode frequency as a function of spin
frequency is approximately
\begin{equation}
f=A\nu-B\left(\frac{\nu}{\nu_K}\right)^2, 
\end{equation} 
where $\nu_K$ is the Kepler frequency. Here we neglect other effects so that $A$
and $B$ are the general relativistic and slow rotation corrections.  
As the modes are retrograde in the rotating frame but prograde in an inertial frame,
the general relativistic correction decreases the mode frequency in a co-rotating
frame but increases in an inertial frame. Since the modes restoring force is the 
Coriolis force which is determined by the fluid angular velocity relative to the
local inertial frame, the modes oscillate less
rapidly when the frame dragging effect is
significant~\cite{Lockitch:2000aa,1968ApJ...153..807H}. The effect of rapid
rotation is to decrease the mode frequency in an inertial frame.  The
gravitational redshift decreases the mode frequency as a clock will be ticking
slower in the relativistic case as seen by a distant observer. 

The ranges of $A$ and $B$ is chosen as follows: We consider a range of compactness of
neutron stars ($0.11\leq M/R\leq 0.31$)~\cite{Idrisy:2014qca} which comes from
the uncertainty in the equation of state. The range on compactness gives a range
$1.39\leq A \leq 1.57$. The range of $B$ is derived from the relation of $f/\nu$
to the ratio of the rotational energy ($T$) to the gravitational potential energy
($W$)~\cite{Yoshida:2004gk}. For different polytropic indices and compactnesses,
the range is 1.23--1.95 times the rotational parameter ($T/W$).
\citet{PhysRevD.100.064013}
converted $T/W$ into $(\nu/\nu_k)^2$ using $M/R=0.1$ which gives a maximum
value of $B$ = 0.195. The minimum value of $B$ is not well known, so it is taken
to be zero. 

Our search covers a parameters ($f$, $\dot{f}$, $\ddot{f}$). 
The parameter ranges for our searches are given by \citet{PhysRevD.100.064013} based on
the consideration above:
\begin{eqnarray}
\nu \left[ A_{\min} - B_{\max} \left( \nu/\nu_K \right)^2 \right] & \le & f
\le \nu\, A_{\max},
\\
\dot\nu \left[ f/\nu - 2B_{\max} \left( \nu/\nu_K \right)^2 \right] & \le &
\dot f \le \dot\nu f/\nu,
\\
0 & \le & \ddot f \le \ddot\nu f/\nu,
\end{eqnarray}
where, $f$ is spin frequency, $\dot{f}$ and $\ddot{f}$ is the first and second
spin derivative. And $A_{min}$=1.39, $A_{max}$=1.57 and $B_{max}$=0.195.

We use the parameter space metric $g_{ij}$ to control the computational
cost of our search. 
The distance between two templates is given by minimal
match ($\mu$) which is the loss in signal to noise ratio when signal falls
exactly between two theoretical waveforms~\cite{Owen_1996}.  
The template number is given by dividing the proper volume of the parameter
space by the proper volume per template~\cite{PhysRevD.100.064013}:
\begin{equation}\label{templatescalc}
\frac{\sqrt{g} \nu \left| \dot\nu \right| \ddot\nu B_{\max} \left(\nu/\nu_K
\right)^2 \left[ A_{\max}^2 - A_{\min}^2 \right]}{\left( 2 \sqrt{\mu/3}
\right)^3}
\end{equation}
where g is the metric determinant.
The parameter space metric is given by~\cite{Wette:2008hg,Owen_1996}:
\begin{equation}
g_{ij}=\frac{4\pi^2T^{i+j+2}(i+1)(j+1)}{(i+2)!(j+2)!(i+j+3)}
\end{equation}\\
where $i$, $j$ stand for parameters $(f,\dot{f},\ddot{f})=(0,1,2)$ and $T$ is the time spanned by the observation. 

We look at the highest values of the $\mathcal{F}$-statistic
($2\mathcal{F}^*$) that survived the automated vetoes. The probability that a
given value of $2\mathcal{F}^*$ will be observed when no signal is present is
given by~\cite{Abadie_2010}: 
\begin{equation} 
p(N;2\mathcal{F}^*)= Np(\chi^2_4;2\mathcal{F}^*)[\mathrm{cdf}(\chi^2_4;2\mathcal{F}^*)]^{N-1} 
\end{equation} 
where $N$ is the number of templates (assuming statistically independence) and $p(\chi^2_4;2\mathcal{F}^*)$ is the
central $\chi^2$ distribution with four degrees of freedom. A high
$2\mathcal{F}$ is not enough to claim the detection, as instrumental lines might
act as a periodic signal. A search is followed by the F-scan veto and
an interferometer consistency veto that remove narrow band instrumental
lines~\cite{Lindblom_2020}. The F-scan veto looks for lines by normalizing the
\acp{SFT}  using a running median of 51 frequency bins and looking for
excursions. Then the interferometer
veto is applied which discards the templates when the joint $2\mathcal{F}$
value from the two interferometers is less than for either single interferometer.

%where i,j are the indices for frequency and its derivatives
%labeled as
%$(f,\dot{f},\ddot{f})=(0,1,2)$.~\cite{PhysRevD.100.064013}\cite{Wette:2008hg}
%For the pulsars and integration time of interests, both $f$ and $\dot{f}$
%parameters are 
%necessary to build  the \ac{GW} template.  The requirement of $\ddot{f}$ depends
%on the mismatch between the signals and waveform when the $\ddot{f}$ is ignored.
%For the continuous \ac{GW} search the maximum mismatch between template and
%signal is 0.2, i.e. we won't lose more than $20\%$ of signal to noise ratio
%$\ddot{f}$ is necessary. If the mismatch  $g_{22}\ddot{f}^2 >0$ $\ddot{f}$ is
%necessary. Using the maximum $\ddot{f}$ observed during our search,and with $T=
%10^7s(O1)$ $g_{22}\ddot{f}$ = $5.5$, so $\ddot{f}$ is a requirement. 
	

\section{The Crab pulsar} \label{timing}
The Crab pulsar is the remnant of a supernova explosion seen by
Chinese astronomers in the year 1054 AD. This is our first target due to its
high spin down limit which is well above the LIGO \ac{O1} noise
curve~\cite{PhysRevD.100.064013}. The rotational energy loss of the
pulsar is given by $\dot{E}=4\pi^2I_{zz}\nu\dot{\nu}\approx
4.33\times10^{31}$\,W, 
where ($I_{zz}$ = $10^{38}$\,$\mathrm{kgm^2}$) is the principal moment of inertia ~\cite{Abbott_2008}. 
The spin-down power of the Crab is high enough that, even if
$r$-mode gravitational wave emission is only responsible for a small fraction of
it, the \acp{GW} could be detectable.
 
The sky location of the Crab pulsar in J2000 coordinates is~\cite{1993MNRAS.265.1003L} 
\begin{equation}
\begin{aligned} 
\alpha =05^h34^m31.94^s\\ 
\delta=
+22^\circ00^\prime52.1^{\prime\prime} 
\end{aligned} 
\end{equation}
 
The spin frequency and its derivative~\cite{1993MNRAS.265.1003L} were obtained
from the Jodrell Bank observatory monthly ephemeris and interpolated to the
start dates of the LIGO run from the nearest dates of 09/16/2015 (\ac{O1}),
11/23/2016 (early O2) and 04/16/2017 (late O2). There were no glitches of the
Crab during \ac{O1}, so the whole run can be coherently integrated. The Crab
glitched at 2017-03-27 22:04:48.000 UTC~\cite{Espinoza_2011} during the \ac{O2}
run, approximately halfway through. So we divided the \ac{O2} run into two
roughly equal stretches.  The glitch was comparatively small
($\frac{\Delta\nu}{\nu}= 2.14\times 10^{-9}$)~\cite{Espinoza_2011} and the
relaxation time period for the glitches are not well known, so we started the
search of late \ac{O2} a few hours afterwards.

Fast spinning down pulsars are contaminated by timing
noise~\cite{Lyne1992GlitchesAP} specially when $\ddot{\nu}$ is negative. During
our observational time, the monthly $\ddot{\nu}$ measurement was fluctuating.
This might be due to external torque from magnetosphere. So, we chose the
maximum $\ddot{\nu}$ as maximum value observed during the observational time and
minimum $\ddot{\nu}$ as zero. The extra number of templates required by
including $\ddot{\nu}$ is factor of a few, so the searches will not be too
expensive if we used $\ddot{\nu}$ ranging from $0$ to the maximum observed value of
$\ddot{\nu}$. 
\begin{table*} 
\centering
\footnotesize
\begin{tabular}{ lccccc} 
\hline
\hline
\textrm{Search} & \textrm{Start time(UTC)} & \textrm{End time(UTC)} &
\textrm{Time span(days)} &  \textrm{Obs.Time(days)} & \textrm{SFTs}\\[5pt]
%\colrule
\hline
\textrm{\ac{O1}} & 09/12/15(06:03:57) & 01/19/16(15:34:47) & 129.36 & 66.55
&6389\\[5pt]
\textrm{\ac{O2} (early)} & 11/30/16(18:01:57) & 03/27/17(16:28:25) &  116.93 &
56.167 & 6186\\[5pt] 
\textrm{\ac{O2} (late)} & 03/28/17(23:47:38) & 08/25/17(21:59:34) & 149.92 & 74.9 &
8035 \\ 
\hline
\hline
\end{tabular} 
\caption{The start and end times of the three searches.  The observational time
is the total duration of \acp{SFT} (1800s each) divided by the number of
interferometers (two).}
\label{table:Time} 
\end{table*} \\

\begin{table} 
%\begin{ruledtabular}
\centering
\begin{tabular}{lccc } 
\hline
\hline
\textrm{Search} & \textrm{$\nu$} &\textrm{$\dot{\nu}$} &
\textrm{$\ddot{\nu}$}\\[5pt] 
%\colrule
\hline
\ac{O1} & $29.66181$ & $-3.69383\times10^{-10}$ &  $2.41\times10^{-20}$\\[5pt] 
\ac{O2} (early) & $29.64761$ & $-3.689673\times10^{-10}$ &
$1.92\times10^{-20}$\\[5pt] 
\ac{O2} (late) & $29.6438$ & $-3.688438\times10^{-10}$ &  $2.5\times10^{-20}$\\ 
\hline
\hline
\end{tabular} 
\caption{Timing of the Crab pulsar at the beginning of our three different searches.
The timing is measured by Jodrell Bank Observatory~\cite{1993MNRAS.265.1003L}
and interpolated to the start time of each LIGO run. The displayed $\ddot{\nu}$ is the
maximum monthly value observed during each run.}
\label{table:timing} 
\end{table} 

\section{Search Implementation} 
We used data from the \ac{GWOSC}, starting with time-domain strain data sampled
at 4\,kHz.  We downloaded all such \ac{O1} data for the official duration of the
run (from GPS times 1126051217 to 1137254417) from both interferometers (H1 and
L1), gating on only ``CBC CAT1'' vetoes. These indicate disastrous conditions
for the instruments, such as loss of laser power. We ignored the other vetoes
used in searches for binary black holes and neutron stars because they are aimed
at short-duration disturbances which are not an issue for continuous wave
searches.  We then used the code \texttt{lalapps\_MakeSFTDAG} from LALSuite to
generate 1800\,s long \acp{SFT}, version\,2 format, high pass filtered with a knee
frequency of 7\,Hz and windowed with the default Tukey window. This produced
6,389 \acp{SFT} for \ac{O1} (3,474 from H1 and 2,915 from L1). A similar
procedure for \ac{O2} produced 14,231 \acp{SFT} (7,242 from H1 and 6,989 from
L1). Our $r$-mode search used a modified version of the Supernova remnant
code pipeline\cite{Aasi_2015} and the ``Texas Tech University High Performance Computing
Center'' Quanah cluster for computational power.

The frequency bands of the searches were 41.2--46.6\,Hz (O1) and 41.1--46.6\,Hz
(O2). The minimum and maximum frequencies are rounded down and up respectively.
This is because upper limit bands were chosen to be uniform  0.1\,Hz. We used a
bank of templates with minimal match 0.2. The ideal template numbers for O1,
early O2 and late O2 are $2.27\times 10^9$, $9.71\times 10^8$ and $4.02\times
10^9$ respectively. The actual search produced $1.72\times 10^{10}$, $1.17\times
10^{10}$ and $3.62 \times 10^{10}$ templates. The number of templates for the
search is around an order of magnitude greater than the ideal templates
calculated. This is because the code uses extra templates to cover the
boundaries outside the parameter space~\cite{Abadie_2010}. The template banks
are made through a three dimensional tiling over a $f$, $\dot{f}$ and
$\ddot{f}$~\cite{Wette:2009uea}. The templates are narrow in the $\ddot{f}$
direction, which leads to an especial lot of overcoverage. O1, early O2 and late
O2 searches used computational times of 1584, 944 and 3892 core hours on Quanah
respectively.

\begin{table*} 
%\begin{ruledtabular}
\centering
\begin{tabular}{lccc} 
\hline
\hline
\textrm{Search} & \textrm{Search bands} & \textrm{Vetoed bands} &
\textrm{Fraction of band searched}\\[5pt]   
\hline
%\colrule  
\ac{O1} & 5.4 Hz & 1.286 Hz & 0.76\\[5pt] 
\ac{O2} (early)& 5.5 Hz  & 1.38 Hz & 0.754\\[5pt] 
\ac{O2} (late) & 5.5 Hz & 1.33 Hz & 0.758\\  
\hline
\hline
\end{tabular} 
%\end{ruledtabular}
\caption{Total vetoed frequency band of the search due to instrumental
lines.}\label{table:veto} 
\end{table*} 

 
The F-scan veto threshold was fixed to twenty standard
deviations. The total frequency bands vetoed are 1.29\,Hz (O1), 1.38\,Hz
(early O2) and 1.33\,Hz (late O2). This is about $25\%$ (O1), $25\%$ (early O2)
and $24\%$ (late O2) of the total frequency band searched. We set the false
alarm probability as $5\%$. If the $2\mathcal{F}$
value from the search has less than 5\% chance of being drawn from
$p(2\mathcal{F}^*)$ then it is the first step to claim the \ac{GW} signal. None
of the non-vetoed templates had a $2\mathcal{F}$ value that beat the 95\%
significance level, so no detection was made in our searches.

\section{Upper limits}
In the absence of a detection, we set upper limits on gravitational wave
emission. The method is similar to ~\citet{Lindblom_2020}. Upper limits are the
weakest signal that can be detected from our search with certain probability.
The frequentist upper limit depends on the observed $2\mathcal{F}$ value and the
choice of confidence level. For our case, the upper limit is a $95\%$ confidence
level, i.e. for a minimum $h_o$, $2\mathcal{F} > 2\mathcal{F}^*$ $95\%$ of the
time~\cite{Romano_2017}. This means we set the false dismissal rate to $5\%$,
and the loudest $2\mathcal{F}$ observed (even if vetoed) will set the false alarm
rate~\cite{Aasi_2015}. The upper limit bands are chosen to be small enough for
the interferometer noise to stay constant. The upper limit band for all the
searches is 0.1\,Hz. 

First, we use a computationally inexpensive Monte Carlo to determine the loudest
non-vetoed $2\mathcal{F}$ and find a $95\%$ confidence level intrinsic strain
($h_o$) that gives such a $2\mathcal{F}$. The value of $h_o$ is updated using
Newton-Raphson root-finding method, if the
confidence level is less than or greater than the $95\%$~\cite{ Wette:2009uea}.
Then we use computationally expensive
($20- 30\%$ of the cost of search)~\cite{Aasi_2015} software injection searches
to make the validation of the upper limit. 1000 fake signals with fixed $h_o$
(that was determined in previous step) and variable $f,\dot{f},\ddot{f}$ and
other unknown parameters are injected in each upper limit band to check the
validity of $h_o$. The maximum value of $2\mathcal{F}$ obtained from the
injection is compared to the $2\mathcal{F}^*$ from the search results. If
$2\mathcal{F}> 2\mathcal{F}^*$ then this $h_o$ will be our upper limit.  The
upper limits of intrinsic strain and $r$-mode amplitude for our three searches
are
shown in figure ~\ref{fig:UL}.  The best $95\%$ confidence level upper limits on
intrinsic amplitude strain of our search are $1.61\times 10^{-25}$ (O1),
$1.45\times 10^{-25}$ (early O2) and $1.20\times 10^{-25}$ (late O2).

We can also set upper limits on the $r$-mode amplitude.  
The derivation of $r$-mode amplitude ($\alpha$) is given by \citet{Owen_2010},
who took the fiducial value of moment of inertia ($I_{zz}$ = $10^{38}$\,kg\,m$^2$)
and typical $M=1.4\,M_\odot$. We convert the upper limit on
intrinsic strain ($h_0$) to a r-mode amplitude ($\alpha$).   
\begin{equation} 
\alpha
=0.028\left(\frac{h_o}{10^{-24}}\right)\left(\frac{r}{1\,kpc}\right)\left(\frac{100\,Hz}{f}\right)^3
\end{equation} 

The best $r$-mode amplitude upper limits for our different searches are $0.0898$
(O1), $0.082$ (early O2) and $0.067$ (late O2). 


\begin{figure}[h!] 
\includegraphics[width=3in]{figure/Crabupper.png}
\includegraphics[width=3in]{figure/modeamplitude.png}

\caption{Comparison of upper limits on intrinsic strain (left) and $r$-mode
amplitude (right) for our three searches. The late O2 search is the most
sensitive one as it has the most observational
time and detectors were more sensitive later.} 
\label{fig:UL}
\end{figure}


\section{Conclusion} 

Our search for the Crab pulsar didn't detect $r$-mode \acp{GW} from \ac{LIGO}
\ac{O1} and \ac{O2} run. The second run of \ac{O2} has the better sensitivity
mostly due to the longer observational time and both interferometers were taking
data simultaneously. Although the spin down limit of Crab is higher, the \ac{GW}
amplitudes might not be high enough to beat the LIGO sensitivity. The longer
observational time and better sensitivity might be able to detect continuous
\ac{GW} in future LIGO run.  We will extend our search of the Vela pulsar which
beat the spin down limit for the LIGO \ac{O2} run.  

J0537-6910 might be another interesting pulsar as the average inter-glitch
braking index is closer to 7 when subtracting the exponential decay component of
$\dot{\nu}$ during the glitch recoveries~\cite{Andersson_2018,Ferdman_2018};
this means it might be spinning down due to $r$-mode \ac{GW} emission. We do not
have a timing of this pulsar during the \ac{O1} and \ac{O2} runs and it glitches
almost every 100 days, so longer observational times will mismatch the signal
with the theoretical templates. The $r$-mode \ac{GW} from J0537-6910 pulsar has
recently been searched by ~\citet{Fesik2020}, using the timing from Rossi X-ray
Timing Explorer from Nov 2011 and interpolating to the LIGO \ac{O1} and \ac{O2}
runs. They have considered two scenarios where $r$-modes are active or shut off
just at the beginning of observation. The search didn't find any evidence of
$r$-modes.  

Some nearby pulsars J0437-4715 and J0711-6830 are important due to its proximity
and its frequency lies where \ac{LIGO} is more sensitive. But their spin down
frequencies are very low $(10^{-15})$, so the spin down limit only beats LIGO A+
noise curve. J0437-4715 is a billion year old pulsar but its temperature is
more than expected for an old pulsar, so some astronomers think there might be
some internal heating mechanism~\cite{Durant_2012} increasing the temperature of
the neutron star, and $r$-mode heating might be one of the factors. Both
J0437-4715 and J0711-6830 spin down limits are well below the LIGO \ac{O1} and
\ac{O2} noise curve so the search might only be important for future LIGO runs.  

\section*{acknowledgments}
This work was supported by NSF grant PHY-1912625.
This research has made use of data, software and/or web tools obtained from the
Gravitational Wave Open Science Center (\url{https://www.gw-openscience.org}), a
service of LIGO Laboratory, the LIGO Scientific Collaboration and the Virgo
Collaboration. LIGO is funded by the U.S. National Science Foundation. Virgo is
funded by the French Centre National de Recherche Scientifique (CNRS), the
Italian Istituto Nazionale della Fisica Nucleare (INFN) and the Dutch Nikhef,
with contributions by Polish and Hungarian institutes. The authors acknowledge
the High Performance Computing Center (HPCC) at Texas Tech University for
providing computational resources that have contributed to the research results
reported within this paper (\url{http://www.depts.ttu.edu/hpcc/}).


 
%%%%%%%%%%%%%%%%%%%%%%%%%%%%%%%%%%%%%%%%%%%%%%%%%%%%%%%%%%%%%%%%%%%%%%%%%%%%%%%%%%%%%%%%%
%%%%%%%%%%%%%%%%%%%%%%%%%%%%%%%%%%%%%%%%%%%%%%%%%%%%%%%%%%%%%%%%%%%%%%%%%%%%%%%%%%%%%%%%%
%%%                                                     END OF SIXTH CHAPTER
%%%%
%%%%%%%%%%%%%%%%%%%%%%%%%%%%%%%%%%%%%%%%%%%%%%%%%%%%%%%%%%%%%%%%%%%%%%%%%%%%%%%%%%%%%%%%%
%%%%%%%%%%%%%%%%%%%%%%%%%%%%%%%%%%%%%%%%%%%%%%%%%%%%%%%%%%%%%%%%%%%%%%%%%%%%%%%%%%%%%%%%%


%This thesis presents the work on $r$-mode \ac{GW} searches from a Crab pulsar
%for the \ac{LIGO} \ac{O1}and \ac{O2} run. We also showed the right frequency
%parameter and estimated computational cost for the $r$-mode search. In chapter 1
%we started discussing about some of the historic discovery of \acp{GW} by
%\ac{LIGO} and Virgo interferometer. Then in chapter 2 we derived the Einstein
%field equation and showed the perturbation in flat space-time travels at speed
%of light known as gravitational waves. Then, we described the polarization of
%\acp{GW} and energy radiated from \ac{GW} emission.  The principal of \acp{GW}
%detector is based on the Michelson interferometer. The interferometer are
%affected by various noise sources, we mentioned some origin of the noises and
%the steps to extract signals that are buried in noise. We gave a details on
%neutron star structure and emission of \ac{GW} radiation due to mountains on
%neutron stars.
%
%In chapter 3 we talked about the $r$-mode emission from a rotating neutron star.
%R-modes are damped by the viscous mechanism but are unstable to \ac{GW}
%radiation. We discussed the dependence of $r$-mode
%instability window to the temperature and spin frequency of neutron star. We
%later explained the $r$-mode frequency deviate from Newtonian case due to the
%correction from general relativity and rapid rotation. 
%
%Chapter 4 shows the correct frequency band and frequency derivatives to search
%for $r$-modes. The right parameter is derived using the first and second order
%correction in $r$-modes frequencies. The correction includes but not limited to
%unknown compactness of neutron star and rapid rotation correction.  The chapter
%also includes the spin down limit of selected pulsar that beats the LIGO noise
%curve. Crab has the best spin down limit in terms of intrinsic strain and it can
%be the first pulsar to search for $r$-mode \acp{GW}. 
%
%In chapter 5 we discussed the method of extracting continuous wave signals from
%the \ac{LIGO} noisy data. We started by the interferometer response to the
%\ac{GW} signal and how the continuous wave signal are the functions of Doppler
%and Amplitudes parameter. Then we gave a brief introduction of maximum
%likelihood functions and how the signals are maximized to the unknown amplitude
%parameters. We gave details of construction of \ac{SFT} and waveforms
%from the interferometer data for the continuous waves. To claim the signal
%detection, the signal should be above a certain confidence level. If there is a
%signal in the data, the expected $\mathcal{F}-statistic$ value will be the non
%central chi squared deviation, where the non central parameter will be
%proportional to the signal. 
%
%The next chapter is about the $r$-mode \ac{GW} search from the Crab pulsar. On
%chapter 4 we gave a details why Crab is our first candidate. The Crab pulsar
%timing has been constantly monitored by the Jodrell Bank Observatory. We
%searched for a first two observing run of a Advanced \ac{LIGO}. Crab glitched in
%the middle of the second run, so we divided the second \ac{LIGO} run into two
%halves. We described the process of data analysis in our search that are
%explained in detail in chapter 5. We did not find any evidence of \ac{GW} so, we
%set up the upper limit for the three different searches.   
%
%With the improvement of \ac{LIGO} sensitivity and computational power in few
%years, the detection of continuous gravitational wave will be more probable. The
%better understanding of $r$-mode instability and saturation amplitude will also
%help to know the energy of the $r$-mode gravitational wave emission. Since, the
%detection of continuous \ac{GW} will help us to understand the neutron star
%structure, this will benefit not only the astronomer but nuclear and particle
%physicist too.
%%%%%%%%%%%%%%%%%%%%%%%%%%%%%%%%%%%%%%%%%%%%%%%%%%%%%%%%%%%%%%%%%%%%%%%%%%%%%%%%%%%%%%%%%
%%%%%%%%%%%%%%%%%%%%%%%%%%%%%%%%%%%%%%%%%%%%%%%%%%%%%%%%%%%%%%%%%%%%%%%%%%%%%%%%%%%%%%%%%
%%% 						End of Fifth Chapter 											  %%%
%%%%%%%%%%%%%%%%%%%%%%%%%%%%%%%%%%%%%%%%%%%%%%%%%%%%%%%%%%%%%%%%%%%%%%%%%%%%%%%%%%%%%%%%%
%%%%%%%%%%%%%%%%%%%%%%%%%%%%%%%%%%%%%%%%%%%%%%%%%%%%%%%%%%%%%%%%%%%%%%%%%%%%%%%%%%%%%%%%%







%%%%%%%%%%%%%%%%%%%%%%%%%%%%%%%%%%%%%%%%%%%%%%
%Backmatter -- Bibliography, appendices, etc.%
%%%%%%%%%%%%%%%%%%%%%%%%%%%%%%%%%%%%%%%%%%%%%%
\backmatter


%%%%%%%%%%%%%%%%%%%%%%%%%%%%%%%%%%%%%%%%%%%%%%%%%%%%%%%%
%Bibliography:  Use BibTeX 							   %
%%%%%%%%%%%%%%%%%%%%%%%%%%%%%%%%%%%%%%%%%%%%%%%%%%%%%%%%
%\bibliographystyle{abbrvnat}
\bibliographystyle{chicago}
\addcontentsline{toc}{chapter}{\textbf{References}}
\bibliography{thesis}


%%%%%%%%%%%%%%%%%%%%%%%%%%%%%%%%%%%%%%%%%%%%%
%        APPENDIX A                         %
%%%%%%%%%%%%%%%%%%%%%%%%%%%%%%%%%%%%%%%%%%%%%

\end{document}




